\section{Applications}
\label{sec:uses}

\subsection{Nyström approximation}

Standard kernel machines are often impossible to use in large-scale applications because of their high computational cost associated with the kernel matrix $\rmK$ which has $O(n^2)$ storage and $O(n^2d)$ computational complexity: $\forall i,j \in [\![n]\!], \rmK_{i,j} = k(\rmX_i, \rmX_j)$. A well-known strategy to overcome this problem is to use the Nyström method which computes a low-rank approximation of the kernel matrix on the basis of some pre-selected landmark points. 

Given $K \ll n$ landmark points $\{\rmU_i\}_{i=1}^{K}$, the Nyström method gives the following approximation of the full kernel matrix:
%
\begin{equation}
 \label{eq:nystrom}
 \rmK \approx \tilde\rmK = \rmC\rmW^\dagger\rmC^T,
\end{equation}
%
with $\rmW \in \R^{K \times K}$ containing all the kernel values between landmarks: $\forall i,j \in [\![K]\!]~ \rmW_{i,j} = k(\rmU_i, \rmU_j)$; $\rmW^\dagger$ being the pseudo-inverse of $\rmW$ and $\rmC \in \R^{n \times K}$ containing the kernel values between landmark points and all data points: $\forall i \in [\![n]\!], \forall j \in [\![K]\!]~ \rmC_{i, j} = k(\rmX_i, \rmU_j)$.

\subsubsection{Efficient Nyström approximation}

A substantial amount of research has been conducted toward landmark point selection methods for improved approximation accuracy \cite{kumar2012sampling} \cite{musco2017recursive}, but much less has been done to improve computation speed. In \cite{si2016computationally}, the authors propose an algorithm to learn the matrix of landmark points with some structure constraint, so that its utilisation is fast, taking advantage of fast-transforms. This results in an efficient Nyström approximation that is faster to use both in the training and testing phases of some ulterior machine learning application.

Remarking that the main computation cost of the Nyström approximation comes from the computation of the kernel function between the train/test samples and the landmark points, \cite{si2016computationally} aim at accelerating this step. In particular, they focus on a family of kernel functions that has the following form:
%
\begin{equation}
 K(\rvx_i, \rvx_j) = f(\rvx_i) f(\rvx_j) g(\rvx_i^T\rvx_j),
\end{equation}
%
where $f: \R^d \rightarrow \R$ and $g: \R \rightarrow \R$. They show that this family of functions contains some widely used kernels such as the Gaussian and the polynomial kernel. Given a set of $K$ landmark points $\rmU \in \R^{K \times d}$ and a sample $\rvx$, the computational time for computing the kernel between $\rvx$ and each row of $\rmU$ (necessary for the Nyström approximation) is bottlenecked by the computation of the product $\rmU\rvx$. They hence propose to write the $\rmU$ matrix as the concatenation of structured $s = K / d$ product of matrices:
%
\begin{equation}
 \rmU = \left[ \rmV_1 \rmH^T, \cdots, \rmV_s\rmH^T  \right]^T,
\end{equation}
%
where the $\rmH$ is a $d \times d$ matrix associated with a fast transform such as the \textit{Haar} or \textit{Hadamard} matrix, and the $\rmV_i$s are some $d \times d$ diagonal matrices to be either chosen with a standard landmark selection method or learned using an algorithm they provide.

Depending on the $\rmH$ matrix chosen, it is possible to improve the time complexity for the computation of $\rmU\rvx$ from $O(Kd)$ to $O(K \log{d})$ (\textit{Fast Hadamard transform}) or $O(K)$ (\textit{Fast Haar Transform}).

\subsubsection{Q-means in Nyström}

We propose to use our Q-means algorithm in order to learn directly the $\rmU$ matrix in the Nyström approximation so that the matrix-vector multiplication $\rmU \rvx$ is cheap to compute, but the structure of $\rmU$ is not constrained by some pre-defined transform matrix. We propose to take the objective $\rmU$ matrix as the K-means matrix of $\rmX$ since it has been shown to achieve good reconstruction accuracy in the Nyström method.


Our algorithm could allow one to obtain an efficient Nyström approximation, while keeping the quality of the K-means landmark points which are expressed as a factorization of sparse matrix.  

%{RBF networks}

%Besoin d'éclaircir les liens avec RBF networks

%\subsection{nearest-neighbours}

%Besoin d'éclaircir les liens avec nearest neighbours
