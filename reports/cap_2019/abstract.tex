\begin{abstract}
OBSOLETE, basculer sur fichier \texttt{neurips2019\_qmeans.tex}
%Beyond its popularity for clustering, the K-means algorithm is a pivotal procedure for other core machine learning and data analysis techniques such as indexing, nearest-neightbors prediction, as well as for more specific approaches like the Nyström approximation for kernel machines.
%
%In this paper, we propose {\em Q-means}: an accelerated version of $K$-means that stems from recent advances in optimization to learn the centroid matrix as a product of sparse matrices.
%This decomposition provides a structure similar to that of fast transforms (e.g., Fourier, Hadamard) in order to benefit from its computationnal efficiency while being adapted to the training data.
%Indeed, the complexity of the matrix-vector product between the factorized $K \times D$ matrix $\mathbf{U}$ and any vector is lowered from $\mathcal{O}(KD)$ to $\mathcal{O}(P+Q \log Q)$, with $Q=\min (K, D)$ and $P=\max (K, D)$.
%This dramatic acceleration is beneficial whenever a point is assigned to a cluster, i.e., at prediction time and in the assignation step at learning time.
%In addition, we show that the computational overhead due to the decomposition procedure does not penalize the computational cost of the learning stage, 
%which may be faster than the traditionnal Lloyd algorithm depending on the context.
%
%Finally, we provide discussions and numerical experiments that show the versatility of the proposed computationally-efficient Q-means algorithm.
%
%\addVE{Remarque: on ne mentionne pas la qualité de l'approximation qu'on obtient en remplaçant K-means par Q-means?!} \addLG{je crois qu'on peut assez peu s'exprimer à ce sujet sans borne...}
%
%\addVE{Remarque sur la complexité $\mathcal{O}(p+q \log q)$: on a $\log q$ facteurs dont un de taille $p\times q$ (ou l'inverse) a $\mathcal{O}(p)$ valeurs non-nulles et tous les autres de taille $q \times q$ ont $\mathcal{O}(q)$ valeurs non-nulles.}
%

%on core machine learning models and techniques such as Nyström approximation, Gaussian mixtures, K-nearest neighbors.

%As evoked before, we show how our algorithm could be used in the context of the Nyström approximation. % to speed up its use in subsequent kernel machines.

% This paper is a description of work in progress and as such does not contain experimental validation. We give the proposed algorithm with convergence proof, and discuss foreseen experiments and a scope of use.
\end{abstract}
