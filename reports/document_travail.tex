\documentclass{article}

% if you need to pass options to natbib, use, e.g.:
%     \PassOptionsToPackage{numbers, compress}{natbib}
% before loading neurips_2018

% ready for submission
% \usepackage{neurips_2018}

% to compile a preprint version, e.g., for submission to arXiv, add add the
% [preprint] option:
%     \usepackage[preprint]{neurips_2018}

% to compile a camera-ready version, add the [final] option, e.g.:
     \usepackage[final]{nips_2018}

% to avoid loading the natbib package, add option nonatbib:
%     \usepackage[nonatbib]{neurips_2018}

\usepackage[utf8]{inputenc} % allow utf-8 input
\usepackage[T1]{fontenc}    % use 8-bit T1 fonts
\usepackage{hyperref}       % hyperlinks
\usepackage{url}            % simple URL typesetting
\usepackage{booktabs}       % professional-quality tables
\usepackage{amsfonts}       % blackboard math symbols
\usepackage{nicefrac}       % compact symbols for 1/2, etc.
\usepackage{microtype}      % microtypography

\title{Document de travail QALM}
\author{%
  David S.~Hippocampus \\
  Department of Computer Science\\
  Cranberry-Lemon University\\
  Pittsburgh, PA 15213 \\
  \texttt{hippo@cs.cranberry-lemon.edu} \\
  % examples of more authors
  % \And
  % Coauthor \\
  % Affiliation \\
  % Address \\
  % \texttt{email} \\
  % \AND
  % Coauthor \\
  % Affiliation \\
  % Address \\
  % \texttt{email} \\
  % \And
  % Coauthor \\
  % Affiliation \\
  % Address \\
  % \texttt{email} \\
  % \And
  % Coauthor \\
  % Affiliation \\
  % Address \\
  % \texttt{email} \\
}

\usepackage{natbib}
\usepackage{algorithm}
\usepackage{algorithmic}
\usepackage{graphicx}

\def\rmGamma{{\mathbf{\Gamma}}}

%%%%% NEW MATH DEFINITIONS %%%%%

\usepackage{amsmath,amsfonts,bm}

% Mark sections of captions for referring to divisions of figures
\newcommand{\figleft}{{\em (Left)}}
\newcommand{\figcenter}{{\em (Center)}}
\newcommand{\figright}{{\em (Right)}}
\newcommand{\figtop}{{\em (Top)}}
\newcommand{\figbottom}{{\em (Bottom)}}
\newcommand{\captiona}{{\em (a)}}
\newcommand{\captionb}{{\em (b)}}
\newcommand{\captionc}{{\em (c)}}
\newcommand{\captiond}{{\em (d)}}

% Highlight a newly defined term
\newcommand{\newterm}[1]{{\bf #1}}


% Figure reference, lower-case.
\def\figref#1{figure~\ref{#1}}
% Figure reference, capital. For start of sentence
\def\Figref#1{Figure~\ref{#1}}
\def\twofigref#1#2{figures \ref{#1} and \ref{#2}}
\def\quadfigref#1#2#3#4{figures \ref{#1}, \ref{#2}, \ref{#3} and \ref{#4}}
% Section reference, lower-case.
\def\secref#1{section~\ref{#1}}
% Section reference, capital.
\def\Secref#1{Section~\ref{#1}}
% Reference to two sections.
\def\twosecrefs#1#2{sections \ref{#1} and \ref{#2}}
% Reference to three sections.
\def\secrefs#1#2#3{sections \ref{#1}, \ref{#2} and \ref{#3}}
% Reference to an equation, lower-case.
\def\eqref#1{equation~\ref{#1}}
% Reference to an equation, upper case
\def\Eqref#1{Equation~\ref{#1}}
% A raw reference to an equation---avoid using if possible
\def\plaineqref#1{\ref{#1}}
% Reference to a chapter, lower-case.
\def\chapref#1{chapter~\ref{#1}}
% Reference to an equation, upper case.
\def\Chapref#1{Chapter~\ref{#1}}
% Reference to a range of chapters
\def\rangechapref#1#2{chapters\ref{#1}--\ref{#2}}
% Reference to an algorithm, lower-case.
\def\algref#1{algorithm~\ref{#1}}
% Reference to an algorithm, upper case.
\def\Algref#1{Algorithm~\ref{#1}}
\def\twoalgref#1#2{algorithms \ref{#1} and \ref{#2}}
\def\Twoalgref#1#2{Algorithms \ref{#1} and \ref{#2}}
% Reference to a part, lower case
\def\partref#1{part~\ref{#1}}
% Reference to a part, upper case
\def\Partref#1{Part~\ref{#1}}
\def\twopartref#1#2{parts \ref{#1} and \ref{#2}}

\def\ceil#1{\lceil #1 \rceil}
\def\floor#1{\lfloor #1 \rfloor}
\def\1{\bm{1}}
\newcommand{\train}{\mathcal{D}}
\newcommand{\valid}{\mathcal{D_{\mathrm{valid}}}}
\newcommand{\test}{\mathcal{D_{\mathrm{test}}}}

\def\eps{{\epsilon}}


% Random variables
\def\reta{{\textnormal{$\eta$}}}
\def\ra{{\textnormal{a}}}
\def\rb{{\textnormal{b}}}
\def\rc{{\textnormal{c}}}
\def\rd{{\textnormal{d}}}
\def\re{{\textnormal{e}}}
\def\rf{{\textnormal{f}}}
\def\rg{{\textnormal{g}}}
\def\rh{{\textnormal{h}}}
\def\ri{{\textnormal{i}}}
\def\rj{{\textnormal{j}}}
\def\rk{{\textnormal{k}}}
\def\rl{{\textnormal{l}}}
% rm is already a command, just don't name any random variables m
\def\rn{{\textnormal{n}}}
\def\ro{{\textnormal{o}}}
\def\rp{{\textnormal{p}}}
\def\rq{{\textnormal{q}}}
\def\rr{{\textnormal{r}}}
\def\rs{{\textnormal{s}}}
\def\rt{{\textnormal{t}}}
\def\ru{{\textnormal{u}}}
\def\rv{{\textnormal{v}}}
\def\rw{{\textnormal{w}}}
\def\rx{{\textnormal{x}}}
\def\ry{{\textnormal{y}}}
\def\rz{{\textnormal{z}}}

% Random vectors
\def\rvepsilon{{\mathbf{\epsilon}}}
\def\rvtheta{{\mathbf{\theta}}}
\def\rva{{\mathbf{a}}}
\def\rvb{{\mathbf{b}}}
\def\rvc{{\mathbf{c}}}
\def\rvd{{\mathbf{d}}}
\def\rve{{\mathbf{e}}}
\def\rvf{{\mathbf{f}}}
\def\rvg{{\mathbf{g}}}
\def\rvh{{\mathbf{h}}}
\def\rvu{{\mathbf{i}}}
\def\rvj{{\mathbf{j}}}
\def\rvk{{\mathbf{k}}}
\def\rvl{{\mathbf{l}}}
\def\rvm{{\mathbf{m}}}
\def\rvn{{\mathbf{n}}}
\def\rvo{{\mathbf{o}}}
\def\rvp{{\mathbf{p}}}
\def\rvq{{\mathbf{q}}}
\def\rvr{{\mathbf{r}}}
\def\rvs{{\mathbf{s}}}
\def\rvt{{\mathbf{t}}}
\def\rvu{{\mathbf{u}}}
\def\rvv{{\mathbf{v}}}
\def\rvw{{\mathbf{w}}}
\def\rvx{{\mathbf{x}}}
\def\rvy{{\mathbf{y}}}
\def\rvz{{\mathbf{z}}}

% Elements of random vectors
\def\erva{{\textnormal{a}}}
\def\ervb{{\textnormal{b}}}
\def\ervc{{\textnormal{c}}}
\def\ervd{{\textnormal{d}}}
\def\erve{{\textnormal{e}}}
\def\ervf{{\textnormal{f}}}
\def\ervg{{\textnormal{g}}}
\def\ervh{{\textnormal{h}}}
\def\ervi{{\textnormal{i}}}
\def\ervj{{\textnormal{j}}}
\def\ervk{{\textnormal{k}}}
\def\ervl{{\textnormal{l}}}
\def\ervm{{\textnormal{m}}}
\def\ervn{{\textnormal{n}}}
\def\ervo{{\textnormal{o}}}
\def\ervp{{\textnormal{p}}}
\def\ervq{{\textnormal{q}}}
\def\ervr{{\textnormal{r}}}
\def\ervs{{\textnormal{s}}}
\def\ervt{{\textnormal{t}}}
\def\ervu{{\textnormal{u}}}
\def\ervv{{\textnormal{v}}}
\def\ervw{{\textnormal{w}}}
\def\ervx{{\textnormal{x}}}
\def\ervy{{\textnormal{y}}}
\def\ervz{{\textnormal{z}}}

% Random matrices
\def\rmA{{\mathbf{A}}}
\def\rmB{{\mathbf{B}}}
\def\rmC{{\mathbf{C}}}
\def\rmD{{\mathbf{D}}}
\def\rmE{{\mathbf{E}}}
\def\rmF{{\mathbf{F}}}
\def\rmG{{\mathbf{G}}}
\def\rmH{{\mathbf{H}}}
\def\rmI{{\mathbf{I}}}
\def\rmJ{{\mathbf{J}}}
\def\rmK{{\mathbf{K}}}
\def\rmL{{\mathbf{L}}}
\def\rmM{{\mathbf{M}}}
\def\rmN{{\mathbf{N}}}
\def\rmO{{\mathbf{O}}}
\def\rmP{{\mathbf{P}}}
\def\rmQ{{\mathbf{Q}}}
\def\rmR{{\mathbf{R}}}
\def\rmS{{\mathbf{S}}}
\def\rmT{{\mathbf{T}}}
\def\rmU{{\mathbf{U}}}
\def\rmV{{\mathbf{V}}}
\def\rmW{{\mathbf{W}}}
\def\rmX{{\mathbf{X}}}
\def\rmY{{\mathbf{Y}}}
\def\rmZ{{\mathbf{Z}}}

% Elements of random matrices
\def\ermA{{\textnormal{A}}}
\def\ermB{{\textnormal{B}}}
\def\ermC{{\textnormal{C}}}
\def\ermD{{\textnormal{D}}}
\def\ermE{{\textnormal{E}}}
\def\ermF{{\textnormal{F}}}
\def\ermG{{\textnormal{G}}}
\def\ermH{{\textnormal{H}}}
\def\ermI{{\textnormal{I}}}
\def\ermJ{{\textnormal{J}}}
\def\ermK{{\textnormal{K}}}
\def\ermL{{\textnormal{L}}}
\def\ermM{{\textnormal{M}}}
\def\ermN{{\textnormal{N}}}
\def\ermO{{\textnormal{O}}}
\def\ermP{{\textnormal{P}}}
\def\ermQ{{\textnormal{Q}}}
\def\ermR{{\textnormal{R}}}
\def\ermS{{\textnormal{S}}}
\def\ermT{{\textnormal{T}}}
\def\ermU{{\textnormal{U}}}
\def\ermV{{\textnormal{V}}}
\def\ermW{{\textnormal{W}}}
\def\ermX{{\textnormal{X}}}
\def\ermY{{\textnormal{Y}}}
\def\ermZ{{\textnormal{Z}}}

% Vectors
\def\vzero{{\bm{0}}}
\def\vone{{\bm{1}}}
\def\vmu{{\bm{\mu}}}
\def\vtheta{{\bm{\theta}}}
\def\va{{\bm{a}}}
\def\vb{{\bm{b}}}
\def\vc{{\bm{c}}}
\def\vd{{\bm{d}}}
\def\ve{{\bm{e}}}
\def\vf{{\bm{f}}}
\def\vg{{\bm{g}}}
\def\vh{{\bm{h}}}
\def\vi{{\bm{i}}}
\def\vj{{\bm{j}}}
\def\vk{{\bm{k}}}
\def\vl{{\bm{l}}}
\def\vm{{\bm{m}}}
\def\vn{{\bm{n}}}
\def\vo{{\bm{o}}}
\def\vp{{\bm{p}}}
\def\vq{{\bm{q}}}
\def\vr{{\bm{r}}}
\def\vs{{\bm{s}}}
\def\vt{{\bm{t}}}
\def\vu{{\bm{u}}}
\def\vv{{\bm{v}}}
\def\vw{{\bm{w}}}
\def\vx{{\bm{x}}}
\def\vy{{\bm{y}}}
\def\vz{{\bm{z}}}

% Elements of vectors
\def\evalpha{{\alpha}}
\def\evbeta{{\beta}}
\def\evepsilon{{\epsilon}}
\def\evlambda{{\lambda}}
\def\evomega{{\omega}}
\def\evmu{{\mu}}
\def\evpsi{{\psi}}
\def\evsigma{{\sigma}}
\def\evtheta{{\theta}}
\def\eva{{a}}
\def\evb{{b}}
\def\evc{{c}}
\def\evd{{d}}
\def\eve{{e}}
\def\evf{{f}}
\def\evg{{g}}
\def\evh{{h}}
\def\evi{{i}}
\def\evj{{j}}
\def\evk{{k}}
\def\evl{{l}}
\def\evm{{m}}
\def\evn{{n}}
\def\evo{{o}}
\def\evp{{p}}
\def\evq{{q}}
\def\evr{{r}}
\def\evs{{s}}
\def\evt{{t}}
\def\evu{{u}}
\def\evv{{v}}
\def\evw{{w}}
\def\evx{{x}}
\def\evy{{y}}
\def\evz{{z}}

% Matrix
\def\mA{{\bm{A}}}
\def\mB{{\bm{B}}}
\def\mC{{\bm{C}}}
\def\mD{{\bm{D}}}
\def\mE{{\bm{E}}}
\def\mF{{\bm{F}}}
\def\mG{{\bm{G}}}
\def\mH{{\bm{H}}}
\def\mI{{\bm{I}}}
\def\mJ{{\bm{J}}}
\def\mK{{\bm{K}}}
\def\mL{{\bm{L}}}
\def\mM{{\bm{M}}}
\def\mN{{\bm{N}}}
\def\mO{{\bm{O}}}
\def\mP{{\bm{P}}}
\def\mQ{{\bm{Q}}}
\def\mR{{\bm{R}}}
\def\mS{{\bm{S}}}
\def\mT{{\bm{T}}}
\def\mU{{\bm{U}}}
\def\mV{{\bm{V}}}
\def\mW{{\bm{W}}}
\def\mX{{\bm{X}}}
\def\mY{{\bm{Y}}}
\def\mZ{{\bm{Z}}}
\def\mBeta{{\bm{\beta}}}
\def\mPhi{{\bm{\Phi}}}
\def\mLambda{{\bm{\Lambda}}}
\def\mSigma{{\bm{\Sigma}}}

% Tensor
\DeclareMathAlphabet{\mathsfit}{\encodingdefault}{\sfdefault}{m}{sl}
\SetMathAlphabet{\mathsfit}{bold}{\encodingdefault}{\sfdefault}{bx}{n}
\newcommand{\tens}[1]{\bm{\mathsfit{#1}}}
\def\tA{{\tens{A}}}
\def\tB{{\tens{B}}}
\def\tC{{\tens{C}}}
\def\tD{{\tens{D}}}
\def\tE{{\tens{E}}}
\def\tF{{\tens{F}}}
\def\tG{{\tens{G}}}
\def\tH{{\tens{H}}}
\def\tI{{\tens{I}}}
\def\tJ{{\tens{J}}}
\def\tK{{\tens{K}}}
\def\tL{{\tens{L}}}
\def\tM{{\tens{M}}}
\def\tN{{\tens{N}}}
\def\tO{{\tens{O}}}
\def\tP{{\tens{P}}}
\def\tQ{{\tens{Q}}}
\def\tR{{\tens{R}}}
\def\tS{{\tens{S}}}
\def\tT{{\tens{T}}}
\def\tU{{\tens{U}}}
\def\tV{{\tens{V}}}
\def\tW{{\tens{W}}}
\def\tX{{\tens{X}}}
\def\tY{{\tens{Y}}}
\def\tZ{{\tens{Z}}}


% Graph
\def\gA{{\mathcal{A}}}
\def\gB{{\mathcal{B}}}
\def\gC{{\mathcal{C}}}
\def\gD{{\mathcal{D}}}
\def\gE{{\mathcal{E}}}
\def\gF{{\mathcal{F}}}
\def\gG{{\mathcal{G}}}
\def\gH{{\mathcal{H}}}
\def\gI{{\mathcal{I}}}
\def\gJ{{\mathcal{J}}}
\def\gK{{\mathcal{K}}}
\def\gL{{\mathcal{L}}}
\def\gM{{\mathcal{M}}}
\def\gN{{\mathcal{N}}}
\def\gO{{\mathcal{O}}}
\def\gP{{\mathcal{P}}}
\def\gQ{{\mathcal{Q}}}
\def\gR{{\mathcal{R}}}
\def\gS{{\mathcal{S}}}
\def\gT{{\mathcal{T}}}
\def\gU{{\mathcal{U}}}
\def\gV{{\mathcal{V}}}
\def\gW{{\mathcal{W}}}
\def\gX{{\mathcal{X}}}
\def\gY{{\mathcal{Y}}}
\def\gZ{{\mathcal{Z}}}

% Sets
\def\sA{{\mathbb{A}}}
\def\sB{{\mathbb{B}}}
\def\sC{{\mathbb{C}}}
\def\sD{{\mathbb{D}}}
% Don't use a set called E, because this would be the same as our symbol
% for expectation.
\def\sF{{\mathbb{F}}}
\def\sG{{\mathbb{G}}}
\def\sH{{\mathbb{H}}}
\def\sI{{\mathbb{I}}}
\def\sJ{{\mathbb{J}}}
\def\sK{{\mathbb{K}}}
\def\sL{{\mathbb{L}}}
\def\sM{{\mathbb{M}}}
\def\sN{{\mathbb{N}}}
\def\sO{{\mathbb{O}}}
\def\sP{{\mathbb{P}}}
\def\sQ{{\mathbb{Q}}}
\def\sR{{\mathbb{R}}}
\def\sS{{\mathbb{S}}}
\def\sT{{\mathbb{T}}}
\def\sU{{\mathbb{U}}}
\def\sV{{\mathbb{V}}}
\def\sW{{\mathbb{W}}}
\def\sX{{\mathbb{X}}}
\def\sY{{\mathbb{Y}}}
\def\sZ{{\mathbb{Z}}}

% Entries of a matrix
\def\emLambda{{\Lambda}}
\def\emA{{A}}
\def\emB{{B}}
\def\emC{{C}}
\def\emD{{D}}
\def\emE{{E}}
\def\emF{{F}}
\def\emG{{G}}
\def\emH{{H}}
\def\emI{{I}}
\def\emJ{{J}}
\def\emK{{K}}
\def\emL{{L}}
\def\emM{{M}}
\def\emN{{N}}
\def\emO{{O}}
\def\emP{{P}}
\def\emQ{{Q}}
\def\emR{{R}}
\def\emS{{S}}
\def\emT{{T}}
\def\emU{{U}}
\def\emV{{V}}
\def\emW{{W}}
\def\emX{{X}}
\def\emY{{Y}}
\def\emZ{{Z}}
\def\emSigma{{\Sigma}}

% entries of a tensor
% Same font as tensor, without \bm wrapper
\newcommand{\etens}[1]{\mathsfit{#1}}
\def\etLambda{{\etens{\Lambda}}}
\def\etA{{\etens{A}}}
\def\etB{{\etens{B}}}
\def\etC{{\etens{C}}}
\def\etD{{\etens{D}}}
\def\etE{{\etens{E}}}
\def\etF{{\etens{F}}}
\def\etG{{\etens{G}}}
\def\etH{{\etens{H}}}
\def\etI{{\etens{I}}}
\def\etJ{{\etens{J}}}
\def\etK{{\etens{K}}}
\def\etL{{\etens{L}}}
\def\etM{{\etens{M}}}
\def\etN{{\etens{N}}}
\def\etO{{\etens{O}}}
\def\etP{{\etens{P}}}
\def\etQ{{\etens{Q}}}
\def\etR{{\etens{R}}}
\def\etS{{\etens{S}}}
\def\etT{{\etens{T}}}
\def\etU{{\etens{U}}}
\def\etV{{\etens{V}}}
\def\etW{{\etens{W}}}
\def\etX{{\etens{X}}}
\def\etY{{\etens{Y}}}
\def\etZ{{\etens{Z}}}

% The true underlying data generating distribution
\newcommand{\pdata}{p_{\rm{data}}}
% The empirical distribution defined by the training set
\newcommand{\ptrain}{\hat{p}_{\rm{data}}}
\newcommand{\Ptrain}{\hat{P}_{\rm{data}}}
% The model distribution
\newcommand{\pmodel}{p_{\rm{model}}}
\newcommand{\Pmodel}{P_{\rm{model}}}
\newcommand{\ptildemodel}{\tilde{p}_{\rm{model}}}
% Stochastic autoencoder distributions
\newcommand{\pencode}{p_{\rm{encoder}}}
\newcommand{\pdecode}{p_{\rm{decoder}}}
\newcommand{\precons}{p_{\rm{reconstruct}}}

\newcommand{\laplace}{\mathrm{Laplace}} % Laplace distribution

\newcommand{\E}{\mathbb{E}}
\newcommand{\Ls}{\mathcal{L}}
\newcommand{\R}{\mathbb{R}}
\newcommand{\emp}{\tilde{p}}
\newcommand{\lr}{\alpha}
\newcommand{\reg}{\lambda}
\newcommand{\rect}{\mathrm{rectifier}}
\newcommand{\softmax}{\mathrm{softmax}}
\newcommand{\sigmoid}{\sigma}
\newcommand{\softplus}{\zeta}
\newcommand{\KL}{D_{\mathrm{KL}}}
\newcommand{\Var}{\mathrm{Var}}
\newcommand{\standarderror}{\mathrm{SE}}
\newcommand{\Cov}{\mathrm{Cov}}
% Wolfram Mathworld says $L^2$ is for function spaces and $\ell^2$ is for vectors
% But then they seem to use $L^2$ for vectors throughout the site, and so does
% wikipedia.
\newcommand{\normlzero}{L^0}
\newcommand{\normlone}{L^1}
\newcommand{\normltwo}{L^2}
\newcommand{\normlp}{L^p}
\newcommand{\normmax}{L^\infty}

\newcommand{\parents}{Pa} % See usage in notation.tex. Chosen to match Daphne's book.

\DeclareMathOperator*{\argmax}{arg\,max}
\DeclareMathOperator*{\argmin}{arg\,min}

\DeclareMathOperator{\sign}{sign}
\DeclareMathOperator{\Tr}{Tr}
\let\ab\allowbreak


\begin{document}

\maketitle

\section{Notations}

$[\![K]\!]$ with $K \in \sN$ is the set of all i such as $i \in \sN$ and $i < K$ . $\rmX \in \R^{n \times d}$ is the considered dataset. For a given matrix $\rmA$, the notation $\rmA_i$ refers to the $i^{th}$ row of the matrix $\rmA$.

\section{Background}

\subsection{K-means}

The K-means algorithm is used to partition a given set of observations $\rmX$ into a predefined amount of $K$ clusters while minimizing the distance between the observation in each partition:

\begin{equation}
\label{eq:kmean_problem}
    \argmin_{\rmU, \rvt} \sum_{k=1}^{K} \sum_{j: \rvt_j = k} ||\rmX_j -\rmU_k||^2 \\
\end{equation}
% autre écriture de l'objectif de k-means
% = \argmin_{\rmU, \rvt} \sum_{k=1}^{K} c_k + \sum_{k=1}^{K} n_k||\hat{\rmU}_k - \rmU_k||^2
where $\rmU \in \R^{K \times d}$ is the matrix of the clusters center-points and $\rvt \in  [\![K]\!]^n$ is the indicator vector.

The algorithm (Algorithm \ref{algo:kmeans}) starts with an initialized set of $K$ center-points ($\{\rmU_i \in \R^d\}_{i=1}^{K}$). Each update step $\tau$ is divided into two parts: (i) all observations $\rmX_i$ are assigned to their nearest cluster based on the center-points $\rmU_i^{\tau-1}$s at this step (Line \ref{line:assignation_kmeans}) in $O(ndK)$ operations. (ii) the new center-points $\rmU_i^{\tau}$s are computed as the means of the assignated $\rmX_i$ (Line \ref{line:compute_means}) for a total of $O(nd)$ operations.

\begin{algorithm}
\caption{K-means algorithm}
\label{algo:kmeans}
\begin{algorithmic}[1]


\REQUIRE $\rmX \in \R^{n \times d}$, $K$, $\{\rmU_i \in \R^d\}_{i=1}^{K}$
\ENSURE $\{\rmU_i\}_{i=1}^{K}$ the K means of $n$ $d$-dimensional samples
\STATE $\tau \leftarrow 0$
\REPEAT
\STATE $\tau \leftarrow \tau + 1$
\STATE $\rvt^\tau \leftarrow \argmin_{\rvt \in [\![K]\!]^n} \sum_{i=1}^{n} {||\rvx_i - \rmU^{\tau-1}_{\rvt(i)}||_2^2}$
\label{line:assignation_kmeans}
\FORALL {$k \in [\![K]\!]^n$}
\STATE $n_k^\tau \leftarrow |\{i: \rvt^\tau_i=k\}|$
\STATE $\rmU^\tau_k \leftarrow \frac{1}{n_k^\tau} \sum_{i: \rvt^\tau_i = k} {\rvx_i}$
\label{line:compute_means}
\ENDFOR
\UNTIL{stop criterion}
\RETURN $\rmU^\tau$
\end{algorithmic}
\end{algorithm}


Once the cluster have been defined, for any $\rvx \in \R^d$ the cluster associated with this $\rvx$ is:

\begin{equation}
\label{eq:assignation_problem_kmeans}
\argmin_{k \in [\![K]\!]} ||\rvx - \rmU_{k}||_2^2 = \argmin_{k \in [\![K]\!]} ||U_k||_2^2 - 2 \rmU_{k}^T\rvx
\end{equation}.


We remark here that the computational bottleneck of this assignation lies in the computation of $\rmU_k^T\rvx$ for all $k$. This computation is also encountered in the assignation step Line \ref{line:assignation_kmeans} of the Algorithm \ref{algo:kmeans}.


\subsection{Learning Fast transforms as the product of sparse matrices}

A popular way for providing concise description of high-dimensional vectors $\rmU \in \R^{K \times d}$ is to compute a sparse representation using a dictionnary:

\begin{equation}
\rmU^T \approx \rmD\rmGamma
\end{equation}

where $\rmD \in \R^{d \times d}$ is a dictionnary and $\rmGamma \in \R^{d \times K}$ has sparse columns. Historically, the dictionnary is either (i) analytic: $\rmD$ is chosen to give a fast reconstruction of the initial matrix by taking advantage of some fast-transform algorithm (the \textit{Fast Hadamard Transform} for instance) or (ii) learned: $\rmD$ is learned from the data itself to give a good reconstruction of the initial matrix.

Building on the observation that the fast-transform associated with an analytic dictionnary can be expressed as the product of sparse matrices, \cite{magoarou2014learning} proposes an algorithm to learn a dictionnary from the data with sparsity constraints such that this dictionnary would be both well-suited with the data and fast to use:

\begin{equation}
\rmD = \lambda \prod_{j=1}^{M}\rmS_j
\end{equation}

with $\rmS_j \in \mathcal{E}_j$, $\mathcal{E}_j = \lbrace \rmA \in \R^{a \times a+1}~\text{s.t.}~||\rmA||_0^0 \leq p_j, ||\rmA||_1 = 1 \rbrace$ and $p_j$ being chosen suitably. The $\lambda$ parameter has been added along with the normalization constraint in the $\mathcal{E}_j$ in order to remove scaling ambiguity in the learned $\rmS_j$.

Considering $\rmGamma$ being a sparse matrice too, it can be renamed as $\rmGamma = \rmS_{M+1}$. We set $Q = M+1$ and the overall dictionnary learning problem can be expressed as the following optimisation problem:

\begin{equation}
\label{eq:problem_gribon}
\min_{\{\rmS_1 \dots \rmS_P, \lambda\}} ||\rmU - \lambda \prod_{j=1}^{Q}{\rmS_j}||_2^2 + \sum_{j=1}^{Q} \delta_j(\rmS_j)
\end{equation}

with the $\delta_j(\rmS_j) = 0$ if $\rmS_j \in \mathcal{E}_j$ being the sparsity constraints to satisfy on the associated $\rmS_j$.

Although this probleme being highly non-convex, the authors derive an algorithm from the PALM algorithm \cite{bolte2014proximal}, which they call \textit{Hierarchical PALM4LED} to find a good local minima and give convergence guarantees to learn efficient dictionaries.


\section{Contribution}

\subsection{Q-means}

We propose an extension of the K-means algorithm in which the output matrix of center-points $\rmU$ is approximated by a product of sparse matrices $\rmS_j: j = 1 \ldots Q$. From Equation \ref{eq:kmean_problem} and Equation \ref{eq:problem_gribon} we can write a new K-means optimisation problem with sparse factorization constraint which we call \textit{Q-means}:

\begin{equation}
\begin{split}
\label{eq:qmean_problem}
    \argmin_{\{\rmS_1 \dots \rmS_P, \lambda\}, \rvt} & \sum_{k=1}^{K} \sum_{j: \rvt_j = k} ||\rmX_j -\rmU_k||^2 + \sum_{j=1}^{Q} \delta_j(\rmS_j) \\
    & s.t. ~ \rmU = \lambda \prod_{j=1}^{Q}{\rmS_j}
\end{split}
\end{equation}.

This problem can be solved using Algorithm \ref{algo:qmeans} which is a simple extension of the K-means algorithm (Algorithm \ref{algo:kmeans}) and is guaranteed to converge. Indeed, the K-means problem at a given time-step $\tau$ with fixed $\rvt$ can be re-written:

\begin{equation}
\label{eq:kmeans_problem_t_fixed}
\begin{split}
 \argmin_{\rmU} & \sum_{k=1}^{K} c_k^\tau + \sum_{k=1}^{K} n_k^\tau||\hat{\rmU}^\tau_k - \rmU_k||^2 \\
 % & = \argmin_{\rmU} \sum_{k=1}^{K} n_k^\tau||\hat{\rmU}^\tau_k - \rmU_k||^2 \\
 % s.t:~& c^\tau_k = \sum_{j: \rvt^\tau_j = k} {||\rmX_j - \hat{\rmU}^\tau_k||^2} \\
 % & \hat{\rmU}^\tau_k = \frac{1}{n^\tau_k} \sum_{j: \rvt^\tau_j = k} {\rmX_j}
\end{split}
\end{equation} 

where $c^\tau_k = \sum_{j: \rvt^\tau_j = k} {||\rmX_j - \hat{\rmU}^\tau_k||^2}$ is constant \textit{w.r.t.} $\rmU$ and $\hat{\rmU}^\tau_k = \frac{1}{n^\tau_k} \sum_{j: \rvt^\tau_j = k} {\rmX_j}$ is the actual K-means solution for $\rmU$ at this time step (Algorithm \ref{algo:kmeans}, Line \ref{line:compute_means}). Roughly speaking, this formulation summarizes to finding the $\rmU$ that minimizes its column-wise distance to the oracle $\rmU^\tau$ at this time-step.

Pluging back this formulation (Equation \ref{eq:kmeans_problem_t_fixed}) into Equation \ref{eq:qmean_problem}, we have at a given time-step $\tau$ of Algorithm \ref{algo:qmeans} and for a fixed $\rvt$:

\begin{equation}
\label{eq:qmeans_problem_t_fixed}
\begin{split}
 \argmin_{\{\rmS_1 \dots \rmS_P, \lambda\}} & \sum_{k=1}^{K} ||\sqrt{n_k^\tau}\hat{\rmU}^{\tau}_k - \sqrt{n_k^\tau}\rmU_k||^2 + \sum_{j=1}^{Q} \delta_j(\rmS_j)\\
 =\argmin_{\{\rmS_1 \dots \rmS_P, \lambda\}} & ||\sqrt{\rvn^\tau} \odot \hat{\rmU}^{\tau} - \sqrt{\rvn^\tau} \odot \rmU||^2 + \sum_{j=1}^{Q} \delta_j(\rmS_j)\\
 %\argmin_{\{\rmS_1 \dots \rmS_P, \lambda\}} & \sum_{k=1}^{K} ||\sqrt{n_k^\tau}\hat{\rmU}^{\surd\tau}_k - \sqrt{n_k^\tau}\rmU^\surd_k||^2 + \sum_{j=1}^{Q} \delta_j(\rmS_j)\\
 s.t. ~& \rmU = \lambda \prod_{j=1}^{Q}{\rmS_j} \\
 %& \hat{\rmU}^{\surd\tau}_k = \sqrt{n_k^\tau}\hat{\rmU}^{\tau}_k \\
 %& \rmU^{\surd}_k = \sqrt{n_k^\tau}\rmU_k \\
\end{split}
\end{equation}

where $\rvn \in \R^K$ is the vector containing the number of elements of $\rmX$ in each cluster and $\rvv \odot \rmA$ refers to the broadcasted multiplication of the elements of $\rvv$ with the rows of $\rmA$. The analogy with Equation \ref{eq:problem_gribon} is now obvious: we can use the \textit{Hierarchical PALM4LED} algorithm \cite{magoarou2014learning} to solve the problem of Equation \ref{eq:qmeans_problem_t_fixed} where $\rvn^\tau$ and $\hat{\rmU}^{\tau}$ are the real K-means computed in Lines \ref{line:startkmeans} to \ref{line:endkmeans} (Algorithm \ref{algo:qmeans}).

\begin{algorithm}
\caption{Q-means algorithm}
\label{algo:qmeans}
\begin{algorithmic}[1]


\REQUIRE $\rmX \in \R^{n \times d}$, $K$, $\{\rmS_1 \dots \rmS_{Q}\}|\rmS_j \in \mathcal{E}_j$
\ENSURE $\{\rmS_1 \dots \rmS_{Q}\}|\rmS_j \in \mathcal{E}_j$ such that $\prod_{j=1}^{Q}\rmS_j \approx \rmU$ the K means of $n$ $d$-dimensional samples
\STATE $\tau \leftarrow 0$
\REPEAT
\STATE $\tau \leftarrow \tau + 1$
\STATE $t^\tau \leftarrow \argmin_{t \in [\![K]\!]^n} \sum_{i=1}^{n} {||\rvx_i - \lambda \left[\prod_{j=1}^{Q}{\rmS_j^{\tau -1}}\right]_{t(i)}||_2^2}$
\FORALL {$k \in [\![K]\!]$}
\label{line:startkmeans}
\STATE $n_k^\tau \leftarrow |\{i: t^\tau_i=k\}|$
\STATE $\hat{\rmU}^\tau_k \leftarrow \frac{1}{n^\tau_k} \sum_{i: t^\tau_i = k} {\rvx_i}$
\ENDFOR
\label{line:endkmeans}
\STATE $\{\rmS^\tau_1 \dots \rmS^\tau_{Q}\} \leftarrow \argmin_{\{\rmS_1 \dots \rmS_P, \lambda\}} ||\sqrt{\rvn^\tau} \odot \hat{\rmU}^\tau - \sqrt{\rvn^\tau} \odot \lambda \prod_{j=1}^{Q}{\rmS_j}||_2^2 + \sum_{j=1}^{Q} \delta_j(\rmS_j)$
\UNTIL{$i=0$}
\end{algorithmic}
\end{algorithm}

The factorization of $\rmU$ could then be used in ulterior algorithm that involve a matrix-vector multiplication with $\rmU$: typically any algorithm involving the assignation of some data points to one of the cluster (Equation \ref{eq:assignation_problem_kmeans}). Such applications of our proposed algorithm are discussed in Section \ref{sec:uses}.

\section{Applications}
\label{sec:uses}

\subsection{Nyström approximation}

Standard kernel machines are often prohibited in large scale applications because of their associated Gram matrix which has $O(n^2)$ storage and $O(n^2d)$ computational complexity. A well-known strategy to overcome this problem is to use the Nyström method which compute a low-rank approximation of the Gram matrix on the basis of some pre-selected landmark points. 

\subsubsection{Efficient Nyström approximation}

A substantial amount of research has been conducted toward landmark point selection methods for improved approximation accuracy \cite{kumar2012sampling} \cite{musco2017recursive} but much less has been done for improved computation speed. In \cite{si2016computationally}, they propose an algorithm to learn the matrix of landmark points with some structure constraint so that its utilisation is fast, taking advantage of fast-transforms. This results in an efficient Nyström approximation that is faster to use both in the training and testing phase of some ulterior machine learning application.

Remarking that the main computation cost of the Nyström approximation comes from the computation of the kernel function between the train/test samples and the landmark points, they aim at accelerating this step. In particular, they focus on a family of kernel functions that have the following form:

\begin{equation}
 K(\rvx_i, \rvx_j) = f(\rvx_i) f(\rvx_j) g(\rvx_i^T\rvx_j)
\end{equation}

where $f: \R^d \rightarrow \R$ and $g: \R \rightarrow \R$. They show that this family of functions contains some widely used kernel such as the Gaussian kernel or the Polynomial one. Given a set of $m$ landmark points $\rmU \in \R^{m \times d}$ and a sample $\rvx$, the computational time for computing the kernel between $\rvx$ and each row of $\rmU$ (necessary for the Nyström approximation) is bottlenecked by the computation of the product $\rmU\rvx$. They hence propose to write the $\rmU$ matrix as the concatenation of structured $s = m / d$ product of matrices:

\begin{equation}
 \rmU = \left[ \rmV_1 \rmH^T, \cdots, \rmV_s\rmH^T  \right]^T
\end{equation}

where the $\rmH$ is a $d \times d$ matrix associated with a fast transform such as the \textit{Haar} or \textit{Hadamard} matrix and the $\rmV_i$s are some $d \times d$ diagonal matrices to be either chosen with a standard landmark selection method or learned using an algorithm they provide.

Depending on the $\rmH$ matrix they chose, they can improve the time complexity for the computation of $\rmU\rvx$ from $O(md)$ to $O(m \log{d})$ (\textit{Fast Hadamard transform}) or $O(m)$ (\textit{Fast Haar Transform}).

\subsubsection{QALM in Nyström}

We propose to use our QALM algorithm in order to learn directly the $\rmU$ matrix in the Nyström approximation so that the matrix-vector multiplaction $\rmU \rvx$ is cheap to compute.

Construction remplacement d'une matrice de hadamard par une transformée apprise

\subsection{RBF networks}

Besoin d'éclaircir les liens avec RBF networks

\subsection{nearest-neighbours}

Besoin d'éclaircir les liens avec nearest neighbours

%\bibliographystyle{plain}
%\bibliography{references}
\bibliographystyle{plain}
\bibliography{biblio}
\end{document}
