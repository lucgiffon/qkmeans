\documentclass{article}
\usepackage[utf8]{inputenc}

\title{Document de travail QALM}
\author{QARMA}
\date{March 2019}

\usepackage{natbib}
\usepackage{algorithm}
\usepackage{algorithmic}
\usepackage{graphicx}

\def\rmGamma{{\mathbf{\Gamma}}}

\input{math_commands.tex}

\begin{document}

\maketitle

\section{Notations}

$[\![K]\!]$ with $K \in \sN$ is the set of all i such as $i \in \sN$ and $i < K$ . $\rmX = \{\rvx_i \in \R^d\}_{i=1}^n$ is the considered dataset.

\section{Background}

\subsection{K-means}

The K-means algorithm is used to partition a given set of observations $\rmX$ into a predefined amount of $K$ clusters. The algorithm starts with an initialized set of $K$ center-points ($\{\rmU_i \in \R\}_{i=1}^{K}$). Each update step is divided in two parts: (i) all observations $\rvx_i$ are assigned to their nearest cluster based on the center-points $\rmU_i$s (Equation \ref{eq:kmeans_assign_step}) in $O(ndK)$ operations. (ii) the new center-points are computed as the means of the assigned $\rvx_i$ (Equation \ref{eq:kmeans_compute_step}) for a total of $O(nd)$ operations.

\begin{algorithm}
\label{algo:kmeans}
\caption{K-means algorithm}
\begin{algorithmic}


\REQUIRE $\rmX \in \R^{n \times d}$, $K$, $\{\rmU_i \in \R^d\}_{i=1}^{K}$
\ENSURE $\{\rmU_i\}_{i=1}^{K}$ the K means of $n$ $d$-dimensional samples
\REPEAT
\STATE \begin{equation}
\label{eq:kmeans_assign_step}
t \leftarrow \argmin_{t \in [\![K]\!]^n} \sum_{i=1}^{n} {||\rvx_i - \rmU_{t(i)}||_2^2}
\end{equation}
\FORALL {$k \in [\![K]\!]^n$}
\STATE $n_k \leftarrow |\{i: t_i=k\}|$
\STATE \begin{equation}
\label{eq:kmeans_compute_step}
\rmU_k \leftarrow \frac{1}{n_k} \sum_{i: t_i = k} {\rvx_i}
\end{equation}
\ENDFOR
\UNTIL{$i=0$}
\end{algorithmic}
\end{algorithm}


Once the cluster have been defined, for any $\rvx \in \R^d$ the cluster associated with this $\rvx$ is:

\begin{equation}
\label{eq:assignation_problem_kmeans}
\argmin_{k \in [\![K]\!]^n} ||\rvx - \rmU_{k}||_2^2 = \argmin_{k \in [\![K]\!]^n} ||U_k||_2^2 - 2 \rmU_{k}^T\rvx
\end{equation}.

We remark here that the computational bottleneck of this assignation lies in the computation of $\rmU_k^T\rvx$ for all $k$. This bottleneck is also encountered at each step in the Equation \ref{eq:kmeans_assign_step} of the Algorithm \ref{algo:kmeans}.


\subsection{Learning Fast transforms as the product of sparse matrices}

A popular way for providing concise description of high-dimensional vectors $\rmU \in \R^{K \times d}$ is to compute a sparse representation using a dictionnary:

\begin{equation}
\rmU^T \approx \rmD\rmGamma
\end{equation}

where $\rmD \in \R^{d \times d}$ is a dictionnary and $\rmGamma \in \R^{d \times K}$ has sparse columns. Historically, the dictionnary is either (i) analytic: $\rmD$ is chosen to give a fast reconstruction of the initial matrix by taking advantage of some fast-transform algorithm (the \textit{Fast Hadamard Transform} for instance) or (ii) learned: $\rmD$ is learned from the data itself to give a good reconstruction of the initial matrix.

Building on the observation that the fast-transform associated with an analytic dictionnary can be expressed as the product of sparse matrices, \cite{magoarou2014learning} proposes an algorithm to learn a dictionnary from the data with sparsity constraints such that this dictionnary would be both well-suited with the data and fast to use:

\begin{equation}
\rmD = \lambda \prod_{j=1}^{M}\rmS_j
\end{equation}

with $\rmS_j \in \mathcal{E}_j$, $\mathcal{E}_j = \lbrace \rmA \in \R^{a \times a+1}~\text{s.t.}~||\rmA||_0^0 \leq p_j, ||\rmA||_1 = 1 \rbrace$ and $p_j$ being chosen suitably. The $\lambda$ parameter has been added along with the normalization constraint in the $\mathcal{E}_j$ in order to remove scaling ambiguity in the learned $\rmS_j$.

Considering $\rmGamma$ being a sparse matrice too, it can be renamed as $\rmGamma = \rmS_{M+1}$ and the overall dictionnary learning problem can be expressed as the following optimisation problem:

\begin{equation}
\min_{\{\rmS_1 \dots \rmS_P, \lambda\}} ||\rmU - \lambda \prod_{j=1}^{M+1}{\rmS_j}||_2^2 + \sum_{j=1}^{M+1} \delta_j(\rmS_j)
\end{equation}

with the $\delta_j(\rmS_j) = 0$ if $\rmS_j \in \mathcal{E}_j$ being the sparsity constraints to satisfy on the associated $\rmS_j$.

Although this probleme being highly non-convex, the authors finaly derive an algorithm from the PALM algorithm \cite{bolte2014proximal} to find a good local minima and give convergence guarantees to learn efficient dictionaries.


\section{Contribution}

\subsection{Q-means}

We propose an extension of the K-means algorithm in which the output matrix of center-points $\rmU$ are approximated by a product of $M+1$ sparse matrices $\rmS_j$. 

\begin{algorithm}
\caption{Q-means algorithm}
\begin{algorithmic}


\REQUIRE $\rmX \in \R^{n \times d}$, $K$, $\{\rmU_i \in \R^d\}_{i=1}^{K}$, $\{\rmS_1 \dots \rmS_{M+1}\}|\rmS_j \in \mathcal{E}_j$
\ENSURE $\{\rmS_1 \dots \rmS_{M+1}\}|\rmS_j \in \mathcal{E}_j$ such that $\prod_{j=1}^{M+1}\rmS_j \approx \rmU$ the K means of $n$ $d$-dimensional samples
\REPEAT
\STATE \begin{equation}
t \leftarrow \argmin_{t \in [\![K]\!]^n} \sum_{i=1}^{n} {||\rvx_i - \rmU_{t(i)}||_2^2}
\end{equation}
\FORALL {$k \in [\![K]\!]^n$}
\STATE $n_k \leftarrow |\{i: t_i=k\}|$
\STATE \begin{equation}
\rmU_k \leftarrow \frac{1}{n_k} \sum_{i: t_i = k} {\rvx_i}
\end{equation}
\STATE $\{\rmS_1 \dots \rmS_{M+1}\} \leftarrow \argmin_{\{\rmS_1 \dots \rmS_P, \lambda\}} ||\rmU - \lambda \prod_{j=1}^{M+1}{\rmS_j}||_2^2 + \sum_{j=1}^{M+1} \delta_j(\rmS_j)$
\ENDFOR
\UNTIL{$i=0$}
\end{algorithmic}
\end{algorithm}

The factorization of $\rmU$ could then be used in ulterior algorithm that involve a matrix-vector multiplication with $\rmU$: typically any algorithm involving the assignation of some data points to one of the cluster (Equation \ref{eq:assignation_problem_kmeans}). Such applications of our proposed algorithm are discussed in Section \ref{sec:uses}.

\section{Applications}
\label{sec:uses}

\subsection{Nyström approximation}

Standard kernel machines are often prohibited in large scale applications because of their associated Gram matrix which has $O(n^2)$ storage and $O(n^2d)$ computational complexity. A well-known strategy to overcome this problem is to use the Nyström method which compute a low-rank approximation of the Gram matrix on the basis of some pre-selected landmark points. 

\subsubsection{Efficient Nyström approximation}

A substantial amount of research has been conducted toward landmark point selection methods for improved approximation accuracy \cite{kumar2012sampling} \cite{musco2017recursive} but much less has been done for improved computation speed. In \cite{si2016computationally}, they propose an algorithm to learn the matrix of landmark points with some structure constraint so that its utilisation is fast, taking advantage of fast-transforms. This results in an efficient Nyström approximation that is faster to use both in the training and testing phase of some ulterior machine learning application.

Remarking that the main computation cost of the Nyström approximation comes from the computation of the kernel function between the train/test samples and the landmark points, they aim at accelerating this step. In particular, they focus on a family of kernel functions that have the following form:

\begin{equation}
 K(\rvx_i, \rvx_j) = f(\rvx_i) f(\rvx_j) g(\rvx_i^T\rvx_j)
\end{equation}

where $f: \R^d \rightarrow \R$ and $g: \R \rightarrow \R$. They show that this family of functions contains some widely used kernel such as the Gaussian kernel or the Polynomial one. Given a set of $m$ landmark points $\rmU \in \R^{m \times d}$ and a sample $\rvx$, the computational time for computing the kernel between $\rvx$ and each row of $\rmU$ (necessary for the Nyström approximation) is bottlenecked by the computation of the product $\rmU\rvx$. They hence propose to write the $\rmU$ matrix as the concatenation of structured $s = m / d$ product of matrices:

\begin{equation}
 \rmU = \left[ \rmV_1 \rmH^T, \cdots, \rmV_s\rmH^T  \right]^T
\end{equation}

where the $\rmH$ is a $d \times d$ matrix associated with a fast transform such as the \textit{Haar} or \textit{Hadamard} matrix and the $\rmV_i$s are some $d \times d$ diagonal matrices to be either chosen with a standard landmark selection method or learned using an algorithm they provide.

Depending on the $\rmH$ matrix they chose, they can improve the time complexity for the computation of $\rmU\rvx$ from $O(md)$ to $O(m \log{d})$ (\textit{Fast Hadamard transform}) or $O(m)$ (\textit{Fast Haar Transform}).

\subsubsection{QALM in Nyström}

We propose to use our QALM algorithm in order to learn directly the $\rmU$ matrix in the Nyström approximation so that the matrix-vector multiplaction $\rmU \rvx$ is cheap to compute.

Construction remplacement d'une matrice de hadamard par une transformée apprise

\subsection{RBF networks}

Besoin d'éclaircir les liens avec RBF networks

\subsection{nearest-neighbours}

Besoin d'éclaircir les liens avec nearest neighbours

%\bibliographystyle{plain}
%\bibliography{references}
\bibliographystyle{plain}
\bibliography{biblio}
\end{document}
