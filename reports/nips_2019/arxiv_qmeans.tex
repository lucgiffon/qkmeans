\documentclass{article}
\pdfoutput=1

% if you need to pass options to natbib, use, e.g.:
%     \PassOptionsToPackage{numbers, compress}{natbib}
% before loading neurips_2019

% ready for submission
 %\usepackage{neurips_2019}

% OUR ADDITIONS IN PREAMBLE
%====================================================================================
\input{math_commands.tex}

\usepackage{algorithm}
\usepackage{algorithmic}
\usepackage{varwidth}
\usepackage{graphicx}
\usepackage{color}
\usepackage{enumitem}
\usepackage{xspace}

\def\rmGamma{{\mathbf{\Gamma}}}

\newcommand{\addLG}[1]{{\color{green}  #1}}
\newcommand{\addVE}[1]{{\color{blue}  #1}}
\newcommand{\addHK}[1]{{\color{red}  #1}}

\def\datadim{D}
\def\nexamples{N}
\def\nclusters{K}
\def\nfactors{Q}

\renewcommand{\algorithmiccomment}[1]{\hfill #1}
\newcommand{\bigO}[1]{\mathcal{O}\left (#1\right )}
\def\qkmeans{\texttt{QK-means}\xspace}
\def\kmeans{\texttt{K-means}\xspace}
\def\palm{\texttt{palm4MSA}\xspace}

\newcommand\norm[1]{\left \| #1\right \|}

\def\eqdef{:=}

\graphicspath{{./figures/}}

\usepackage{authblk}
\usepackage{fullpage}
%====================================================================================


\newtheorem*{remark}{Remark}
\newtheorem*{proposition}{Proposition}
\newcommand{\diag}{\text{diag}}
\newcommand{\indicator}{\mathds{1}}


\title{QuicK-means: Acceleration of K-means by learning a fast transform}

\author[1]{Luc Giffon}
\author[1]{Valentin Emiya}
\author[2,1]{Liva Ralaivola}
\author[1]{Hachem Kadri}
\affil[1]{Aix Marseille Univ, CNRS, LIS, Marseille, France}
\affil[2]{Criteo}


\begin{document}

\maketitle

\begin{abstract}
	
	\kmeans -- and the celebrated Lloyd algorithm -- is more than the clustering method it was originally designed to be. 
	It has indeed proven pivotal to help increase the speed of many machine learning and data analysis techniques such as indexing, nearest-neighbor search and prediction, data compression; its beneficial use has been shown to carry over to the acceleration of kernel machines (when using the Nyström method). 
	Here, we propose a fast extension of \kmeans, dubbed \texttt{QuicK-means}, that rests on the idea of expressing the matrix of the $\nclusters$ centroids as a product of sparse matrices, a feat made possible by recent results devoted to find approximations of matrices as a product of sparse factors. Using such a decomposition squashes the complexity of the matrix-vector product between the factorized $\nclusters \times \datadim$ centroid matrix $\mathbf{U}$ and any vector from $\mathcal{O}(\nclusters \datadim)$ to $\mathcal{O}(A \log A+B)$, with $A=\min (\nclusters, \datadim)$ and $B=\max (\nclusters, \datadim)$, where $\datadim$ is the dimension of the training data. This drastic computational saving has a direct impact in the assignment process of a point to a cluster, meaning that it is not only tangible at prediction time, but also at training time, provided the factorization procedure is performed during Lloyd's algorithm. We precisely show that resorting to a factorization step at each iteration does not impair the convergence of the optimization scheme and that, depending on the context, it may entail a reduction of the training time. Finally, we provide discussions and numerical simulations that show the versatility of our computationally-efficient  \texttt{QuicK-means} algorithm. 
\end{abstract}


% OUR PAPER
%====================================================================================
\section{Introduction}

\kmeans is one of the most popular clustering algorithms~\cite{hartigan1979algorithm,jain2010data}. It can be used beyond clustering, for other tasks such as indexing, data compression,  nearest-neighbor search and prediction, and local network community detection~\cite{muja2014scalable,van2016local}. \kmeans is also a pivotal process to help increase the speed and the accuracy of many machine learning techniques such as the Nyström approximation of kernel machines~\cite{si2016computationally} and RBF networks~\cite{que2016back}.
%
The  conventional  \kmeans  algorithm  has  a  complexity  of~$\bigO{\nexamples \nclusters \datadim}$ per iteration, where $\nexamples$ is the number of data points, $\nclusters$ the number of clusters and $\datadim$ is the dimension of the data points.
However, the larger the number of clusters, the more iterations are needed to converge~\cite{arthur2006slow}.
%
As data dimensionality and data sample size continue to grow, it is critical to produce viable and cost-effective alternatives to the computationally expensive conventional \kmeans. 
Previous attempts to alleviate the computational issues in \kmeans often relied on batch-, sparsity- and randomization-based methods~\cite{Sculley2010Web, boutsidis2014randomized,shen2017compressed,liu2017sparse}.

Fast transforms have recently received increased attention in machine learning community as they can be used  to speed up random projections~\cite{le2013fastfood,gittens2016revisiting} and to improve landmark-based approximations~\cite{si2016computationally}.
%
These works primarily focused on fast transforms such as Fourier and Hadamard transforms, which are fixed before the learning begins. An interesting question is whether one can go beyond that and learn the fast transform from data. 
%
In a recent paper~\cite{LeMagoarou2016Flexible}, the authors introduced a sparse matrix approximation scheme aimed  at  reducing the  complexity  of  applying  linear  operators  in  high  dimension by   approximately   factorizing   the   corresponding   matrix   into few   sparse   factors. One interesting observation is that fast transforms, such as the  Hadamard  transform  and  the  Discrete  Cosine  transform, can be exactly or approximately decomposed as a product of sparse matrices.
%
In this paper, we take this idea further and investigate attractive and computationally less costly implementations of the \kmeans algorithm by learning a fast transform from data.
%
Specifically, we make the following contributions:
\begin{itemize}
	\item we introduce \texttt{QuicK-means}, a fast extension of \kmeans that rests on the idea of expressing the matrix of the $K$ centroids as a product of sparse matrices, a feat made possible by recent results devoted to find approximations of matrices as a product of sparse factors,
	\item we show that each update step in one iteration of our algorithm  reduces the overall objective, which is enough to guarantee the convergence of \texttt{QuicK-means},
	\item we perform a complexity analysis of our algorithm, showing that the computational gain in \texttt{QuicK-means}  has a direct impact in the assignment process of a point to a cluster, meaning that it is not only tangible at prediction time, but also at training time,
	\item we provide an empirical evaluation of \texttt{QuicK-means}  performance which demonstrates its effectiveness on different datasets in the contexts of clustering and kernel Nystr\"om approximation.
\end{itemize}




\section{Preliminaries}
\label{sec:background}

In this section, we begin by introducing some notation and then review the basics of K-means and give some background about learning fast transforms.

\subsection{Notations}


%\paragraph{Notations}

The notations frequently used in the paper are summarized in Table~\ref{tab:notation}. 
%
Throughout the paper we use $N$ as the number of data samples and $D$ the dimensionality of a data point. $\rmX \in \R^{N \times D}$ is the data matrix. We denote by $[\![K]\!]$, where $K \in \sN$, the set of all $i$ such as $i \in \sN$ and $i < K$.
%
For a given vector $\rvv$, $\rvv[i]$ is the $i$th component of $\rvv$.
%
For a given matrix $\rmM$, the notation $\rmM_{[i]}$ (resp. $\rmM^{[i]}$) refers to the $i$th row (column) of the matrix $\rmA$, the entry at the $i$th row and the $j$th column is denoted by $\rmM[i,j]$, and $\|\rmM\|$ denotes the Frobenius norm. \addHK{other norms?}






%%%%%%%%%%%%%%%%%%%%%%%%%%%%%%%%%%%%%%%%%%%%%%%%%%%%%%%%%%%%
\begin{table}[t]
	\centering
	\begin{tabular}{|r|c|l|}
		\hline
		indices &  $i$, $j$, $m$, $n$, $p$, $q$ &  small  Latin characters  \\
		other integers &  $K$, $Q$, $N$, $\ldots$ &  capital  Latin characters \\
	%	vector spaces\footnotemark & $\mathcal{X}$, $\mathcal{Y}$, $\mathcal{H}$, $\ldots$ & Calligraphic letters \\ 
		vectors (or functions) & $\rvx$, $\rvt$, $\rvk$, $\ldots$ & small bold Latin characters \\
		matrices  & $\rmX$, $\rmU$, $\rmK$, $\ldots$ & capital bold Latin characters \\
		transpose & $\top$ & $\rmX^\top$ transpose of  $\rmX$ \\
		\hline
	\end{tabular}
	\caption{Notations used in this paper.}
	\label{tab:notation}
\end{table}
\addtocounter{footnote}{0}
\footnotetext{We also use the standard notations such as $\mathbb{R}^n$ and $\mathbb{M}_n$.}
%%%%%%%%%%%%%%%%%%%%%%%%%%%%%%%%%%%%%%%%%%%%%%%%%%%%%%%%%%%%






\subsection{K-means}
\label{sec:kmeans}
The K-means algorithm is used to partition a given set of observations $\rmX$ into a predefined $K$ clusters while minimizing the distance between the observations in each partition:

\begin{equation}
\label{eq:kmean_problem}
    \argmin_{\rmU, \rvt} \sum_{k=1}^{K} \sum_{j: \rvt_j = k} ||\rmX_j -\rmU_k||^2,
\end{equation}
% autre écriture de l'objectif de k-means
% = \argmin_{\rmU, \rvt} \sum_{k=1}^{K} c_k + \sum_{k=1}^{K} n_k||\hat{\rmU}_k - \rmU_k||^2
where $\rmU \in \R^{K \times d}$ is the matrix of the cluster's center-points and $\rvt \in  [\![K]\!]^n$ is the indicator vector.

The algorithm (Algorithm \ref{algo:kmeans}) starts with an initialized set of $K$ center-points ($\{\rmU_i \in \R^d\}_{i=1}^{K}$). Each update step $\tau$ is divided into two parts: (i) all observations $\rmX_i$ are assigned to their nearest cluster based on the center-points $\rmU_i^{(\tau-1)}$s at this step (Line \ref{line:kmeans:assignment}) in $\mathcal{O}(ndK)$ operations. (ii) the new center-points $\rmU_i^{(\tau)}$s are computed as the means of the assignated $\rmX_i$ (Line \ref{line:kmeans:compute_means}) for a total of $\mathcal{O}(nd)$ operations.

\begin{algorithm}
\caption{K-means algorithm}
\label{algo:kmeans}
\begin{algorithmic}[1]


\REQUIRE $\rmX \in \R^{n \times d}$, $K$, $\{\rmU_i \in \R^d\}_{i=1}^{K}$
\ENSURE $\{\rmU_i\}_{i=1}^{K}$ the K means of $n$ $d$-dimensional samples
\STATE $\tau \leftarrow 0$
\REPEAT
\STATE $\tau \leftarrow \tau + 1$
\STATE $\rvt^{(\tau)} \leftarrow \argmin_{\rvt \in [\![K]\!]^n} \sum_{i=1}^{n} {||\rmX_i - \rmU^{(\tau-1)}_{\rvt_i}||_2^2}$
\label{line:kmeans:assignment}
\FORALL {$k \in [\![K]\!]^n$}
\STATE $n_k^{(\tau)} \leftarrow |\{i: \rvt^{(\tau)}_i=k\}|$
\label{line:kmeans:count}
\STATE $\rmU^{(\tau)}_k \leftarrow \frac{1}{n_k^{(\tau)}} \sum_{i: \rvt^{(\tau)}_i = k} {\rmX_i}$
\label{line:kmeans:compute_means}
\ENDFOR
\UNTIL{stop criterion}
\RETURN $\rmU^{(\tau)}$
\end{algorithmic}
\end{algorithm}


Once the clusters have been defined, for any $\rvx \in \R^d$ the cluster associated with this $\rvx$ is:

\begin{equation}
\label{eq:assignment_problem_kmeans}
\argmin_{k \in [\![K]\!]} ||\rvx - \rmU_{k}||_2^2 = \argmin_{k \in [\![K]\!]} ||\rmU_{k}||_2^2 - 2 \rmU_{k}^T\rvx
\end{equation}.


We remark here that the computational bottleneck of this assignment lies in the computation of $\rmU_k^T\rvx$ for all $k$. This computation is also encountered in the assignment step (line \ref{line:kmeans:assignment}) of the Algorithm \ref{algo:kmeans}.


\subsection{Learning Fast transforms as the product of sparse matrices}
\label{sec:palm4led}


\begin{algorithm}
	\caption{PALM4MSA algorithm}
	\label{algo:palm4msa}
	\begin{algorithmic}[1]
		
		
		\REQUIRE The matrix to factorize $\rmU \in \R^{K \times d}$, the desired number of factors $Q$, the constraint sets $\mathcal{E}_j$ , $j \in [\![Q]\!]$ and a stopping criterion (e.g., here, a number of iterations $N_{iter}$ ).
		
		\ENSURE $\{\mathcal{S}_1 \dots \mathcal{S}_{Q}\}|\mathcal{S}_j \in \mathcal{E}_j$ such that $\prod_{j=1}^{Q}\mathcal{S}_j \approx \rmU$
		
		\FOR {$i = 0$ to $N_{iter}$}
		\FOR {$j = 1$ to $Q$}
		\STATE  $\rmL_j \leftarrow \prod_{l=j+1}^{Q} \mathcal{S}_{l}^{i}$
		\STATE  $\rmR_j \leftarrow \prod_{l=1}^{j-1} \mathcal{S}_{l}^{i+1}$
		\STATE $c_j^i :> (\lambda^i)^2 ||\rmR_j||_2^2 ||\rmL_j||_2^2$
		\STATE $\mathcal{S}^{i+1}_j \leftarrow P_{\mathcal{E}_j}(\mathcal{S}_j^i - \frac{1}{c_j^i} \lambda^i \rmL_j^T(\lambda \rmL_j \mathcal{S}_j^i \rmR_j - \rmU)\rmR_j^T)$
		\ENDFOR
		\STATE $\hat \rmU := \prod_{j=1}^{Q} \mathcal{S}_j^{i+1}$
		\STATE $\lambda^{i+1} \leftarrow \frac{Trace(\rmU^T\hat\rmU)}{Trace(\hat\rmU^T\hat\rmU)}$
		\ENDFOR
		
		\ENSURE $\lambda, \{\mathcal{S}_1 \dots \mathcal{S}_{Q}\}|\mathcal{S}_j \in \mathcal{E}_j$ such that $\lambda \prod_{j=1}^{Q}\mathcal{S}_j \approx \rmU$
		
	\end{algorithmic}
\end{algorithm}



A popular way for providing concise description of high-dimensional vectors $\rmU \in \R^{K \times d}$ is to compute a sparse representation using a dictionary:
%
\begin{equation}
\rmU^T \approx \rmD\rmGamma
\end{equation}
%
where $\rmD \in \R^{d \times d}$ is a dictionary and $\rmGamma \in \R^{d \times K}$ has sparse columns. Historically, the dictionary is either (i) analytic: $\rmD$ is chosen to give a fast reconstruction of the initial matrix by taking advantage of some fast-transform algorithm (the \textit{Fast Hadamard Transform} for instance) or (ii) learned: $\rmD$ is learned from the data itself to give a good reconstruction of the initial matrix.

Building on the observation that the fast-transform associated with an analytic dictionary can be expressed as the product of sparse matrices $\mathcal{S}_j$ from a set $\mathcal{S}$ of size $M$, \cite{magoarou2014learning} proposes an algorithm to learn a dictionary from the data with sparsity constraints such that this dictionary would be both well-suited with the data and fast to use:
%
\begin{equation}
\rmD = \lambda \prod_{j=1}^{M}\mathcal{S}_j
\end{equation}
%
with $\forall j \in \{1 \ldots M\}$, $\mathcal{S}_j \in \mathcal{E}_j$, $\mathcal{E}_j = \lbrace \rmA \in \R^{a \times a+1}~\text{s.t.}~||\rmA||_0^0 \leq p_j, ||\rmA||_1 = 1 \rbrace$ and $p_j$ being chosen suitably. The $\lambda$ parameter has been added along with the normalization constraint in the $\mathcal{E}_j$ in order to remove scaling ambiguity in the learned $\mathcal{S}_j$.

Considering $\rmGamma$ being a sparse matrice too, it can be renamed as $\rmGamma = \mathcal{S}_{M+1}$. We set $Q = M+1$ and the overall dictionary learning problem can be expressed as the following optimisation problem:
%
\begin{equation}
\label{eq:problem_gribon}
\min_{\{\mathcal{S}_1 \dots \mathcal{S}_Q, \lambda\}} ||\rmU - \lambda \prod_{j=1}^{Q}{\mathcal{S}_j}||_2^2 + \sum_{j=1}^{Q} \delta_j(\mathcal{S}_j)
\end{equation}
%
with the $\delta_j(\mathcal{S}_j) = 0$ if $\mathcal{S}_j \in \mathcal{E}_j$ being the sparsity constraints to satisfy on the associated $\mathcal{S}_j$.

Although this problem is highly non-convex, the authors derive an algorithm from the PALM algorithm \cite{bolte2014proximal}, which they call \textit{Hierarchical PALM4LED} to find a good local minima and give convergence guarantees to learn efficient dictionaries.




\section{QuicK-means}
\label{sec:contribution}

We now introduce our algorithm QuicK-means (abbreviated by QK-means), show its convergence property and analyze its computational complexity.

\subsection{Algorithm}


QuicK-means is an extension of the K-means algorithm in which the matrix of center-points $\rmU$ is constrained to be expressed as a product of sparse matrices $\mathcal{S}_j: j = 1 \ldots Q$. From Equation \ref{eq:kmean_problem} and Equation \ref{eq:problem_gribon} we can write a new K-means optimisation problem with sparse factorization constraint which we call \textit{Q-means}:
%
\begin{equation}
\begin{split}
\label{eq:qmean_problem}
 \argmin_{\{\mathcal{S}_1 \dots \mathcal{S}_Q, \lambda\}, \rvt} & g(\{ \mathcal{S}_1, \ldots,\mathcal{S}_Q \}, \lambda, \rvt)\\
    =\argmin_{\{\mathcal{S}_1 \dots \mathcal{S}_Q, \lambda\}, \rvt} & \sum_{k=1}^{K} \left( \sum_{j: \rvt_j = k} ||\rmX_j -\rmU_k||^2 \right) + \sum_{j=1}^{Q} \delta_j(\mathcal{S}_j) \\
    & s.t. ~ \rmU = \lambda \prod_{j=1}^{Q}{\mathcal{S}_j}
\end{split}
\end{equation}.
%
This problem can be solved using Algorithm \ref{algo:qmeans} which is a simple extension of the K-means algorithm (Algorithm \ref{algo:kmeans}) and is guaranteed to converge. 

\subsection{Convergence analysis}

\begin{algorithm}[t]
	\caption{QuicK-means algorithm}
	\label{algo:qmeans}
	\begin{algorithmic}[1]
		
		
		\REQUIRE $\rmX \in \R^{n \times d}$, $K$, $\{\mathcal{S}_1 \dots \mathcal{S}_{Q}\}|\mathcal{S}_j \in \mathcal{E}_j$
		\ENSURE $\{\mathcal{S}_1 \dots \mathcal{S}_{Q}\}|\mathcal{S}_j \in \mathcal{E}_j$ such that $\prod_{j=1}^{Q}\mathcal{S}_j \approx \rmU$ the K means of $n$ $d$-dimensional samples
		\STATE $\tau \leftarrow 0$
		\REPEAT
		\STATE $\tau \leftarrow \tau + 1$
		\STATE $\rvt^{(\tau)} \leftarrow \argmin_{\rvt \in [\![K]\!]^n} \sum_{i=1}^{n} {||\rmX_i - \rmU^{(\tau -1)}_{\rvt(i)}||_2^2}$
		\label{line:qmeans:assignment}
		\FORALL {$k \in [\![K]\!]$}
		\label{line:qmeans:startkmeans}
		\STATE $n_k^{(\tau)} \leftarrow |\{i: \rvt^{(\tau)}_i=k\}|$
		\STATE $\hat{\rmX}^{(\tau)}_k \leftarrow \frac{1}{n^\tau_k} \sum_{i: \rvt^\tau_i = k} {\rvx_i}$
		\ENDFOR
		\label{line:qmeans:endkmeans}
		\STATE $\rmA^{(\tau)} \leftarrow \mathcal{D}_{\sqrt{\rvn^{(\tau)}}}~\hat{\rmX}^{(\tau)} $
		\STATE $\{\mathcal{S}^{(\tau)}_1 \dots \mathcal{S}^{(\tau)}_{Q}\}, \lambda^{(\tau)} \leftarrow \argmin_{\{\mathcal{S}_1 \dots \mathcal{S}_Q, \lambda\}} ||\rmA^{(\tau)} - ~\lambda\prod_{j=0}^{Q}{\mathcal{S}_j}||_\mathcal{F}^2 + \sum_{j=0}^{Q} \delta_j(\mathcal{S}_j)$
		\STATE $\rmU^{(\tau)}_k \leftarrow \lambda^{(\tau)} \prod_{j=1}^{Q}{\mathcal{S}_j^{(\tau)}}$
		
		\UNTIL{stop criterion}
	\end{algorithmic}
\end{algorithm}



To show this convergence, we need to show that each update step in one iteration $\tau$ of the algorithm actually reduces the overall objective. To this end, we start by re-writing the objective at a given time-step $\tau$:
%
\begin{equation}
\begin{split}
\label{eq:qmean_problem_2}
    g(&\{ \mathcal{S}_1^{(\tau)}, \ldots,\mathcal{S}_Q^{(\tau)} \}, \lambda^{(\tau)}, \rvt^{(\tau)})\\
    = & \sum_{k=1}^{K} \left( \sum_{j: \rvt^{(\tau)}_j = k} ||\rmX_j - \rmU^{(\tau)}_k||^2 \right) + \sum_{j=1}^{Q} \delta_j(\mathcal{S}_j^{(\tau)})\\
    & s.t. ~ \rmU = \lambda^{(\tau)} \prod_{j=1}^{Q}{\mathcal{S}_j^{(\tau)}}
\end{split}
\end{equation}.
%
We then assess whether or not this objective diminishes at each time-step in Algorithm \ref{algo:qmeans}.




\paragraph{Assignment step (Line \ref{line:qmeans:assignment})} For a fixed $\rmU^{(\tau-1)}$ the new indicator vector $\rvt^{(\tau)}$ is defined as:
%
\begin{equation}
\label{eq:qmean_problem_U_fixed}
 \rvt^{(\tau)}_i = \argmin_{k \in [\![K]\!]} ||\rmX_i - \rmU^{(\tau-1)}||_2^2
\end{equation}
%
for any $i \in [\![n]\!]$. This step is exactly identical in the K-means algorithm (Algorithm \ref{algo:kmeans}) and is clearly minimizing the objective function \textit{w.r.t.} to vector $\rvt$.

\paragraph{Centroids computation step (Line \ref{line:qmeans:startkmeans} to \ref{line:qmeans:endkmeans})} For a fixed $\rvt^{(\tau)}$, the new sparsely-factorized centroids are solutions of the following subproblem:
%
\begin{equation}
\label{eq:qmeans_problem_t_fixed}
\begin{split}
 \argmin_{\{ \mathcal{S}_1, \ldots,\mathcal{S}_Q\}, \lambda} & g(\{ \mathcal{S}_1, \ldots,\mathcal{S}_Q\}, \lambda, \rvt^{(\tau)}) \\
 = \argmin_{\{ \mathcal{S}_1, \ldots,\mathcal{S}_Q\}, \lambda} &\sum_{k=1}^{K} \left( \sum_{j: \rvt^{(\tau)}_j = k} ||\rmX_j - \rmU_k||^2_2 \right) + \sum_{j=1}^{Q} \delta_j(\mathcal{S}_j)  \\
 = \argmin_{\{ \mathcal{S}_1, \ldots,\mathcal{S}_Q\}, \lambda} & ||\mathcal{D}_{\sqrt{\rvn^{(\tau)}}}~(\hat{\rmX}^{(\tau)} - \rmU)||_{\mathcal{F}} ^ 2  \\
 &+ \sum_{k=1}^{K} c_k^{(\tau)} + \sum_{j=1}^{Q} \delta_j(\mathcal{S}_j)\\
 & s.t. ~ \rmU = \lambda \prod_{j=1}^{Q}{\mathcal{S}_j}
\end{split} 
\end{equation}
%
where :
%
\begin{itemize}
 \item $\sqrt{\rvn^{(\tau)}} \in {\R^{K}}$ is the pair-wise square root of the vector indicating the number of observations in each cluster at step $\tau$: $\rvn_k^{(\tau)} = |\{i: \rvt^\tau_i = k\}|$;
 \item $\mathcal{D}_\rvv \in \R^{K \times K}$ refers to a diagonal matrix with entries in the diagonal from a vector $\rvv$;
 \item $\hat{\rmX}^{(\tau)} \in \R^{K \times d}$ refers to the real centroid matrix obtained at step $\tau$ \textit{w.r.t} the indicator vector at this step $\rvt^{(\tau)}$: $\hat{\rmX}^{(\tau)}_k = \frac{1}{\rvn_k}\sum_{j:\rvt^{(\tau)}_j = k} {\rmX_j}$. When $\rvt^{(\tau)}$ is fixed, this is constant.
 \item $c_k^{(\tau)} = \sum_{j: \rvt^{(\tau)}_j = k}^{}||\rmX_j - \hat{\rmX}_k^{(\tau)}||$ is constant \textit{w.r.t} $\{ \mathcal{S}_1, \ldots,\mathcal{S}_Q\}$ and $\lambda$.
\end{itemize}

Again,  the minimization of the overall objective $g$ from Equation \ref{eq:qmean_problem_2} is clear since the $\{ \mathcal{S}_1, \ldots,\mathcal{S}_Q\}$ and $\lambda$ are precisely chosen to minimize $g$.

Note that the formulation of the problem in Equation \ref{eq:qmeans_problem_t_fixed} shows the connection between the K-means and \textit{Hierarchical PALM4LED} objectives, which allows us to combine them without trouble. Indeed, we can set
%
\begin{equation*}
\rmA^{(\tau)} = \mathcal{D}_{\sqrt{\rvn^{(\tau)}}}~\hat{\rmX}^{(\tau)}
\end{equation*}
and
\begin{equation*}
\rmB^{(\tau)} = \mathcal{D}_{\sqrt{\rvn^{(\tau)}}}~\rmU = \mathcal{D}_{\sqrt{\rvn^{(\tau)}}}~\lambda \prod_{j=1}^{Q}{\mathcal{S}_j} = \lambda \prod_{j=0}^{Q}{\mathcal{S}_j}
\end{equation*}
%
with $\mathcal{S}_0$ fixed and equal to $\mathcal{D}_{\sqrt{\rvn^{(\tau)}}}$. The Equation \ref{eq:qmeans_problem_t_fixed} can then be rewritten as
%
\begin{equation}
\begin{split}
 \argmin_{\{ \mathcal{S}_1, \ldots,\mathcal{S}_Q\}, \lambda} & ||\rmA^{(\tau)} - \rmB^{(\tau)}||_{\mathcal{F}} ^ 2  +  \sum_{j=0}^{Q} \delta_j(\mathcal{S}_j)\\
 s.t. &~ \rmB^{(\tau)} = \lambda \prod_{j=0}^{Q}{\mathcal{S}_j}
\end{split}
\end{equation}

Since \textit{Hierarchical PALM4LED} successivly updates the $\mathcal{S}_j$s independently and in an alternating fashion, we can still use \textit{PALM4LED} in to solve this problem with the $\mathcal{S}_0$ fixed.





The factorization of $\rmU$ could then be used in an ulterior algorithm that involves a matrix-vector multiplication with $\rmU$: typically any algorithm involving the assignment of some data points to one of the clusters (Equation \ref{eq:assignment_problem_kmeans}). Such applications of our proposed algorithm are discussed in Section \ref{sec:uses}.

\subsection{Complexity analysis}

In the following, we call a matrix or a vector dense if it is not sparse.

We give a thorough analysis of the Q-means algorithm and we show the theoretical benefits of using our method compared to the classical K-means algorithm.

%We first give essential knowledge on sparse and dense matrix multiplication and we study the complexity of the PALM4MSA algorithm proposed in \cite{magoarou2014learning} \addLG{cette source ne correspond pas à la dernière version du papier (préférable)}. We then show how to take advantage of the sparse factorization of the K-means matrix both while forming it and using it in further algorithms. 

\subsubsection{Preliminaries}

We start by giving some general information about the complexity of some standard linear algebra operations then we analyse precisely the cost of the PALM4MSA algorithm proposed in 
\cite{magoarou2014learning} \addLG{cette source ne correspond pas à la dernière version du papier (préférable)} and we finally recall the complexity involved in the K-means algorithm.

\paragraph{Complexity of a matrix multiplication between a dense matrix and a dense vector.}
Let $\rmA$ be a $K \times d $ matrix and $\rvv$ be a $d$ dimensional dense vector. The matrix-vector product $\rmA\rvv$ can be done in $\mathcal{O}\left(Kd \right)$ operations.

\paragraph{Complexity of a matrix multiplication between two dense matrices.}
Let $\rmA$ be a $K \times d $ matrix and $\rmB$ be a $d \times N$ matrix, then computing $\rmA \rmB$ can be done in $\mathcal{O}\left (KdN \right )$ operations.

\paragraph{Complexity of a matrix multiplication between a sparse matrix and a dense vector.}
Let $\rmA$ be a $K \times d$ sparse matrix with $\mathcal{O}(p)$ non-zero entries and $\rvv$ be a $d$ dimensional dense vector. The matrix-vector product $\rmA\rvv$ can be done in $\mathcal{O}\left(p \right)$ operations.

\paragraph{Complexity of a matrix multiplication between a sparse matrix and a dense matrix.}
Let $\rmA$ be a $K \times d$ sparse matrix with $\mathcal{O}(p)$ non-zero entries and $\rmB$ be a $d \times N$ dense matrix. The matrix-matrix product $\rmA\rmB$ can be done in $\mathcal{O}\left(p N\right)$ operations.

\paragraph{Complexity of a matrix multiplication between a sparse matrix and a sparse matrix.}
Let $\rmA$ be a $K \times d $ sparse matrix and $\rmB$ be a $d \times N$ sparse matrix, both having $\mathcal{O}(p)$ non-zero values.
To the best of our knowledge, the best achievable complexity for the matrix-matrix product in this general scenario is $\mathcal{O}(p~\min{\{K, N\}})$. We remark here that the number of values in such resulting matrix is $\mathcal{O}(p^2)$.
%The $\min$ term appears because we can either compute $\rmA\rmB$ or $(\rmB^T\rmA^T)^T$ for the same result.

\paragraph{Complexity of the evaluation of Q sparse factors: $\prod_{j=1}^{Q}\mathcal{S}_j$}
Let $\mathcal{S}_j$ be a sparse matrice of $p$ non-zero values for any $j \in [\![Q]\!]$. Let also the resulting matrix be of size $K \times d$.  Finally, let $\mathcal{S}_1$ be a $K \times q$ matrix, $\mathcal{S}_Q$ be a $q \times d$. and all the other $\mathcal{S}_j$ be $q \times q$ matrices; $q$ is set to be the minimum of $\{K, d\}$. We consider, for the sake of simplicity, that $p$ is $\mathcal{O}(q)$: e.g. there is one value by row or column in the $\mathcal{S}_j$s. In this case (which is considered to be our case), the product $\prod_{j=1}^{Q}\mathcal{S}_j$ can be done in time $\mathcal{O}(Qpq)$: once a sparse-sparse matrix multiplication then $Q-2$ times the sparse-dense matrix multiplication.

\paragraph{Complexity of the multiplication between Q sparse factors and a dense vector}
Let $\rmS$ be a short-hand for $\prod_{j=1}^{Q}\mathcal{S}_j$ that has been detailed above. $\rmS$ is kept as a factorization. Let also $\rvv$ be a $d$ dimensional dense vector. Then the product $\rmS \rvv$ can be computed right to left in time $\mathcal{O}(Qp)$ operations.
%If $\rmB$ is also sparse, with $b$ non-zero entries, then the bound is $\mathcal{O}\left ( \min\left ( a \min\left (b, N \right ), b \min\left (a, M\right ) \right ) \right )$ where $M$ is the number of rows in $\rmA$.
%This is a naive upper bound for sparse matrices, some tighter bound may be found.

\paragraph{Complexity of algorithm \textit{PALM4MSA}.}
Each iteration takes $\mathcal{O}(Q(Qpq + Kd + q^2\log q^2) + K^2d)$\addLG{$q^2 \log q^2$ peut être remplacé par $q^2 + p\log p$ grâce à l'algo quickselect}. In the following analysis, we refer to the lines in Algorithm 2 of \cite{magoarou2014learning}. This algorithm is repeated here for simplicity (Algorithm \ref{algo:palm4msa}). Note that the displayed complexities are for one full iteration of the algorithm.

\begin{description}[leftmargin=\parindent,labelindent=\parindent]
 
 \item [Line 3] The $\rmL$s can be \textit{precomputed} incrementaly for each iteration $i$, involving a total cost of $\mathcal{O}(Qpq)$ operations: for all $j < Q$, $\rmL_j = \rmL_{j+1} \mathcal{S}^i_{j+1}$; for $j = Q$, $\rmL_j = \textbf{Id}$;
 \item [Line 4] The $\rmR$s is computed incrementaly for each iteration $j$: $\rmR_j = \mathcal{S}^{i+1}_{j-1} \rmR_{j-1}$ if $j > 1$; $\rmR_j = \textbf{Id}$ otherwise. This costs an overall $\mathcal{O}(Qpq)$ operations;
 \item [Line 5] The time complexity for computing the operator norm of a matrix of dimension $K \times q$ is $\mathcal{O}(Kq)$, which leads a $\mathcal{O}(QKq)$ number of operations for this line \addLG{à éclaircir...};
 \item [Line 6] \addLG{avec Valentin on avait trouvé O(Kd min \{K, d\}) mais je ne suis plus d'accord} Taking advantage of the decompositions of $\rmL$ and $\rmR$ as products of sparse factors, the time complexity of this line ends up being $\mathcal{O}(Q(Qpq + Kd + q^2\log q^2))$ for a complete iteration: the $\mathcal{O}(Qpq)$ part comes from the various sparse-dense matrix multiplications with $\rmR$ and $\rmL$; the $\mathcal{O}(Kd)$ part comes from the pairwise substraction inside the parentheses and the $\mathcal{O}(q^2 \log q^2)$ part from the projection operator that involves sorting of the inner matrix.\addLG{$q^2 \log q^2$ peut être remplacé par $q^2 + p\log p$ grâce à l'algo quickselect}
 \item [Line 8] The reconstructed $\hat \rmU$ can be computed from the $\rmR_{Q-1}$ and $\mathcal{S}_Q^{i+1}$ obtained just before: $\hat \rmU = \mathcal{S}_Q^{i+1} \rmR_{Q-1}$. This sparse-dense matrix multiplication cost a time $\mathcal{O}(pq)$.
 \item [Line 9] \addLG{Avec valentin, on avait écrit $O(min\{K, d\} ^ 3)$ mais je ne suis plus d'accord}The computational complexity of this line is majored by the matrix multiplications that cost $\mathcal{O}(K^2d)$ operations.
\end{description}

Adding up the complexity for each of those lines and then simplifying gives an overall complexity of $\mathcal{O}(Q(Qpq + Kd + q^2\log q^2) + K^2d)$\addLG{$q^2 \log q^2$ peut être remplacé par $q^2 + p\log p$ grâce à l'algo quickselect}. Note that $\mathcal{O}(Kq)$ is majored by $\mathcal{O}(Kd)$ since $d \geq q$.

\begin{algorithm}
\caption{PALM4MSA algorithm}
\label{algo:palm4msa}
\begin{algorithmic}[1]


\REQUIRE The matrix to factorize $\rmU \in \R^{K \times d}$, the desired number of factors $Q$, the constraint sets $\mathcal{E}_j$ , $j \in [\![Q]\!]$ and a stopping criterion (e.g., here, a number of iterations $N_{iter}$ ).

\ENSURE $\{\mathcal{S}_1 \dots \mathcal{S}_{Q}\}|\mathcal{S}_j \in \mathcal{E}_j$ such that $\prod_{j=1}^{Q}\mathcal{S}_j \approx \rmU$

\FOR {$i = 0$ to $N_{iter}$}
\FOR {$j = 1$ to $Q$}
\STATE  $\rmL_j \leftarrow \prod_{l=j+1}^{Q} \mathcal{S}_{l}^{i}$
\STATE  $\rmR_j \leftarrow \prod_{l=1}^{j-1} \mathcal{S}_{l}^{i+1}$
\STATE $c_j^i :> (\lambda^i)^2 ||\rmR_j||_2^2 ||\rmL_j||_2^2$
\STATE $\mathcal{S}^{i+1}_j \leftarrow P_{\mathcal{E}_j}(\mathcal{S}_j^i - \frac{1}{c_j^i} \lambda^i \rmL_j^T(\lambda \rmL_j \mathcal{S}_j^i \rmR_j - \rmU)\rmR_j^T)$
\ENDFOR
\STATE $\hat \rmU := \prod_{j=1}^{Q} \mathcal{S}_j^{i+1}$
\STATE $\lambda^{i+1} \leftarrow \frac{Trace(\rmU^T\hat\rmU)}{Trace(\hat\rmU^T\hat\rmU)}$
\ENDFOR

\ENSURE $\lambda, \{\mathcal{S}_1 \dots \mathcal{S}_{Q}\}|\mathcal{S}_j \in \mathcal{E}_j$ such that $\lambda \prod_{j=1}^{Q}\mathcal{S}_j \approx \rmU$

\end{algorithmic}
\end{algorithm}


%Also, we consider the scenario when the first factor is in $\R^{K \times d}$ and all the others are in $\R^{d \times d}$. Finaly, we set the number of values in each sparse matrix to be the same and equal to $p$.

%For each of the $Q$ factors, the complexity is $\mathcal{O}(dpQ + Kd + d^2))$:
%
%\begin{itemize}
% \item Lines 3 and 4: $\mathcal{O}(dpQ)$ by computing the products right to left and taking advantage of the sparsity of the factors;
% \item Line 5: $\mathcal{O}(Kd + d^2)$ because $\rmR \in \R^{K \times d}$ and $\rmL \in \R^{d \times d}$;
% \item Line 6: $\mathcal{O}(dpQ + Kd)$ for the sparse product and the projection operation.
%\end{itemize}
%
%The two last statement of each iteration are in time $\mathcal{O}(dpQ)$, again taking advantage of the sparse factorisation.
%
%\begin{itemize}
% \item Line 8: $\mathcal{O}(dpQ)$;
% \item Line 9: $\mathcal{O}(dpQ)$.
%\end{itemize}

\paragraph{Complexity of algorithm \textit{Hierarchical PALM4MSA}.}

The hierarchical version of the algorithm corresponds to the same algorithm repeated $Q$ times. The overall complexity is then $\mathcal{O}(Q^2(Qpq + Kd + q^2\log q^2) + K^2d)$. 

\paragraph{Complexity of algorithm K-means}
We recall here that the K-means algorithm complexity is majored by its cluster assignation step (Line~\ref{line:kmeans:assignment} of Algorithm~\ref{algo:kmeans}) which requires $\mathcal{O}(ndK)$ operations. This comes out from the assignation of one cluster to each observation of the data set. this assignation step involves a matrix-vector multiplication as described in Equation~\ref{eq:assignment_problem_kmeans}.

\subsubsection{Complexity of the Q-means algorithm}

We now show how the cluster assignation step of our method is computationaly less expensive than the one of the previous K-means algorithm and how this feature might even fasten the computation of the K-means algorithm in general.

\paragraph{Cluster assignation}

In Equation~\ref{eq:assignment_problem_kmeans}, we have seen that the cost of the assignation of an observation to a cluster is majored by the matrix-vector multiplication between the cluster matrix and the observation. Using our method, this cost is reduced from $\mathcal{O}(Kd)$ operations to $\mathcal{O}(Qp)$ operations. \addLG{Envolée: In the experiments, we will se that $Q$ can be chosen sufficiently little so that this complexity becomes to $\mathcal{O}(p \log q)$ operations.}

\paragraph{Q-means factorization construction}

This fastening of the assignation step can also be used while constructing the Q-means factorization: the complexity of this step is reduced from $\mathcal{O}(ndk)$ to $\mathcal{O}(npQ)$. Nevertheless, this reduction has to be taken cautiously because of the extra-step in the Q-means algorithm: the inner call to the hierarchical-PALM4MSA at each iteration that costs $\mathcal{O}(Q^2(Qpq + Kd + q^2\log q^2) + K^2d)$. This leaves us with an overall time complexity of $\mathcal{O}(Q^2(Qpq + Kd + q^2\log q^2) + K^2d) + npQ$. \addLG{We note here that our use of the PALM4MSA algorithm doesn't rely on the number of sample in the full dataset: if $n$ is large enough compared to $K$ and $d$, then the complexity is majored by $\mathcal{O}(np \log q)$}

%We now note that, in practice, $Q$ is supposed to be small compared to $K$ or $d$, and that $p$ should be of the same order than $d$. In that case, we can simplify the complexity of the final algorithm to $\mathcal{O}(Kd + d^2)$.

%\paragraph{Complexity of Kmeans (algorithm~\ref{algo:kmeans}).}
%Each iteration takes $\mathcal{O}\left (ndk\right )$
%\begin{itemize}
%\item Assignment, line~\ref{line:kmeans:assignment}: $\mathcal{O}\left (ndk\right )$\\
%it is dominated by the computation of $\rmU \rmX^T$ using $\left \|\rvx-\rvu\right \|_2^2=\left \|\rvx\right \|_2^2+\left \|\rvu\right \|_2^2-2\left <\rvx, \rvu\right >$
%\item Computing size of cluster, line~\ref{line:kmeans:count}: $\mathcal{O}\left (n \right )$\\ it consists of one pass over $\rvt$.
% \item Updating the centroids, line~\ref{line:kmeans:compute_means}: $\mathcal{O}\left (nd\right )$\\
%since each example is summed once.
%\end{itemize}

%\paragraph{Complexity of Q-means (algorithm~\ref{algo:qmeans}).}

%Naively, each iteration of the \textit{Q-means} algorithm has the same complexity than K-means with an additional $\mathcal{O}(Kd + d^2)$ for the \textit{Hierarchical PALM4LED} step. This leads to the overall complexity of $\mathcal{O}(Kdn + d^2)$. In the case where the dimensionality of the data is negligible in front of the size of the dataset, this is of the same order of complexity than the vanilla K-means algorithm, e.g. $\mathcal{O}(ndk)$. Nevertheless, we can already take advantage of the decomposition of the center-point matrix in the assignment step of the Algorithm: indeed, when the intermediate center-point matrix $\rmU^{(\tau-1)}$ has the sparse factorization constraint, assigning a $n$ data points to their cluster can be done in time $\mathcal{O}(n K \log d)$ operations, which leads to the new complexity for \textit{Q-means}: $\mathcal{O}(nK\log d + Kd + d^2)$

%Once the decomposed center-point matrix has been obtained, the assignation of a new data point to a cluster becomes $\mathcal{O}(nK\log d)$ which would be usefull in many machine learning application usually using the K-means algorithm.

%!TEX root=aaai2020_qmeans.tex


\section{Experiments and applications}
\label{sec:uses}

\subsection{Experimental setting}
\label{sec:uses:settings}

\paragraph{Implementation details.}
The simulations have been conducted in Python, including for the \palm algorithm.
Running times are measured on computer grid with 3.8GHz-CPUs (2.5GHz in Figure~\ref{fig:time_csr}).
Fast operators $\rmV$ based on sparse matrices $\rmS_q$ are implemented with \texttt{csr\_matrix} objects from the \texttt{scipy.linalg} package. 
While more efficient implementations may be beneficial for larger deployment, our implementation is sufficient as a proof of concept for assessing the performance of the proposed approach as illustrated by running times benchmarking in the Section~\ref{seq:sparse_factor_benchmarking} of supplementary material. \addLG{Je ne sais pas trop comment faire des références au supplementary material}
% In particular, the running times of fast operators of the form $\prod_{q\in\intint{\nfactors}}{\rmS_q}$ have been measured when applying to random vectors, for several sparsity levels: 
% as shown in Figure~\ref{fig:time_csr}, they are significantly faster than dense operators -- implemented as a \texttt{numpy.ndarray} matrix --, especially when the data size is larger than $10^3$.


% \begin{figure}[tbh]
% \centering
% \includegraphics[width=.5\textwidth]{RunningTime4VaryingSparsity.png}
% \caption{Running times, averaged over 30 runs, when applying dense or fast $\datadim \times \datadim$ operators to a set of 100 random vectors. The number of factors in fast operators equals $\log_2\left (\datadim\right )$ and the sparsity level denotes the number of non-zero coefficients per row and per column in each factor.}
% \label{fig:time_csr}
% \end{figure}

%\begin{figure}[tbh]
%\centering
%\includegraphics[width=.8\textwidth]{Run_time_sparsity_2.png}
%\caption{Running times, averaged over 30 runs, when applying dense or product of fast operators to a set of 100 random vectors. The number of factors in fast operators equals $\log_2\left (\#~row\right )$ and the sparsity level denotes the number of non-zero coefficients per row and per column in each factor.}
%\label{fig:time_csr_fixed_row_size}
%\end{figure}


\paragraph{Datasets.}
We present results on real-world and toy datasets summarized in supplementary material (Table \ref{table:data}) \addLG{Comment faire référence au supplementary material}. On the one hand, the real world datasets \texttt{MNIST}~\cite{lecun-mnisthandwrittendigit-2010} and \texttt{Fashion-Mnist}~\cite{Pedregosa2011Scikit} %and \texttt{Labeled Faces in the Wild}~\cite{Huang07e.:labeled} (\texttt{LFW}) 
are used to show --- quantitatively and qualitatively --- the good quality of the obtained centers when using our method \qkmeans. On the other hand, we use the \texttt{blobs} synthetic dataset from \texttt{sklearn.dataset} to show the speed up offered by our method \qkmeans when the number of clusters and the dimensionality of the data are sufficiently large.
%The code of our method \qkmeans is available on request and will be available online soon. \addLG{je serais d'avis de ne pas dire ça mais soit de dire qu'il est déjà disponible, soit de ne rien dire. Sachant qu'on ne peut pas dire qu'il est déjà disponible en ligne avant le processus de reviewing}


\paragraph{Algorithm settings.} 
\qkmeans is used with $Q\eqdef\log_2\left (A\right )$ sparse factors, where  $A\eqdef\min\left (\nclusters, \datadim\right )$. 
All factors $\rmS_q$ are with shape $A \times A$ except, depending on the shape of $\rmA$, the leftmost one ($\nclusters\times A$) or the rightmost one ($A\times\datadim$). 
The sparsity constraint of each factor $\rmS_q$ is set in $\mathcal{E}_q$ and is governed by a global parameter denoted as \textit{sparsity level}, which indicates the desired number of non-zero coefficients in each row and in each column of $\rmS_q$. 
Since the projection onto this set of structured-sparsity constraints may be computationally expensive, this projection is relaxed in the implementation of \palm and only guarantees that the number of non-zero coefficients in each row and each column is at least the sparsity level, as in~\cite{LeMagoarou2016Flexible}.
The actual number of non-zero coefficients in the sparse factors is measured at the end of the optimization process and reported in the results.
%The sparse factors are updated using the \palm rather than its hierarchical version, since we observed that this was a better choice in terms of computational cost, with satisfying approximation results (See Figure~\ref{fig:mnist:objfun}~and~\ref{fig:fmnist:objfun}).
Additional details about \palm are given in Appendix~\ref{sec:app:palm4msa}.
The stopping criterion of \kmeans and \qkmeans consists of a tolerance set to $10^{-6}$ on the relative variation of the objective function and a maximum number of iterations set to 10 for the \texttt{Blobs} dataset and to 20 for others. 
The same principle governs the stopping criterion of \palm with a tolerance set to $10^{-6}$ and a maximum number of iterations set to 300. Each experiment have been replicated using different seed values for random initialisation. 
Competing techniques share the same seed values, hence share the same initialisation of centers.

%\subsection{Sparse factors multiplication}
%
%\subsubsection{Sparse factor object}

%\todo[inline]{Parler ici de la configuration de \qkmeans: $Q\eqdef\log_2\left (A\right )$, critère d'arrêt (nombre d'itération, tolérance), ordre des mises à jours, palm4msa plutôt que la version hiérarchique, taille des matrices $\rmS_q$, scaling coefficient, définition de 	$\mathcal{E}_q$.
%}

\subsection{Clustering}

\begin{figure*}[h]
\begin{subfigure}[b]{.49\textwidth}
\includegraphics[width=\textwidth]{mnist30_objective.png}
\caption{MNIST, $\nclusters=30$: objective function.}
\label{fig:mnist:objfun}
\end{subfigure}
\begin{subfigure}[b]{.49\textwidth}
\includegraphics[width=\textwidth]{fashmnist30_objective.png}
\caption{Fashion-MNIST, $\nclusters=30$: objective function.}
\label{fig:fmnist:objfun}
\end{subfigure}
\begin{subfigure}[t]{.49\textwidth}
\includegraphics[width=\textwidth]{mnist30_kmeans_centroids.png}
\caption{\kmeans centers.}
\label{fig:mnist:kmeans:centers}
\end{subfigure}
\begin{subfigure}[t]{.49\textwidth}
\includegraphics[width=\textwidth]{fashmnist30_kmeans_centroids.png}
\caption{\kmeans centers.}
\label{fig:fmnist:kmeans:centers}
\end{subfigure}
\begin{subfigure}[t]{.49\textwidth}
\includegraphics[width=\textwidth]{mnist30_qkmeans_centroids.png}
\caption{\qkmeans centers.}
\label{fig:mnist:qkmeans:centers}
\end{subfigure}
\begin{subfigure}[t]{.49\textwidth}
\includegraphics[width=\textwidth]{fashmnist30_qkmeans_centroids.png}
\caption{\qkmeans centers.}
\label{fig:fmnist:qkmeans:centers}
\end{subfigure}
% \begin{subfigure}[t]{.49\textwidth}
% \includegraphics[width=\textwidth]{mnist30_hqkmeans_centroids.png}
% \caption{Hierarchical-\palm \qkmeans centers.}
% \label{fig:mnist:hqkmeans:centers}
% \end{subfigure}
% \begin{subfigure}[t]{.49\textwidth}
% \includegraphics[width=\textwidth]{fashmnist30_hqkmeans_centroids.png}
% \caption{Hierarchical-\palm \qkmeans centers.}
% \label{fig:fmnist:hqkmeans:centers}
% \end{subfigure}
\caption{Clustering results on MNIST (left) and Fashion-MNIST (right) for $\nclusters=30$ clusters.}
\label{fig:clustering:realdata}
\end{figure*}

\todo[inline]{Pour gagner de la place facilement: mettre les centeres en 6x5 et faire une ligne MNIST avec courbes et le 2 jeux de centeres, puis une autre ligne de figures avec les courbes Fashion-MNIST et les 2 jeux de centeres}

\paragraph{Approximation quality.} One important question is the ability of the fast-structure model to fit arbitrary data.
Indeed, no theoretical result about the expressivity of such models is currently available.
In order to assess this approximation quality, the MNIST and Fashion-MNIST data have been clustered into $\nclusters=30$ clusters by \kmeans and \qkmeans with several sparsity levels.
Results are reported in Figure~\ref{fig:clustering:realdata}.
In Figures~\ref{fig:mnist:objfun} and~\ref{fig:fmnist:objfun}, one can observe that the objective function of \qkmeans is decreasing in a similar way as \kmeans over iterations.
In particular, the use of the fast-structure model does not seem to increase the number of iteration necessary before convergence.
At the end of the iterations, the value of objective function for \qkmeans is slightly above that of \kmeans.
As expected, the sparser the model, the more degradation in the objective function.
However, even very sparse models do not degrade the results significantly. These Figures also demonstrate the convergence property of the \qkmeans algorithm when using the standard, proved convergent, \textit{Palm4MSA} algorithm: in this case, the objective function is always non-increasing whereas the \qkmeans version with \textit{Hiearchical Palm4MSA}, not guaranteed to converge, suffers a small bump in its objective function (see Figure~\ref{fig:fmnist:objfun} iteration~6).
The approximation quality can be assessed visually, in a more subjective and interpretable way, in Figures~\ref{fig:mnist:kmeans:centers} to~\ref{fig:fmnist:hqkmeans:centers} where the obtained centers are displayed as images.
Although some degradation may be observed in some images, one can note that each image obtained with \qkmeans clearly represents a single visual item without noticeable interference with other items.

\paragraph{Clustering assignation times.}
Higher dimensions are required to assess the computational benefits of the proposed approach, as shown here.
The assignation times of the clustering procedure were measured on the \texttt{Blobs} dataset.
The center matrices are with shape $\nclusters \times \datadim$ with $\datadim=2000$  and $\nclusters\in\left \lbrace 128, 256, 512\right \rbrace$.
Results reported in Figure~\ref{fig:clustering:blobs:assignation_time} show that in this setting and with the current implementation, the computational advantage of \qkmeans is observed in high dimension, for $\nclusters=256$ and $\nclusters=512$ clusters. It is worth noticing that when $K$ increases, the running times are not affected that much for \qkmeans while it significantly grows for \kmeans. These trends are directly related to the number of model parameters that are reported in the figure.


%\todo[inline]{Montrer ensuite les temps d'assignation en mode batch 5000 sur blobs, cf. Figure~\ref{fig:clustering:blobs:assignation_time}. Objectif: montrer qu'à partir d'une certaine dimension, \qkmeans est plus rapide.}
%on-line\footnote{Anonymous URL.}.

% \begin{figure}[tbh]
% \centering
% \includegraphics[width=.45\textwidth]{blobs_assignation_time.png}
% \caption{Clustering Blobs data: running times of the assignation step, averaged over 5 runs. The vertical black lines are the standard deviation w.r.t. the runs and the average number of parameters actually learned  in the models are reported above those lines.\addVE{to be completed}.}
% \label{fig:clustering:blobs:assignation_time}
% \end{figure}

\todo[inline]{supplementary material montrant l'impact de la sparsity factor}

\subsection{Nearest-neighbor search in a large dataset}
The Nearest-neighbor search is a fundamental task that suffers from computational limitations when the dataset is large.
Fast strategies have been proposed, e.g., using kd trees or ball trees.
One may also use a clustering strategy to perform an approximate nearest-neighbor search: the query is first compared to $\nclusters$ centers computed beforehand by clustering the whole dataset, and the nearest neighbor search is then performed among a lower number of data points, within the related cluster.
We compare this strategy using \kmeans and \qkmeans against the \texttt{scikit-learn} implementation~\cite{Pedregosa2011Scikit} of the nearest-neighbor search (brute force search, kd tree, ball tree).
Inference time results on the \texttt{Blobs} dataset are reported in Figure~\ref{fig:nn:blobs} and accuracy results are displayed in Table~\ref{table:results_blobs}. 
% As shown in Figure~\ref{fig:nn:blobs:accuracy}, the accuracy of the approximate nearest neighbor search is above $0.99$ \todo{Accuracy $>0.99$ to be checked} for all the tested variants of \qkmeans, which is an solid evidence about the reliability of the approach.
The results for the Brute Force Search, KD Tree and Ball Tree are not available in Table~\ref{table:results_blobs} and in Figure~\ref{fig:nn:blobs} because they were longer than 10 times the \kmeans search version.
The running times reported in Figure~\ref{fig:nn:blobs} show a dramatic advantage of using a clustering-based approximate search 
%\todo{to be completed by reporting the actual acceleration ratio obtained by \qkmeans over the three sklearn options.} 
and this advantage is even stronger with the clustering obtained by our \qkmeans method. This speed-up comes at a cost though, we can see a drop in classification performance in Table~\ref{table:results_blobs}. 
% 
% \begin{figure}[tbh]
% \centering
% \includegraphics[width=.45\textwidth]{blobs_1nn_inference_time.png}
% \label{fig:nn:blobs:times}
% \caption{Running time of nearest neighbor search on blobs data. Results are averaged over 5 runs (vertical lines: standard deviation) and the average number of parameters actually learned is reported above each bar.}
% \label{fig:nn:blobs}
% \end{figure}


\begin{table*}[!htb]
% \centering
\resizebox{\textwidth}{!}{\begin{tabular}{@{}ccccc|ccc|ccc}
\toprule                                                                                                                                                                             
                                                &                               & \thead{\texttt{Blobs} \\ K=128}       & \thead{\texttt{Blobs} \\ K=256}   & \thead{\texttt{Blobs} \\ K=512}     & \thead{\texttt{Caltech} \\ K=128}       & \thead{\texttt{Caltech} \\ K=256}   & \thead{\texttt{Caltech} \\ K=512}     & \thead{\texttt{MNIST} \\ K=10}   & \thead{\texttt{MNIST} \\ K=16} & \thead{\texttt{MNIST} \\ K=30} \\ 
                                                
\midrule

                                                                                                                                                                                                                                                                             
\multirow{2}{*}{\shortstack{Vector assignation \\ time (ms)}} & \kmeans               & \boldsymbol{$0.05$}                   & \boldsymbol{$0.08$}                  & $0.12$                     & \boldsymbol{$remp$}                  & \boldsymbol{$remp$}                    & $remp$                       & \boldsymbol{$0.004$}           & \boldsymbol{$0.005$}   & \boldsymbol{$0.005$}   \\
                                                        & \qkmeans              & $0.06$                                & \boldsymbol{$0.08$}                 & \boldsymbol{$0.07$}                     & $remp$                               & $remp$                               & \boldsymbol{$remp$}            & $0.01$                       & $0.01$                 & $0.02$   \\

                                                        
\midrule \midrule                                                                                                                                                                                                                                                                                                
                                                        
                                                        

\multirow{3}{*}{\shortstack{Nyström \\ approximation \\ error}}   & \kmeans               & \boldsymbol{$0.032$}                  & \boldsymbol{$0.030$}                   & \boldsymbol{$0.027$}           & \boldsymbol{$remp$}                  & \boldsymbol{$remp$}                   & \boldsymbol{$remp$}           & \boldsymbol{$0.087$}          & \boldsymbol{$0.069$}        & \boldsymbol{$0.049$}   \\
                                                        & \qkmeans              & \boldsymbol{$0.032$}                  &  \underline{$0.031$}                  & \underline{$0.029$}             & \boldsymbol{$remp$}                  &  \underline{$remp$}                  & \underline{$remp$}             & \underline{$0.091$}           & \underline{$0.083$}         & \underline{$0.061$}   \\
                                                        & Uniform sampling      & $0.035$                               & $0.032$                               & $0.030$                         & $remp$                               & $remp$                               & $remp$                        & $0.184$              & $0.147$            & $0.105$   \\

                                                        
\midrule \midrule                                                                                                                                                                                                                                                                 


\multirow{2}{*}{\shortstack{Inference \\ time (ms)}}    & \kmeans               & \boldsymbol{$0.11$}                   & \boldsymbol{$0.17$}                  & $0.32$                            & \boldsymbol{$remp$}                  & \boldsymbol{$remp$}                    & $remp$                       & \boldsymbol{$0.06$}           & \boldsymbol{$0.06$}   & $0.07$   \\
                                                        & \qkmeans              & $0.13$                                & $0.18$                             & \boldsymbol{$0.22$}                 & $remp$                               &$remp$                               & \boldsymbol{$remp$}            & $0.08$                        & $0.07$                 & \boldsymbol{$0.06$}   \\

                                                        
\midrule \midrule                                                                                                                                                                                                                                                                                                


\multirow{5}{*}{\shortstack{1NN \\ Accuracy}}           & \kmeans               & \boldsymbol{$0.96$}                   & \boldsymbol{$0.97$}                & \boldsymbol{$0.99$}                 & \boldsymbol{$remp$}                  & \boldsymbol{$remp$}                    & $remp$                       & \boldsymbol{$0.96$}           & \boldsymbol{$0.96$}   & \boldsymbol{$0.96$}   \\
                                                        & \qkmeans              & $0.61$                                & $0.49$                             & $0.37$                              & $remp$                               & $remp$                               & \boldsymbol{$remp$}            & \boldsymbol{$0.96$}           & \boldsymbol{$0.96$}                 & $0.95$   \\
                                                        & Brute force          &                                           \multicolumn{3}{c|}{$N/A$}                                               & $remp$                               & $remp$                               & \boldsymbol{$remp$}           &                    \multicolumn{3}{c}{$0.97$}                                 \\
                                                        & Ball-tree            &                                           \multicolumn{3}{c|}{$N/A$}                                               & $remp$                               & $remp$                               & \boldsymbol{$remp$}           &                    \multicolumn{3}{c}{$0.97$}                                 \\
                                                        & Kd-tree              &                                           \multicolumn{3}{c|}{$N/A$}                                               & $remp$                               & $remp$                               & \boldsymbol{$remp$}           &                    \multicolumn{3}{c}{$0.97$}                                 \\

                                                        
\midrule \midrule                                                                                                                                                                                                                                                                                                
                                                                                                                                                                                                                                                                             
                                                                                                                                                                                                                                                                             
\multirow{5}{*}{\shortstack{1NN \\ Runtime (s)}} & \kmeans               & \boldsymbol{$17.2$}                   & \boldsymbol{$15.8$}                  & \boldsymbol{$9.5$}                     & \boldsymbol{$remp$}                  & \boldsymbol{$remp$}                    & $remp$                       & \boldsymbol{$0.74$}           & \boldsymbol{$0.83$}   & \boldsymbol{$0.88$}   \\
                                                        & \qkmeans              & $5.3$                                & \boldsymbol{$3.0$}                 & \boldsymbol{$1.2$}                     & $remp$                               & $remp$                               & \boldsymbol{$remp$}            & $0.73$                       & $0.79$                 & $0.86$   \\
                                                        & Brute force          &                                           \multicolumn{3}{c|}{$N/A$}                                               &  $remp$                               & $remp$                               & \boldsymbol{$remp$}           &                    \multicolumn{3}{c}{$2176.1$}                                 \\
                                                        & Ball-tree            &                                           \multicolumn{3}{c|}{$N/A$}                                               & $remp$                               & $remp$                               & \boldsymbol{$remp$}            &                    \multicolumn{3}{c}{$553.0$}                                 \\
                                                        & Kd-tree              &                                           \multicolumn{3}{c|}{$N/A$}                                               &$remp$                               & $remp$                               & \boldsymbol{$remp$}             &                    \multicolumn{3}{c}{$707.0$}                                 \\
                                                                                                                                                                                                  
                                                        
\midrule \midrule                                                                                                                                                                                                                                                                                                
                                                                                                                                                                                                                                                                             
                                                                                                                                                                                                                                                                             
\multirow{2}{*}{\shortstack{Nyström + SVM \\ Accuracy}} & \kmeans               & \boldsymbol{$0.98$}                   & \boldsymbol{$1.0$}                  & \boldsymbol{$1.0$}                     & \boldsymbol{$remp$}                  & \boldsymbol{$remp$}                    & $remp$                       & \boldsymbol{$0.74$}           & \boldsymbol{$0.83$}   & \boldsymbol{$0.88$}   \\
                                                        & \qkmeans              & $0.95$                                & \boldsymbol{$1.0$}                 & \boldsymbol{$1.0$}                     & $remp$                               & $remp$                               & \boldsymbol{$remp$}            & $0.73$                       & $0.79$                 & $0.86$   \\
                                                                                                                                                                                                  
                                                                                    
\bottomrule
\end{tabular}}
\caption{
Results of numerical experiments for landmark selection methods based on either \kmeans, \qkmeans or uniform sampling: Nyström approximation error and average transformation time for a sample set of size $5000$; 1-nearest neighbor classification accuracy on the test set; SVM classification on top of Nyström transformation on the test set. The \qkmeans results are obtained with sparse factors with at least 2 values in each line and each column. Every experiment results are averaged over 5 runs. Best results are bold while second best are underlined (when there are three). Cells marked as ``$N/A$'' refer to experiments that timed-out.}
\label{tab:nystrom}
\end{table*}

\todo[inline]{ajouter 1NN brute results a ce tableau}
\todo[inline]{verifier que l'on commente le fait que les résultats de vitesse avec mnist sont mauvais}

\subsection{Nyström approximation}

In this sub-section, we show how we can take advantage of the fast-operator obtained as output of our \qkmeans algorithm in order to speed-up the computation in the Nyström approximation. 
We start by giving background knowledge on the Nyström approximation then we present some recent work aiming at accelerating it using well know fast-transform method. 
We finally stem on this work to present a novel approach based on our \qkmeans algorithm.

\subsubsection{Background on the Nyström approximation}

Standard kernel machines are often impossible to use in large-scale applications because of their high computational cost associated with the kernel matrix $\rmK$ which has $O(n^2)$ storage and $O(n^2d)$ computational complexity: $\forall i,j \in\intint{\nexamples}, \rmK_{i,j} = k(\rvx_i, \rvx_j)$. A well-known strategy to overcome this problem is to use the Nyström method which computes a low-rank approximation of the kernel matrix on the basis of some pre-selected landmark points. 

Given $K \ll n$ landmark points $\{\rmU_i\}_{i=1}^{K}$, the Nyström method gives the following approximation of the full kernel matrix:
%
\begin{equation}
 \label{eq:nystrom}
 \rmK \approx \tilde\rmK = \rmC\rmW^\dagger\rmC^T,
\end{equation}
%
with $\rmW \in \R^{K \times K}$ containing all the kernel values between landmarks: $\forall i,j \in [\![K]\!]~ \rmW_{i,j} = k(\rmU_i, \rmU_j)$; $\rmW^\dagger$ being the pseudo-inverse of $\rmW$ and $\rmC \in \R^{n \times K}$ containing the kernel values between landmark points and all data points: $\forall i \in [\![n]\!], \forall j \in [\![K]\!]~ \rmC_{i, j} = k(\rmX_i, \rmU_j)$.

\subsubsection{Efficient Nyström approximation}

A substantial amount of research has been conducted toward landmark point selection methods for improved approximation accuracy \cite{kumar2012sampling} \cite{musco2017recursive}, but much less has been done to improve computation speed. In \cite{si2016computationally}, the authors propose an algorithm to learn the matrix of landmark points with some structure constraint, so that its utilisation is fast, taking advantage of fast-transforms. This results in an efficient Nyström approximation that is faster to use both in the training and testing phases of some ulterior machine learning application.

Remarking that the main computation cost of the Nyström approximation comes from the computation of the kernel function between the train/test samples and the landmark points, \cite{si2016computationally} aim at accelerating this step. In particular, they focus on a family of kernel functions that has the following form:
%
\begin{equation}
 k(\rvx_i, \rvx_j) = f(\rvx_i) f(\rvx_j) g(\rvx_i^T\rvx_j),
\end{equation}
%
where $f: \R^d \rightarrow \R$ and $g: \R \rightarrow \R$. They show that this family of functions contains some widely used kernels such as the Gaussian and the polynomial kernel. Given a set of $K$ landmark points $\rmU \in \R^{K \times d}$ and a sample $\rvx$, the computational time for computing the kernel between $\rvx$ and each row of $\rmU$ (necessary for the Nyström approximation) is bottlenecked by the computation of the product $\rmU\rvx$. They hence propose to write the $\rmU$ matrix as the concatenation of structured $s = K / d$ product of matrices:
%
\begin{equation}
 \rmU = \left[ \rmV_1 \rmH^T, \cdots, \rmV_s\rmH^T  \right]^T,
\end{equation}
%
where the $\rmH$ is a $d \times d$ matrix associated with a fast transform such as the \textit{Haar} or \textit{Hadamard} matrix, and the $\rmV_i$s are some $d \times d$ diagonal matrices to be either chosen with a standard landmark selection method or learned using an algorithm they provide.

Depending on the $\rmH$ matrix chosen, it is possible to improve the time complexity for the computation of $\rmU\rvx$ from $O(Kd)$ to $O(K \log{d})$ (\textit{Fast Hadamard transform}) or $O(K)$ (\textit{Fast Haar Transform}).

\subsubsection{\qkmeans in Nyström}

We propose to use our \qkmeans algorithm in order to learn directly the $\rmU$ matrix in the Nyström approximation so that the matrix-vector multiplication $\rmU \rvx$ is cheap to compute, but the structure of $\rmU$ is not constrained by some pre-defined transform matrix. We propose to take the objective $\rmU$ matrix as the \kmeans matrix of $\rmX$ since it has been shown to achieve good reconstruction accuracy in the Nyström method \cite{kumar2012sampling}.

As shown in the next sub-section, our algorithm allows to obtain an efficient Nyström approximation, while not reducing too much the quality of the \kmeans landmark points which are encoded as a factorization of sparse matrix. 

\subsubsection{Results}

The Table~\ref{tab:nystrom} summarizes the results achieved in the Nyström approximation setting. More results may be found in supplementary material \addLG{ajouter references au supplementary material. Comment faire? sachant que ce sera dans un document séparé}. 

The bottom part displays the average time for computing one line of the approximated matrix in Equation~\ref{eq:nystrom}. For $K=512$, when it is big enough, we clearly see the speed-up offered using the \qkmeans method on the \texttt{Blobs} dataset. On the \texttt{Mnist} and \texttt{Fashion-MNIST} dataset, this speed-up is sensible but not as clear because the standard deviation is much larger. 

The top part shows the approximation error of the Nyström approximation based on different sampling schemes w.r.t. the real kernel matrix. This error is computed by the Froebenius norm of the difference between the matrices and then normalized:

\begin{equation}
 error = \frac{||\rmK - \tilde\rmK||_F}{||\rmK||_F}.
\end{equation}

The \qkmeans approach gives better reconstruction error than the Nyström method based on uniform sampling although it is slightly worse than the one obtained with the regular \kmeans centers. We see that that the difference in approximation error between \kmeans and \qkmeans is almost negligible when compared to the approximation error obtained with the uniform sampling scheme.

From a more practical point of view, we show in Table~\ref{table:results_blobs} and Table~\ref{table:results_mnist_fmnist} that the Nyström approximation based on \qkmeans can then be used in a linear SVM and achieve as good performance as the one based on the \kmeans approach.






%{RBF networks}

%Besoin d'éclaircir les liens avec RBF networks

%\subsection{nearest-neighbours}

%Besoin d'éclaircir les liens avec nearest neighbours

\section{Conclusion}
\label{sec:conclusion}

In this paper, we have proposed a variant of the \kmeans algorithm, named \qkmeans, designed to achieve a similar goal -- clustering data points around $\nclusters$ learned centroids -- with a much lower computational complexity as the dimension of the data, the number of examples and the number of clusters get high. Our approach is based on the approximation of the centroid matrix by an operator structured as a product of a small number of sparse matrices, resulting in a low time and space complexity when applied to data vectors.
We have shown the convergence properties of the proposed algorithm and provided its complexity analysis.

An implementation prototype has been run in several core machine learning use cases including clustering, nearest-neighbor search and Nystr\"om approximation. The experimental results illustrate the computational gain in high dimension at inference time as well as the good approximation qualities of the proposed model.

Beyond these modeling, algorithmic and experimental contributions to low-complexity high-dimensional machine learning, we have identified several important questions that are still to be addressed.
First, although learning the fast-structure operator has been nicely integrated in the training algorithm with an advantageous theoretical time and space complexity, exhibiting gains in actual running times has not been achieved yet for the \qkmeans learning procedure, compared to \kmeans.
This may be obtained in even higher dimensions than in the proposed experimental settings, which may require a new version of \qkmeans using batches of data in order to process amounts of data that do not fit in memory.
Second, the expressiveness of the fast-structure model is still to be theoretically studied, while our experiments seems to show that arbitrary matrices may be well fitted by such models.
Third, we believe that learning fast-structure linear operators during the training procedure may be generalized to many core machine learning methods in order to speed them up and make them scale to larger dimensions.
\todo[inline]{Any other perspective?}

%a new algorithm with convergence proof, that allows to learn a matrix of \kmeans center-points with sparse factorization constraint. We provide complexity analysis showing, that this particular matrix may speed up further machine learning algorithms that would usually make use of the standard K-means center-point matrix. In particular, we show that this algorithm could be used in the context of the Nyström approximation.

%This paper does not contain any experimental results as it describes ongoing work. In the following, %next subsection (Section \ref{sec:foreseen_experiments}), 
%we discuss foreseen experiments that would aim at illustrating the speed gain when using our method, while not reducing the overall accuracy in machine learning settings.
%%
%%Last but not least, we discuss in the Section \ref{sec:discussion} 
%We also discuss a broader scope of application of our algorithm and some possible theoretical advantages.
%
%\subsection{Foreseen experiments}
%\label{sec:foreseen_experiments}
%
%Our algorithm will be evaluated on the same metrics than \cite{si2016computationally} on the Nyström approximation, namely the reconstruction error of the kernel matrix and the accuracy error in subsequent machine learning experiments. Those errors will be considered with respect to the computation speed. The considered baseline will be Nyström with simple \kmeans selected landmark points and, obviously, the efficient Nyström algorithm proposed in \cite{si2016computationally}.
%
%\subsection{Discussion}
%\label{sec:discussion}
%
%The algorithm we propose, \textit{Q-means}, could be applied to any other method that uses the \kmeans algorithm in its initialization. The \kmeans Nyström method is only one instance of such algorithm but we can already think of other examples, such as some nearest neighbour search algorithm based on the K-means clustering of the input space \cite{wang2011fast}.
%
%Finally, the sparse factorization constraint for the \kmeans center point matrix may play the role of a regularization, with parameter being the number of values in each factor. 
%We wish to investigate also this property more thoroughly from both theoretical and experimental point of view.
%This is just an intuition for now and this must be investigated more thoroughly from both the theoretical and experimental point of view.

%\begin{itemize}
% \item rbf networks
% \item hierarchical nearest neighbours
%\end{itemize}

%====================================================================================


\bibliographystyle{plain}
\bibliography{qmeans}

\appendix
\section{\palm algorithm}
\label{sec:app:palm4msa}

The \palm algorithm~\cite{LeMagoarou2016Flexible} is given in Algorithm~\ref{algo:palm4msa} together with the time complexity of each line, using $A=\min(\nclusters, \datadim)$ and $B=\max(\nclusters, \datadim)$. 
Even more general constraints can be used, the constraint sets $\mathcal{E}_q$ are typically defined as the intersection of the set of unit Frobenius-norm matrices and of a set of sparse matrices.
The unit Frobenius norm is used together with the $\lambda$ factor to avoid a scaling indeterminacy.
Note that to simplify the model presentation, factor $\lambda$ is used internally in \palm and is integrated in factor $\rmS_1$ at the end of the algorithm (Line~\ref{line:palm:postprocess:S1}) so that $\rmS_1$ does not satisfy the unit Frobenius norm in $\mathcal{E}_1$ at the end of the algorithm.
The sparse constraints we used, as in~\cite{LeMagoarou2016Flexible}, consist of trying to have a given number of non-zero coefficients in each row and in each column.
This number of non-zero coefficients is called sparsity level in this paper.
In practice, the projection function at Line~\ref{line:palm:update:S} keeps the largest non-zero coefficients in each row and in each column, which only guarantees
the actual number of non-zero coefficients is at least equal to the sparsity level.



%\addVE{Pour que $S_0=\lambda \rmI$ soit à gauche, inverser l'ordre du produit ($S_1, S_2, \ldots, S_{Q-1}$ contre $S_{Q-1}...S_2 S_1$) précédemment: done. The current order is from left to right for indices ($S_1, S_2, \ldots, S_{Q-1}$) while the update is from right to left ($q = \nfactors$ down to $1$). Is it ok?}

\begin{algorithm}
	\caption{\palm algorithm}
	\label{algo:palm4msa}
	\begin{algorithmic}[1]
		
		\REQUIRE The matrix to factorize $\rmU \in \R^{\nclusters \times \datadim}$, the desired number of factors $\nfactors$, the constraint sets $\mathcal{E}_q$ , $q\in \intint{\nfactors}$ and a stopping criterion (e.g., here, a number of iterations $I$ ).
		
%		\ENSURE $\{\rmS_1 \dots \rmS_{\nfactors}\}|\rmS_q \in \mathcal{E}_q$ such that $\prod_{q\in\intint{\nfactors}}\rmS_q \approx \rmU$
		\STATE $\lambda \leftarrow \norm{S_1}_F$
		\COMMENT{$\bigO{B}$}
		\label{line:palm:init:lambda}
		\STATE $S_1 \leftarrow \frac{1}{\lambda} S_1$
		\COMMENT{$\bigO{B}$}
		\label{line:palm:normalize:S1}
		\FOR {$i \in\intint{I}$ while the stopping criterion is not met}
		\FOR {$q = \nfactors$ down to $1$}
%		\FOR {$q = 2$ to $\nfactors$}
%		\STATE  $\rmL_q \leftarrow \prod_{l=q+1}^{\nfactors} \rmS_{l}^{(i)}$
		\STATE  $\rmL_q \leftarrow \prod_{l=1}^{q-1} \rmS_{l}^{(i)}$
%		\COMMENT{$\bigO{1}$}
		\label{line:palm:L}
%		\STATE  $\rmR_q \leftarrow \prod_{l=0}^{q-1} \rmS_{l}^{(i+1)}$
		\STATE  $\rmR_q \leftarrow \prod_{l=q+1}^{\nfactors} \rmS_{l}^{(i+1)}$
%		\COMMENT{$\bigO{1}$}
		\label{line:palm:R}
%		\STATE Choose $c > (\lambda^{(i)})^2 ||\rmR_q||_2^2 ||\rmL_q||_2^2$
		\STATE Choose $c > \lambda^2 ||\rmR_q||_2^2 ||\rmL_q||_2^2$
%		\COMMENT{in $\mathcal{O}\left (A\nfactors+B\right )$}
		\COMMENT{$\bigO{A \log A+B}$}
		\label{line:palm:c}
		\STATE $\rmD \leftarrow \rmS_q^i - \frac{1}{c} \lambda \rmL_q^T\left (\lambda\rmL_q \rmS_q^i \rmR_q - \rmU\right )\rmR_q^T$
%		\COMMENT{in $\mathcal{O}\left (\nclusters\datadim\nfactors\right )$}
		\COMMENT{$\bigO{AB\log A}$}
		\label{line:palm:D}
%		\STATE $\rmD \leftarrow \rmS_q^i - \frac{1}{c} \lambda^{(i)} \rmL_q^T(\lambda^{(i)} \rmL_q \rmS_q^i \rmR_q - \rmU)\rmR_q^T$
		\STATE $\rmS^{(i+1)}_q \leftarrow P_{\mathcal{E}_q}(\rmD)$
%		\COMMENT{in $\mathcal{O}\left (\nclusters\datadim\nfactors\right )$}
		\COMMENT{$\bigO{A^2\log A}$ or $\bigO{AB\log B}$}
		\label{line:palm:update:S}
		\ENDFOR
		\STATE $\hat \rmU \eqdef \prod_{j=1}^{\nfactors} \rmS_q^{(i+1)}$
		\COMMENT{$\bigO{A^2\log A + AB}$}
		\label{line:palm:U}
		\STATE $\lambda \leftarrow \frac{Trace(\rmU^T\hat\rmU)}{Trace(\hat\rmU^T\hat\rmU)}$% \rmI$
%		\STATE $\rmS_1 \leftarrow \frac{Trace(\rmU^T\hat\rmU)}{Trace(\hat\rmU^T\hat\rmU)} \rmI$
%		\COMMENT{in $\mathcal{O}\left (\nclusters\datadim\right )$}
		\COMMENT{$\bigO{AB}$}
		\label{line:palm:update:lambda}
		\ENDFOR
		\STATE $S_1 \leftarrow \lambda S_1$
		\COMMENT{$\bigO{B}$}
		\label{line:palm:postprocess:S1}
		\ENSURE $\left \lbrace \rmS_q : \rmS_q \in \mathcal{E}_q\right \rbrace_{q\in\intint{\nfactors}}$ such that $\prod_{q\in\intint{\nfactors}}\rmS_q \approx \rmU$
		
	\end{algorithmic}
\end{algorithm}

The complexity analysis is proposed under the following assumptions, which are satisfied in the mentioned applications and experiments: the number of factors is $\nfactors=\mathcal{O}\left (\log A\right )$; all but one sparse factors are of shape $A \times A$ and have $\bigO{A}$ non-zero entries while one of them is of shape $A\times B$ or $B\times A$ with $\bigO{B}$ non-zero entries.
In such conditions, the complexity of each line is:
\begin{itemize}
 \item [Lines~\ref{line:palm:init:lambda}-\ref{line:palm:normalize:S1}] Computing these normalization steps is linear in the number of non-zeros coefficients in $\rmS_1$.
 \item [Lines~\ref{line:palm:L}-\ref{line:palm:R}] Fast operators $\rmL$ and $\rmR$ are defined for subsequent use without computing explicitly the product.
% s can be \textit{precomputed} incrementaly for each iteration $i$, involving a total cost of $\mathcal{O}(Qpq)$ operations: for all $j < Q$, $\rmL_j = \rmL_{j+1} \mathcal{S}^i_{j+1}$; for $j = Q$, $\rmL_j = \textbf{Id}$;
% \item [Line 4] The $\rmR$s is computed incrementaly for each iteration $j$: $\rmR_j = \mathcal{S}^{i+1}_{j-1} \rmR_{j-1}$ if $j > 1$; $\rmR_j = \textbf{Id}$ otherwise. This costs an overall $\mathcal{O}(Qpq)$ operations;
 \item [Line~\ref{line:palm:c}] The spectral norm of $\rmL$ and $\rmR$ is obtained via a power method by iteratively applying each operator, benefiting from the fast transform.
 \item [Line~\ref{line:palm:D}] The cost of the gradient step is dominated by the product of sparse matrices.
% \addLG{avec Valentin on avait trouvé O(Kd min \{K, d\}) mais je ne suis plus d'accord} Taking advantage of the decompositions of $\rmL$ and $\rmR$ as products of sparse factors, the time complexity of this line ends up being $\mathcal{O}(Q(Qpq + Kd + q^2\log q^2))$ for a complete iteration: the $\mathcal{O}(Qpq)$ part comes from the various sparse-dense matrix multiplications with $\rmR$ and $\rmL$; the $\mathcal{O}(Kd)$ part comes from the pairwise substraction inside the parentheses and the $\mathcal{O}(q^2 \log q^2)$ part from the projection operator that involves sorting of the inner matrix.\addLG{$q^2 \log q^2$ peut être remplacé par $q^2 + p\log p$ grâce à l'algo quickselect}
\item [Line~\ref{line:palm:update:S}] The projection onto a sparse-constraint set takes $\bigO{A^2\log A}$ for all the $A\times A$ matrices and $\bigO{AB\log B}$ for the rectangular matrix at the leftmost or the rightmost position.
 \item [Line~\ref{line:palm:U}] The reconstructed matrix $\hat \rmU$ is computed using $\bigO{\log A}$ products between $A\times A$ sparse matrices, in $\bigO{A^2}$ operations each, and one product with a sparse matrix in $\bigO{AB}$.
 \item [Line~\ref{line:palm:update:lambda}] The numerator and denominator can be computed using a Hadamard product between the matrices followed by a sum over all the entries.
  \item [Line~\ref{line:palm:postprocess:S1}] Computing renormalization step is linear in the number of non-zeros coefficients in $\rmS_1$.
\end{itemize}

Hence, the overal time complexity of \palm is in $\bigO{AB\left (\log^2 A+\log B\right )}$, due to Lines~\ref{line:palm:D} and~\ref{line:palm:update:S}.

%\todo[inline]{Valentin: I have reintroduced $\lambda$, but just as an internal variable. Is it ok? Check also the compliance of this formulation with the rest of the paper, e.g., with the definition of $\mathcal{E}_q$. Factors are updated from right to left, exact?}
%\todo[inline]{Define $\mathcal{E}_q$ with the sparsity constraint and the Frobenius-norm constraint.}
%\todo[inline]{add details about the hierarchical algorithm? \addVE{Non, pas besoin?}}
%\todo[inline]{The stopping criterion is the number of iterations $I$ here, which is incoherent with section~\ref{sec:uses:settings}.}


\end{document}
