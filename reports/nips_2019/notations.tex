%\paragraph{Notations}

The notations frequently used in the paper are summarized in Table~\ref{tab:notation}. 
%
Throughout the paper we use $\nexamples$ as the number of data samples and $\datadim$ the dimensionality of a data point. 
$\rmX \in \R^{\nexamples \times \datadim}$ is the data matrix. 
For $K \in \sN$, we define $\intint{K}=\left \lbrace k\in \sN: 1 \leq k \leq K\right \rbrace$.
%
For a given vector $\rvv$, $\rvv[i]$ is the $i$th component of $\rvv$.
%
For a given matrix $\rmM$, the notation $\rmM_{[i]}$ (resp. $\rmM^{[i]}$) refers to the $i$th row (column) of $\rmM$, the entry at the $i$th row and the $j$th column is denoted by $\rmM[i,j]$, and $\|\rmM\|_F$ denotes the Frobenius norm, $\|\rmM\|_2$ the spectral norm and $\|\rmM\|_0$ counts the number of non-zero entries in $\rmM$. \addHK{other norms?}




