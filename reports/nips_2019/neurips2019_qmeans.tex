\documentclass{article}

% if you need to pass options to natbib, use, e.g.:
%     \PassOptionsToPackage{numbers, compress}{natbib}
% before loading neurips_2019

% ready for submission
 \usepackage{neurips_2019}

% to compile a preprint version, e.g., for submission to arXiv, add add the
% [preprint] option:
%     \usepackage[preprint]{neurips_2019}

% to compile a camera-ready version, add the [final] option, e.g.:
%   \usepackage[final]{neurips_2019}

% to avoid loading the natbib package, add option nonatbib:
%     \usepackage[nonatbib]{neurips_2019}

\usepackage[utf8]{inputenc} % allow utf-8 input
\usepackage[T1]{fontenc}    % use 8-bit T1 fonts
\usepackage{hyperref}       % hyperlinks
\usepackage{url}            % simple URL typesetting
\usepackage{booktabs}       % professional-quality tables
\usepackage{amsfonts}       % blackboard math symbols
\usepackage{nicefrac}       % compact symbols for 1/2, etc.
\usepackage{microtype}      % microtypography

\title{Q-means :\\ acceleration of K-means by learning a fast transform}

% The \author macro works with any number of authors. There are two commands
% used to separate the names and addresses of multiple authors: \And and \AND.
%
% Using \And between authors leaves it to LaTeX to determine where to break the
% lines. Using \AND forces a line break at that point. So, if LaTeX puts 3 of 4
% authors names on the first line, and the last on the second line, try using
% \AND instead of \And before the third author name.

\author{%
  Luc Giffon\thanks{to be completed} \\
  Affiliation\\
  Address\\
  \texttt{email} \\
  % examples of more authors
   \And
   Valentin Emiya\\
  Affiliation\\
  Address\\
  \texttt{email} \\
   \And
   Liva Ralaivola\\
  Affiliation\\
  Address\\
  \texttt{email} \\
   \And
   Hachem Kadri\\
  Affiliation\\
  Address\\
  \texttt{email} \\
  % Coauthor \\
  % Affiliation \\
  % Address \\
  % \texttt{email} \\
  % \AND
  % Coauthor \\
  % Affiliation \\
  % Address \\
  % \texttt{email} \\
  % \And
  % Coauthor \\
  % Affiliation \\
  % Address \\
  % \texttt{email} \\
  % \And
  % Coauthor \\
  % Affiliation \\
  % Address \\
  % \texttt{email} \\
}

% OUR ADDITIONS IN PREAMBLE
%====================================================================================
%%%%% NEW MATH DEFINITIONS %%%%%

\usepackage{amsmath,amsfonts,bm}

% Mark sections of captions for referring to divisions of figures
\newcommand{\figleft}{{\em (Left)}}
\newcommand{\figcenter}{{\em (Center)}}
\newcommand{\figright}{{\em (Right)}}
\newcommand{\figtop}{{\em (Top)}}
\newcommand{\figbottom}{{\em (Bottom)}}
\newcommand{\captiona}{{\em (a)}}
\newcommand{\captionb}{{\em (b)}}
\newcommand{\captionc}{{\em (c)}}
\newcommand{\captiond}{{\em (d)}}

% Highlight a newly defined term
\newcommand{\newterm}[1]{{\bf #1}}


% Figure reference, lower-case.
\def\figref#1{figure~\ref{#1}}
% Figure reference, capital. For start of sentence
\def\Figref#1{Figure~\ref{#1}}
\def\twofigref#1#2{figures \ref{#1} and \ref{#2}}
\def\quadfigref#1#2#3#4{figures \ref{#1}, \ref{#2}, \ref{#3} and \ref{#4}}
% Section reference, lower-case.
\def\secref#1{section~\ref{#1}}
% Section reference, capital.
\def\Secref#1{Section~\ref{#1}}
% Reference to two sections.
\def\twosecrefs#1#2{sections \ref{#1} and \ref{#2}}
% Reference to three sections.
\def\secrefs#1#2#3{sections \ref{#1}, \ref{#2} and \ref{#3}}
% Reference to an equation, lower-case.
\def\eqref#1{equation~\ref{#1}}
% Reference to an equation, upper case
\def\Eqref#1{Equation~\ref{#1}}
% A raw reference to an equation---avoid using if possible
\def\plaineqref#1{\ref{#1}}
% Reference to a chapter, lower-case.
\def\chapref#1{chapter~\ref{#1}}
% Reference to an equation, upper case.
\def\Chapref#1{Chapter~\ref{#1}}
% Reference to a range of chapters
\def\rangechapref#1#2{chapters\ref{#1}--\ref{#2}}
% Reference to an algorithm, lower-case.
\def\algref#1{algorithm~\ref{#1}}
% Reference to an algorithm, upper case.
\def\Algref#1{Algorithm~\ref{#1}}
\def\twoalgref#1#2{algorithms \ref{#1} and \ref{#2}}
\def\Twoalgref#1#2{Algorithms \ref{#1} and \ref{#2}}
% Reference to a part, lower case
\def\partref#1{part~\ref{#1}}
% Reference to a part, upper case
\def\Partref#1{Part~\ref{#1}}
\def\twopartref#1#2{parts \ref{#1} and \ref{#2}}

\def\ceil#1{\lceil #1 \rceil}
\def\floor#1{\lfloor #1 \rfloor}
\def\1{\bm{1}}
\newcommand{\train}{\mathcal{D}}
\newcommand{\valid}{\mathcal{D_{\mathrm{valid}}}}
\newcommand{\test}{\mathcal{D_{\mathrm{test}}}}

\def\eps{{\epsilon}}


% Random variables
\def\reta{{\textnormal{$\eta$}}}
\def\ra{{\textnormal{a}}}
\def\rb{{\textnormal{b}}}
\def\rc{{\textnormal{c}}}
\def\rd{{\textnormal{d}}}
\def\re{{\textnormal{e}}}
\def\rf{{\textnormal{f}}}
\def\rg{{\textnormal{g}}}
\def\rh{{\textnormal{h}}}
\def\ri{{\textnormal{i}}}
\def\rj{{\textnormal{j}}}
\def\rk{{\textnormal{k}}}
\def\rl{{\textnormal{l}}}
% rm is already a command, just don't name any random variables m
\def\rn{{\textnormal{n}}}
\def\ro{{\textnormal{o}}}
\def\rp{{\textnormal{p}}}
\def\rq{{\textnormal{q}}}
\def\rr{{\textnormal{r}}}
\def\rs{{\textnormal{s}}}
\def\rt{{\textnormal{t}}}
\def\ru{{\textnormal{u}}}
\def\rv{{\textnormal{v}}}
\def\rw{{\textnormal{w}}}
\def\rx{{\textnormal{x}}}
\def\ry{{\textnormal{y}}}
\def\rz{{\textnormal{z}}}

% Random vectors
\def\rvepsilon{{\mathbf{\epsilon}}}
\def\rvtheta{{\mathbf{\theta}}}
\def\rva{{\mathbf{a}}}
\def\rvb{{\mathbf{b}}}
\def\rvc{{\mathbf{c}}}
\def\rvd{{\mathbf{d}}}
\def\rve{{\mathbf{e}}}
\def\rvf{{\mathbf{f}}}
\def\rvg{{\mathbf{g}}}
\def\rvh{{\mathbf{h}}}
\def\rvu{{\mathbf{i}}}
\def\rvj{{\mathbf{j}}}
\def\rvk{{\mathbf{k}}}
\def\rvl{{\mathbf{l}}}
\def\rvm{{\mathbf{m}}}
\def\rvn{{\mathbf{n}}}
\def\rvo{{\mathbf{o}}}
\def\rvp{{\mathbf{p}}}
\def\rvq{{\mathbf{q}}}
\def\rvr{{\mathbf{r}}}
\def\rvs{{\mathbf{s}}}
\def\rvt{{\mathbf{t}}}
\def\rvu{{\mathbf{u}}}
\def\rvv{{\mathbf{v}}}
\def\rvw{{\mathbf{w}}}
\def\rvx{{\mathbf{x}}}
\def\rvy{{\mathbf{y}}}
\def\rvz{{\mathbf{z}}}

% Elements of random vectors
\def\erva{{\textnormal{a}}}
\def\ervb{{\textnormal{b}}}
\def\ervc{{\textnormal{c}}}
\def\ervd{{\textnormal{d}}}
\def\erve{{\textnormal{e}}}
\def\ervf{{\textnormal{f}}}
\def\ervg{{\textnormal{g}}}
\def\ervh{{\textnormal{h}}}
\def\ervi{{\textnormal{i}}}
\def\ervj{{\textnormal{j}}}
\def\ervk{{\textnormal{k}}}
\def\ervl{{\textnormal{l}}}
\def\ervm{{\textnormal{m}}}
\def\ervn{{\textnormal{n}}}
\def\ervo{{\textnormal{o}}}
\def\ervp{{\textnormal{p}}}
\def\ervq{{\textnormal{q}}}
\def\ervr{{\textnormal{r}}}
\def\ervs{{\textnormal{s}}}
\def\ervt{{\textnormal{t}}}
\def\ervu{{\textnormal{u}}}
\def\ervv{{\textnormal{v}}}
\def\ervw{{\textnormal{w}}}
\def\ervx{{\textnormal{x}}}
\def\ervy{{\textnormal{y}}}
\def\ervz{{\textnormal{z}}}

% Random matrices
\def\rmA{{\mathbf{A}}}
\def\rmB{{\mathbf{B}}}
\def\rmC{{\mathbf{C}}}
\def\rmD{{\mathbf{D}}}
\def\rmE{{\mathbf{E}}}
\def\rmF{{\mathbf{F}}}
\def\rmG{{\mathbf{G}}}
\def\rmH{{\mathbf{H}}}
\def\rmI{{\mathbf{I}}}
\def\rmJ{{\mathbf{J}}}
\def\rmK{{\mathbf{K}}}
\def\rmL{{\mathbf{L}}}
\def\rmM{{\mathbf{M}}}
\def\rmN{{\mathbf{N}}}
\def\rmO{{\mathbf{O}}}
\def\rmP{{\mathbf{P}}}
\def\rmQ{{\mathbf{Q}}}
\def\rmR{{\mathbf{R}}}
\def\rmS{{\mathbf{S}}}
\def\rmT{{\mathbf{T}}}
\def\rmU{{\mathbf{U}}}
\def\rmV{{\mathbf{V}}}
\def\rmW{{\mathbf{W}}}
\def\rmX{{\mathbf{X}}}
\def\rmY{{\mathbf{Y}}}
\def\rmZ{{\mathbf{Z}}}

% Elements of random matrices
\def\ermA{{\textnormal{A}}}
\def\ermB{{\textnormal{B}}}
\def\ermC{{\textnormal{C}}}
\def\ermD{{\textnormal{D}}}
\def\ermE{{\textnormal{E}}}
\def\ermF{{\textnormal{F}}}
\def\ermG{{\textnormal{G}}}
\def\ermH{{\textnormal{H}}}
\def\ermI{{\textnormal{I}}}
\def\ermJ{{\textnormal{J}}}
\def\ermK{{\textnormal{K}}}
\def\ermL{{\textnormal{L}}}
\def\ermM{{\textnormal{M}}}
\def\ermN{{\textnormal{N}}}
\def\ermO{{\textnormal{O}}}
\def\ermP{{\textnormal{P}}}
\def\ermQ{{\textnormal{Q}}}
\def\ermR{{\textnormal{R}}}
\def\ermS{{\textnormal{S}}}
\def\ermT{{\textnormal{T}}}
\def\ermU{{\textnormal{U}}}
\def\ermV{{\textnormal{V}}}
\def\ermW{{\textnormal{W}}}
\def\ermX{{\textnormal{X}}}
\def\ermY{{\textnormal{Y}}}
\def\ermZ{{\textnormal{Z}}}

% Vectors
\def\vzero{{\bm{0}}}
\def\vone{{\bm{1}}}
\def\vmu{{\bm{\mu}}}
\def\vtheta{{\bm{\theta}}}
\def\va{{\bm{a}}}
\def\vb{{\bm{b}}}
\def\vc{{\bm{c}}}
\def\vd{{\bm{d}}}
\def\ve{{\bm{e}}}
\def\vf{{\bm{f}}}
\def\vg{{\bm{g}}}
\def\vh{{\bm{h}}}
\def\vi{{\bm{i}}}
\def\vj{{\bm{j}}}
\def\vk{{\bm{k}}}
\def\vl{{\bm{l}}}
\def\vm{{\bm{m}}}
\def\vn{{\bm{n}}}
\def\vo{{\bm{o}}}
\def\vp{{\bm{p}}}
\def\vq{{\bm{q}}}
\def\vr{{\bm{r}}}
\def\vs{{\bm{s}}}
\def\vt{{\bm{t}}}
\def\vu{{\bm{u}}}
\def\vv{{\bm{v}}}
\def\vw{{\bm{w}}}
\def\vx{{\bm{x}}}
\def\vy{{\bm{y}}}
\def\vz{{\bm{z}}}

% Elements of vectors
\def\evalpha{{\alpha}}
\def\evbeta{{\beta}}
\def\evepsilon{{\epsilon}}
\def\evlambda{{\lambda}}
\def\evomega{{\omega}}
\def\evmu{{\mu}}
\def\evpsi{{\psi}}
\def\evsigma{{\sigma}}
\def\evtheta{{\theta}}
\def\eva{{a}}
\def\evb{{b}}
\def\evc{{c}}
\def\evd{{d}}
\def\eve{{e}}
\def\evf{{f}}
\def\evg{{g}}
\def\evh{{h}}
\def\evi{{i}}
\def\evj{{j}}
\def\evk{{k}}
\def\evl{{l}}
\def\evm{{m}}
\def\evn{{n}}
\def\evo{{o}}
\def\evp{{p}}
\def\evq{{q}}
\def\evr{{r}}
\def\evs{{s}}
\def\evt{{t}}
\def\evu{{u}}
\def\evv{{v}}
\def\evw{{w}}
\def\evx{{x}}
\def\evy{{y}}
\def\evz{{z}}

% Matrix
\def\mA{{\bm{A}}}
\def\mB{{\bm{B}}}
\def\mC{{\bm{C}}}
\def\mD{{\bm{D}}}
\def\mE{{\bm{E}}}
\def\mF{{\bm{F}}}
\def\mG{{\bm{G}}}
\def\mH{{\bm{H}}}
\def\mI{{\bm{I}}}
\def\mJ{{\bm{J}}}
\def\mK{{\bm{K}}}
\def\mL{{\bm{L}}}
\def\mM{{\bm{M}}}
\def\mN{{\bm{N}}}
\def\mO{{\bm{O}}}
\def\mP{{\bm{P}}}
\def\mQ{{\bm{Q}}}
\def\mR{{\bm{R}}}
\def\mS{{\bm{S}}}
\def\mT{{\bm{T}}}
\def\mU{{\bm{U}}}
\def\mV{{\bm{V}}}
\def\mW{{\bm{W}}}
\def\mX{{\bm{X}}}
\def\mY{{\bm{Y}}}
\def\mZ{{\bm{Z}}}
\def\mBeta{{\bm{\beta}}}
\def\mPhi{{\bm{\Phi}}}
\def\mLambda{{\bm{\Lambda}}}
\def\mSigma{{\bm{\Sigma}}}

% Tensor
\DeclareMathAlphabet{\mathsfit}{\encodingdefault}{\sfdefault}{m}{sl}
\SetMathAlphabet{\mathsfit}{bold}{\encodingdefault}{\sfdefault}{bx}{n}
\newcommand{\tens}[1]{\bm{\mathsfit{#1}}}
\def\tA{{\tens{A}}}
\def\tB{{\tens{B}}}
\def\tC{{\tens{C}}}
\def\tD{{\tens{D}}}
\def\tE{{\tens{E}}}
\def\tF{{\tens{F}}}
\def\tG{{\tens{G}}}
\def\tH{{\tens{H}}}
\def\tI{{\tens{I}}}
\def\tJ{{\tens{J}}}
\def\tK{{\tens{K}}}
\def\tL{{\tens{L}}}
\def\tM{{\tens{M}}}
\def\tN{{\tens{N}}}
\def\tO{{\tens{O}}}
\def\tP{{\tens{P}}}
\def\tQ{{\tens{Q}}}
\def\tR{{\tens{R}}}
\def\tS{{\tens{S}}}
\def\tT{{\tens{T}}}
\def\tU{{\tens{U}}}
\def\tV{{\tens{V}}}
\def\tW{{\tens{W}}}
\def\tX{{\tens{X}}}
\def\tY{{\tens{Y}}}
\def\tZ{{\tens{Z}}}


% Graph
\def\gA{{\mathcal{A}}}
\def\gB{{\mathcal{B}}}
\def\gC{{\mathcal{C}}}
\def\gD{{\mathcal{D}}}
\def\gE{{\mathcal{E}}}
\def\gF{{\mathcal{F}}}
\def\gG{{\mathcal{G}}}
\def\gH{{\mathcal{H}}}
\def\gI{{\mathcal{I}}}
\def\gJ{{\mathcal{J}}}
\def\gK{{\mathcal{K}}}
\def\gL{{\mathcal{L}}}
\def\gM{{\mathcal{M}}}
\def\gN{{\mathcal{N}}}
\def\gO{{\mathcal{O}}}
\def\gP{{\mathcal{P}}}
\def\gQ{{\mathcal{Q}}}
\def\gR{{\mathcal{R}}}
\def\gS{{\mathcal{S}}}
\def\gT{{\mathcal{T}}}
\def\gU{{\mathcal{U}}}
\def\gV{{\mathcal{V}}}
\def\gW{{\mathcal{W}}}
\def\gX{{\mathcal{X}}}
\def\gY{{\mathcal{Y}}}
\def\gZ{{\mathcal{Z}}}

% Sets
\def\sA{{\mathbb{A}}}
\def\sB{{\mathbb{B}}}
\def\sC{{\mathbb{C}}}
\def\sD{{\mathbb{D}}}
% Don't use a set called E, because this would be the same as our symbol
% for expectation.
\def\sF{{\mathbb{F}}}
\def\sG{{\mathbb{G}}}
\def\sH{{\mathbb{H}}}
\def\sI{{\mathbb{I}}}
\def\sJ{{\mathbb{J}}}
\def\sK{{\mathbb{K}}}
\def\sL{{\mathbb{L}}}
\def\sM{{\mathbb{M}}}
\def\sN{{\mathbb{N}}}
\def\sO{{\mathbb{O}}}
\def\sP{{\mathbb{P}}}
\def\sQ{{\mathbb{Q}}}
\def\sR{{\mathbb{R}}}
\def\sS{{\mathbb{S}}}
\def\sT{{\mathbb{T}}}
\def\sU{{\mathbb{U}}}
\def\sV{{\mathbb{V}}}
\def\sW{{\mathbb{W}}}
\def\sX{{\mathbb{X}}}
\def\sY{{\mathbb{Y}}}
\def\sZ{{\mathbb{Z}}}

% Entries of a matrix
\def\emLambda{{\Lambda}}
\def\emA{{A}}
\def\emB{{B}}
\def\emC{{C}}
\def\emD{{D}}
\def\emE{{E}}
\def\emF{{F}}
\def\emG{{G}}
\def\emH{{H}}
\def\emI{{I}}
\def\emJ{{J}}
\def\emK{{K}}
\def\emL{{L}}
\def\emM{{M}}
\def\emN{{N}}
\def\emO{{O}}
\def\emP{{P}}
\def\emQ{{Q}}
\def\emR{{R}}
\def\emS{{S}}
\def\emT{{T}}
\def\emU{{U}}
\def\emV{{V}}
\def\emW{{W}}
\def\emX{{X}}
\def\emY{{Y}}
\def\emZ{{Z}}
\def\emSigma{{\Sigma}}

% entries of a tensor
% Same font as tensor, without \bm wrapper
\newcommand{\etens}[1]{\mathsfit{#1}}
\def\etLambda{{\etens{\Lambda}}}
\def\etA{{\etens{A}}}
\def\etB{{\etens{B}}}
\def\etC{{\etens{C}}}
\def\etD{{\etens{D}}}
\def\etE{{\etens{E}}}
\def\etF{{\etens{F}}}
\def\etG{{\etens{G}}}
\def\etH{{\etens{H}}}
\def\etI{{\etens{I}}}
\def\etJ{{\etens{J}}}
\def\etK{{\etens{K}}}
\def\etL{{\etens{L}}}
\def\etM{{\etens{M}}}
\def\etN{{\etens{N}}}
\def\etO{{\etens{O}}}
\def\etP{{\etens{P}}}
\def\etQ{{\etens{Q}}}
\def\etR{{\etens{R}}}
\def\etS{{\etens{S}}}
\def\etT{{\etens{T}}}
\def\etU{{\etens{U}}}
\def\etV{{\etens{V}}}
\def\etW{{\etens{W}}}
\def\etX{{\etens{X}}}
\def\etY{{\etens{Y}}}
\def\etZ{{\etens{Z}}}

% The true underlying data generating distribution
\newcommand{\pdata}{p_{\rm{data}}}
% The empirical distribution defined by the training set
\newcommand{\ptrain}{\hat{p}_{\rm{data}}}
\newcommand{\Ptrain}{\hat{P}_{\rm{data}}}
% The model distribution
\newcommand{\pmodel}{p_{\rm{model}}}
\newcommand{\Pmodel}{P_{\rm{model}}}
\newcommand{\ptildemodel}{\tilde{p}_{\rm{model}}}
% Stochastic autoencoder distributions
\newcommand{\pencode}{p_{\rm{encoder}}}
\newcommand{\pdecode}{p_{\rm{decoder}}}
\newcommand{\precons}{p_{\rm{reconstruct}}}

\newcommand{\laplace}{\mathrm{Laplace}} % Laplace distribution

\newcommand{\E}{\mathbb{E}}
\newcommand{\Ls}{\mathcal{L}}
\newcommand{\R}{\mathbb{R}}
\newcommand{\emp}{\tilde{p}}
\newcommand{\lr}{\alpha}
\newcommand{\reg}{\lambda}
\newcommand{\rect}{\mathrm{rectifier}}
\newcommand{\softmax}{\mathrm{softmax}}
\newcommand{\sigmoid}{\sigma}
\newcommand{\softplus}{\zeta}
\newcommand{\KL}{D_{\mathrm{KL}}}
\newcommand{\Var}{\mathrm{Var}}
\newcommand{\standarderror}{\mathrm{SE}}
\newcommand{\Cov}{\mathrm{Cov}}
% Wolfram Mathworld says $L^2$ is for function spaces and $\ell^2$ is for vectors
% But then they seem to use $L^2$ for vectors throughout the site, and so does
% wikipedia.
\newcommand{\normlzero}{L^0}
\newcommand{\normlone}{L^1}
\newcommand{\normltwo}{L^2}
\newcommand{\normlp}{L^p}
\newcommand{\normmax}{L^\infty}

\newcommand{\parents}{Pa} % See usage in notation.tex. Chosen to match Daphne's book.

\DeclareMathOperator*{\argmax}{arg\,max}
\DeclareMathOperator*{\argmin}{arg\,min}

\DeclareMathOperator{\sign}{sign}
\DeclareMathOperator{\Tr}{Tr}
\let\ab\allowbreak


\usepackage{algorithm}
\usepackage{algorithmic}
\usepackage{varwidth}
\usepackage{graphicx}
\usepackage{color}
\usepackage{enumitem}

\def\rmGamma{{\mathbf{\Gamma}}}

\newcommand{\addLG}[1]{{\color{green}  #1}}
\newcommand{\addVE}[1]{{\color{blue}  #1}}

%====================================================================================

\begin{document}

\maketitle

\begin{abstract}
	
K-means -- and the celebrated Lloyd algorithm -- is more than the clustering method it was 
originally designed to be. It has indeed proven pivotal to help increase the speed of
many machine learning and data analysis techniques such as indexing, nearest-neighbor 
search and prediction, data compression, Radial Basis Function neetworks;
 its beneficial use has been shown to carry over to the acceleration of kernel machines 
 (when using the Nyström method). Here, we propose a fast extension of K-means, dubbed Q-means, 
 that rests on
the idea of expressing the matrix of the $K$ centroids
as a product of sparse matrix, a feat made possible by recent results devoted 
to find approximations of matrices as a product of sparse factors. Using such 
decomposition squashes the complexity of the matrix-vector product between
the factorized $K \times D$ centroid matrix $\mathbf{U}$ and any vector from
$\mathcal{O}(KD)$ to $\mathcal{O}(P+Q \log Q)$, with $Q=\min (K, D)$ and $P=\max (K, D)$,
where $D$ is the dimension of the training data. This drastic computational saving
has a direct impact in the assignment process of a point to a cluster, 
meaning that it is not only tangible at prediction time, but also at training time,
provided the factorization procedure is performed during Lloyd's algorithm.
We precisely show that resorting to a factorization step at each iteration does not 
impair the convergence of the optimization scheme and that, depending on the context, 
it may entail a reduction of the training time. Finally, we provide discussions and numerical
 simulations that show the versatility of our computationally-efficient Q-means algorithm.




Beyond its popularity for clustering, the K-means algorithm is a pivotal procedure for other core machine learning and data analysis techniques such as indexing, nearest-neightbors prediction, as well as for more specific approaches like the Nyström approximation for kernel machines.

In this paper, we propose the Q-means algorithm, an accelerated version of $K$-means that stems from recent advances in optimization to learn the centroid matrix as a product of sparse matrices.
This decomposition provides a structure similar to that of fast transforms (e.g., Fourier, Hadamard) in order to benefit from its computationnal efficiency while being adapted to the training data.
Indeed, the complexity of the matrix-vector product between the factorized $K \times D$ matrix $\mathbf{U}$ and any vector is lowered from $\mathcal{O}(KD)$ to $\mathcal{O}(P+Q \log Q)$, with $Q=\min (K, D)$ and $P=\max (K, D)$.
This dramatic acceleration is beneficial whenever a point is assigned to a cluster, i.e., at prediction time and in the assignation step at learning time.
In addition, we show that the computational overhead due to the decomposition procedure does not penalize the computational cost of the learning stage, 
which may be faster than the traditionnal Lloyd algorithm depending on the context.

Finally, we provide discussions and numerical experiments that show the versatility of the proposed computationally-efficient Q-means algorithm.

%\addVE{Remarque: on ne mentionne pas la qualité de l'approximation qu'on obtient en remplaçant K-means par Q-means?!} \addLG{je crois qu'on peut assez peu s'exprimer à ce sujet sans borne...}
%
%\addVE{Remarque sur la complexité $\mathcal{O}(p+q \log q)$: on a $\log q$ facteurs dont un de taille $p\times q$ (ou l'inverse) a $\mathcal{O}(p)$ valeurs non-nulles et tous les autres de taille $q \times q$ ont $\mathcal{O}(q)$ valeurs non-nulles.}
\end{abstract}

\addVE{Remarque: on ne mentionne pas la qualité de l'approximation qu'on obtient en remplaçant K-means par Q-means?!} \addLG{je crois qu'on peut assez peu s'exprimer à ce sujet sans borne...}

\addVE{Remarque sur la complexité $\mathcal{O}(p+q \log q)$: on a $\log q$ facteurs dont un de taille $p\times q$ (ou l'inverse) a $\mathcal{O}(p)$ valeurs non-nulles et tous les autres de taille $q \times q$ ont $\mathcal{O}(q)$ valeurs non-nulles.}

%\section{Submission of papers to NeurIPS 2019}
%
%NeurIPS requires electronic submissions.  The electronic submission site is
%\begin{center}
%  \url{https://cmt.research.microsoft.com/NeurIPS2019/}
%\end{center}
%
%Please read the instructions below carefully and follow them faithfully.
%
%\subsection{Style}
%
%Papers to be submitted to NeurIPS 2019 must be prepared according to the
%instructions presented here. Papers may only be up to eight pages long,
%including figures. Additional pages \emph{containing only acknowledgments and/or
%  cited references} are allowed. Papers that exceed eight pages of content
%(ignoring references) will not be reviewed, or in any other way considered for
%presentation at the conference.
%
%The margins in 2019 are the same as since 2007, which allow for $\sim$$15\%$
%more words in the paper compared to earlier years.
%
%Authors are required to use the NeurIPS \LaTeX{} style files obtainable at the
%NeurIPS website as indicated below. Please make sure you use the current files
%and not previous versions. Tweaking the style files may be grounds for
%rejection.
%
%\subsection{Retrieval of style files}
%
%The style files for NeurIPS and other conference information are available on
%the World Wide Web at
%\begin{center}
%  \url{http://www.neurips.cc/}
%\end{center}
%The file \verb+neurips_2019.pdf+ contains these instructions and illustrates the
%various formatting requirements your NeurIPS paper must satisfy.
%
%The only supported style file for NeurIPS 2019 is \verb+neurips_2019.sty+,
%rewritten for \LaTeXe{}.  \textbf{Previous style files for \LaTeX{} 2.09,
%  Microsoft Word, and RTF are no longer supported!}
%
%The \LaTeX{} style file contains three optional arguments: \verb+final+, which
%creates a camera-ready copy, \verb+preprint+, which creates a preprint for
%submission to, e.g., arXiv, and \verb+nonatbib+, which will not load the
%\verb+natbib+ package for you in case of package clash.
%
%\paragraph{Preprint option}
%If you wish to post a preprint of your work online, e.g., on arXiv, using the
%NeurIPS style, please use the \verb+preprint+ option. This will create a
%nonanonymized version of your work with the text ``Preprint. Work in progress.''
%in the footer. This version may be distributed as you see fit. Please \textbf{do
%  not} use the \verb+final+ option, which should \textbf{only} be used for
%papers accepted to NeurIPS.
%
%At submission time, please omit the \verb+final+ and \verb+preprint+
%options. This will anonymize your submission and add line numbers to aid
%review. Please do \emph{not} refer to these line numbers in your paper as they
%will be removed during generation of camera-ready copies.
%
%The file \verb+neurips_2019.tex+ may be used as a ``shell'' for writing your
%paper. All you have to do is replace the author, title, abstract, and text of
%the paper with your own.
%
%The formatting instructions contained in these style files are summarized in
%Sections \ref{gen_inst}, \ref{headings}, and \ref{others} below.
%
%\section{General formatting instructions}
%\label{gen_inst}
%
%The text must be confined within a rectangle 5.5~inches (33~picas) wide and
%9~inches (54~picas) long. The left margin is 1.5~inch (9~picas).  Use 10~point
%type with a vertical spacing (leading) of 11~points.  Times New Roman is the
%preferred typeface throughout, and will be selected for you by default.
%Paragraphs are separated by \nicefrac{1}{2}~line space (5.5 points), with no
%indentation.
%
%The paper title should be 17~point, initial caps/lower case, bold, centered
%between two horizontal rules. The top rule should be 4~points thick and the
%bottom rule should be 1~point thick. Allow \nicefrac{1}{4}~inch space above and
%below the title to rules. All pages should start at 1~inch (6~picas) from the
%top of the page.
%
%For the final version, authors' names are set in boldface, and each name is
%centered above the corresponding address. The lead author's name is to be listed
%first (left-most), and the co-authors' names (if different address) are set to
%follow. If there is only one co-author, list both author and co-author side by
%side.
%
%Please pay special attention to the instructions in Section \ref{others}
%regarding figures, tables, acknowledgments, and references.
%
%\section{Headings: first level}
%\label{headings}
%
%All headings should be lower case (except for first word and proper nouns),
%flush left, and bold.
%
%First-level headings should be in 12-point type.
%
%\subsection{Headings: second level}
%
%Second-level headings should be in 10-point type.
%
%\subsubsection{Headings: third level}
%
%Third-level headings should be in 10-point type.
%
%\paragraph{Paragraphs}
%
%There is also a \verb+\paragraph+ command available, which sets the heading in
%bold, flush left, and inline with the text, with the heading followed by 1\,em
%of space.
%
%\section{Citations, figures, tables, references}
%\label{others}
%
%These instructions apply to everyone.
%
%\subsection{Citations within the text}
%
%The \verb+natbib+ package will be loaded for you by default.  Citations may be
%author/year or numeric, as long as you maintain internal consistency.  As to the
%format of the references themselves, any style is acceptable as long as it is
%used consistently.
%
%The documentation for \verb+natbib+ may be found at
%\begin{center}
%  \url{http://mirrors.ctan.org/macros/latex/contrib/natbib/natnotes.pdf}
%\end{center}
%Of note is the command \verb+\citet+, which produces citations appropriate for
%use in inline text.  For example,
%\begin{verbatim}
%   \citet{hasselmo} investigated\dots
%\end{verbatim}
%produces
%\begin{quote}
%  Hasselmo, et al.\ (1995) investigated\dots
%\end{quote}
%
%If you wish to load the \verb+natbib+ package with options, you may add the
%following before loading the \verb+neurips_2019+ package:
%\begin{verbatim}
%   \PassOptionsToPackage{options}{natbib}
%\end{verbatim}
%
%If \verb+natbib+ clashes with another package you load, you can add the optional
%argument \verb+nonatbib+ when loading the style file:
%\begin{verbatim}
%   \usepackage[nonatbib]{neurips_2019}
%\end{verbatim}
%
%As submission is double blind, refer to your own published work in the third
%person. That is, use ``In the previous work of Jones et al.\ [4],'' not ``In our
%previous work [4].'' If you cite your other papers that are not widely available
%(e.g., a journal paper under review), use anonymous author names in the
%citation, e.g., an author of the form ``A.\ Anonymous.''
%
%\subsection{Footnotes}
%
%Footnotes should be used sparingly.  If you do require a footnote, indicate
%footnotes with a number\footnote{Sample of the first footnote.} in the
%text. Place the footnotes at the bottom of the page on which they appear.
%Precede the footnote with a horizontal rule of 2~inches (12~picas).
%
%Note that footnotes are properly typeset \emph{after} punctuation
%marks.\footnote{As in this example.}
%
%\subsection{Figures}
%
%\begin{figure}
%  \centering
%  \fbox{\rule[-.5cm]{0cm}{4cm} \rule[-.5cm]{4cm}{0cm}}
%  \caption{Sample figure caption.}
%\end{figure}
%
%All artwork must be neat, clean, and legible. Lines should be dark enough for
%purposes of reproduction. The figure number and caption always appear after the
%figure. Place one line space before the figure caption and one line space after
%the figure. The figure caption should be lower case (except for first word and
%proper nouns); figures are numbered consecutively.
%
%You may use color figures.  However, it is best for the figure captions and the
%paper body to be legible if the paper is printed in either black/white or in
%color.
%
%\subsection{Tables}
%
%All tables must be centered, neat, clean and legible.  The table number and
%title always appear before the table.  See Table~\ref{sample-table}.
%
%Place one line space before the table title, one line space after the
%table title, and one line space after the table. The table title must
%be lower case (except for first word and proper nouns); tables are
%numbered consecutively.
%
%Note that publication-quality tables \emph{do not contain vertical rules.} We
%strongly suggest the use of the \verb+booktabs+ package, which allows for
%typesetting high-quality, professional tables:
%\begin{center}
%  \url{https://www.ctan.org/pkg/booktabs}
%\end{center}
%This package was used to typeset Table~\ref{sample-table}.
%
%\begin{table}
%  \caption{Sample table title}
%  \label{sample-table}
%  \centering
%  \begin{tabular}{lll}
%    \toprule
%    \multicolumn{2}{c}{Part}                   \\
%    \cmidrule(r){1-2}
%    Name     & Description     & Size ($\mu$m) \\
%    \midrule
%    Dendrite & Input terminal  & $\sim$100     \\
%    Axon     & Output terminal & $\sim$10      \\
%    Soma     & Cell body       & up to $10^6$  \\
%    \bottomrule
%  \end{tabular}
%\end{table}
%
%\section{Final instructions}
%
%Do not change any aspects of the formatting parameters in the style files.  In
%particular, do not modify the width or length of the rectangle the text should
%fit into, and do not change font sizes (except perhaps in the
%\textbf{References} section; see below). Please note that pages should be
%numbered.
%
%\section{Preparing PDF files}
%
%Please prepare submission files with paper size ``US Letter,'' and not, for
%example, ``A4.''
%
%Fonts were the main cause of problems in the past years. Your PDF file must only
%contain Type 1 or Embedded TrueType fonts. Here are a few instructions to
%achieve this.
%
%\begin{itemize}
%
%\item You should directly generate PDF files using \verb+pdflatex+.
%
%\item You can check which fonts a PDF files uses.  In Acrobat Reader, select the
%  menu Files$>$Document Properties$>$Fonts and select Show All Fonts. You can
%  also use the program \verb+pdffonts+ which comes with \verb+xpdf+ and is
%  available out-of-the-box on most Linux machines.
%
%\item The IEEE has recommendations for generating PDF files whose fonts are also
%  acceptable for NeurIPS. Please see
%  \url{http://www.emfield.org/icuwb2010/downloads/IEEE-PDF-SpecV32.pdf}
%
%\item \verb+xfig+ "patterned" shapes are implemented with bitmap fonts.  Use
%  "solid" shapes instead.
%
%\item The \verb+\bbold+ package almost always uses bitmap fonts.  You should use
%  the equivalent AMS Fonts:
%\begin{verbatim}
%   \usepackage{amsfonts}
%\end{verbatim}
%followed by, e.g., \verb+\mathbb{R}+, \verb+\mathbb{N}+, or \verb+\mathbb{C}+
%for $\mathbb{R}$, $\mathbb{N}$ or $\mathbb{C}$.  You can also use the following
%workaround for reals, natural and complex:
%\begin{verbatim}
%   \newcommand{\RR}{I\!\!R} %real numbers
%   \newcommand{\Nat}{I\!\!N} %natural numbers
%   \newcommand{\CC}{I\!\!\!\!C} %complex numbers
%\end{verbatim}
%Note that \verb+amsfonts+ is automatically loaded by the \verb+amssymb+ package.
%
%\end{itemize}
%
%If your file contains type 3 fonts or non embedded TrueType fonts, we will ask
%you to fix it.
%
%\subsection{Margins in \LaTeX{}}
%
%Most of the margin problems come from figures positioned by hand using
%\verb+\special+ or other commands. We suggest using the command
%\verb+\includegraphics+ from the \verb+graphicx+ package. Always specify the
%figure width as a multiple of the line width as in the example below:
%\begin{verbatim}
%   \usepackage[pdftex]{graphicx} ...
%   \includegraphics[width=0.8\linewidth]{myfile.pdf}
%\end{verbatim}
%See Section 4.4 in the graphics bundle documentation
%(\url{http://mirrors.ctan.org/macros/latex/required/graphics/grfguide.pdf})
%
%A number of width problems arise when \LaTeX{} cannot properly hyphenate a
%line. Please give LaTeX hyphenation hints using the \verb+\-+ command when
%necessary.
%
%\subsubsection*{Acknowledgments}
%
%Use unnumbered third level headings for the acknowledgments. All acknowledgments
%go at the end of the paper. Do not include acknowledgments in the anonymized
%submission, only in the final paper.
%
%\section*{References}
%
%References follow the acknowledgments. Use unnumbered first-level heading for
%the references. Any choice of citation style is acceptable as long as you are
%consistent. It is permissible to reduce the font size to \verb+small+ (9 point)
%when listing the references. {\bf Remember that you can use more than eight
%  pages as long as the additional pages contain \emph{only} cited references.}
%\medskip
%
%\small
%
%[1] Alexander, J.A.\ \& Mozer, M.C.\ (1995) Template-based algorithms for
%connectionist rule extraction. In G.\ Tesauro, D.S.\ Touretzky and T.K.\ Leen
%(eds.), {\it Advances in Neural Information Processing Systems 7},
%pp.\ 609--616. Cambridge, MA: MIT Press.
%
%[2] Bower, J.M.\ \& Beeman, D.\ (1995) {\it The Book of GENESIS: Exploring
%  Realistic Neural Models with the GEneral NEural SImulation System.}  New York:
%TELOS/Springer--Verlag.
%
%[3] Hasselmo, M.E., Schnell, E.\ \& Barkai, E.\ (1995) Dynamics of learning and
%recall at excitatory recurrent synapses and cholinergic modulation in rat
%hippocampal region CA3. {\it Journal of Neuroscience} {\bf 15}(7):5249-5262.

\end{document}
