\section{Preliminaries}
\label{sec:background}

In this section, we begin by introducing some notation and then review the basics of K-means and give some background about learning fast transforms.

\subsection{Notations}


%!TEX root=neurips2019_qmeans.tex
 
%\paragraph{Notations}


%%%%%%%%%%%%%%%%%%%%%%%%%%%%%%%%%%%%%%%%%%%%%%%%%%%%%%%%%%%%
\begin{table}[t]
	\centering
	\begin{footnotesize}
	\begin{tabular}{ll}\\
\toprule
		{\bf Symbol}  & {\bf Meaning}\\
\midrule
$\intint{M}$  & set of integers from $1$ to $M$\\
$\|\cdot\|$ & $L_2$-norm\\
$\|\cdot\|_F$ &    Frobenius norm  \\
$\|\cdot\|_0$ & $L_0$-norm\\
$\|\cdot\|_2$    &    spectral norm  \\
$\rmD_\rvv$ & diagonal matrix with vector $\rvv$ on the diagonal\\                                                
$N$           & number of data points\\
$D$           & data dimension\\
$K$           & number of clusters\\
$Q$           & number of sparse factors\\
$\rvx_1,\ldots, \rvx_N $        &    data points\\
$\rmX \in\mathbb{R}^{N\times D}$&    data matrix\\
$\rvt$        &  cluster assignment vector\\
$\rvu_1,\ldots, \rvu_K $        &    \kmeans centers\\
$\rmU\in\mathbb{R}^{K\times D}$ &    \kmeans center matrix\\
$\rvv_1,\ldots, \rvv_K $        &    \qkmeans centers\\
$\rmV\in\mathbb{R}^{K\times D}$ &    \qkmeans center matrix\\
$\rmS_1, \ldots, \rmS_Q$        &    sparse matrices\\
$\mathcal{E}_1, \ldots, \mathcal{E}_Q$ & sparsity constraint sets\\
$\delta_{\mathcal{E}}$ & 		indicator functions for set $\mathcal{E}$\\
$\tau$  & current iteration \\
\bottomrule
	\end{tabular}
	\end{footnotesize}
	\caption{Notation used in this paper.}
	\label{tab:notation}
\end{table}
%\begin{table}[t]
%	\centering
%	\begin{footnotesize}
%	\begin{tabular}{cllcl}\\
%		\cline{1-2}\cline{4-5}\vspace*{1mm}
%		{\bf Symbol}  & {\bf Meaning}                      &  &    {\bf Symbol}          & {\bf Meaning}                    \\ 		\cline{1-2}\cline{4-5}
%		$N$           & number of data points              &  &    $\rvx_1,\ldots, \rvx_N $        &    data points            \\
%		$D$           & data dimension &  &    $\rmX \in\mathbb{R}^{N\times D}$&    data matrix            \\
%		$K$           & number of clusters                 &  &    $\rvu_1,\ldots, \rvu_K $        &    \kmeans centroids        \\
%		$\rvt$        &  cluster assignment vector           &  &    $\rmU\in\mathbb{R}^{K\times D}$ &    \kmeans centroid matrix  \\
%		&                 &  &    $\rvv_1,\ldots, \rvv_K $        &    \qkmeans centroids        \\
%		&          &  &    $\rmV\in\mathbb{R}^{K\times D}$ &    \qkmeans centroid matrix  \\
%		$Q$           & number of sparse factors    &  &    $\rmS_1, \ldots, \rmS_Q$        &    sparse matrices        \\
%		$\|\cdot\|$, & $L_2$-norm&  &    $\|\cdot\|_F$, &    Frobenius norm  \\
%		$\|\cdot\|_0$ & $L_0$-norm&  &    $\|\cdot\|_2$    &    spectral norm  \\
%		$\mathcal{E}_1, \ldots, \mathcal{E}_Q$ & sparsity constraint sets           &  & $\delta_{\mathcal{E}}$ & 		indicator functions for set $\mathcal{E}$\\
%		$\intint{M}$  & set of integers from $1$ to $M$ &  & $\tau$  &                       		current iteration  \\
%		
%		$\rmD_\rvv$ & diagonal matrix with vector $\rvv$ on the diagonal\\                                                          		\cline{1-2}\cline{4-5}        \\      
%	\end{tabular}
%	\end{footnotesize}
%	\caption{Notation used in this paper.}
%	\label{tab:notation}
%\end{table}
%\addtocounter{footnote}{0}
%\footnotetext{We also use the standard notations such as $\mathbb{R}^n$ and $\mathbb{M}_n$.}
%%%%%%%%%%%%%%%%%%%%%%%%%%%%%%%%%%%%%%%%%%%%%%%%%%%%%%%%%%%%




%%%%%%%%%%%%%%%%%%%%%%%%%%%%%%%%%%%%%%%%%%%%%%%%%%%%%%%%%%%%%
%\begin{table}[t]
%	\centering
%	\begin{tabular}{|r|c|l|}
%		\hline
%		indices &  $i$, $j$, $m$, $n$, $p$, $q$ &  small  Latin characters  \\
%		other integers &  $K$, $Q$, $N$, $\ldots$ &  capital  Latin characters \\
%	%	vector spaces\footnotemark & $\mathcal{X}$, $\mathcal{Y}$, $\mathcal{H}$, $\ldots$ & Calligraphic letters \\ 
%		vectors (or functions) & $\rvx$, $\rvt$, $\rvk$, $\ldots$ & small bold Latin characters \\
%		matrices  & $\rmX$, $\rmU$, $\rmK$, $\ldots$ & capital bold Latin characters \\
%		transpose & $\top$ & $\rmX^\top$ transpose of  $\rmX$ \\
%		\hline
%	\end{tabular}
%	\caption{Notations used in this paper.}
%	\label{tab:notation}
%\end{table}
%\addtocounter{footnote}{0}
%\footnotetext{We also use the standard notations such as $\mathbb{R}^n$ and $\mathbb{M}_n$.}
%%%%%%%%%%%%%%%%%%%%%%%%%%%%%%%%%%%%%%%%%%%%%%%%%%%%%%%%%%%%%


%The notations frequently used in the paper are summarized in Table~\ref{tab:notation}. 
%%
%Throughout the paper we use $\nexamples$ as the number of data samples and $\datadim$ the dimensionality of a data point. 
%$\rmX \in \R^{\nexamples \times \datadim}$ is the data matrix. 
%For $K \in \sN$, we define $\intint{K}=\left \lbrace k\in \sN: 1 \leq k \leq K\right \rbrace$.
%%
%For a given vector $\rvv$, $\rvv[i]$ is the $i$th component of $\rvv$.
%%
%For a given matrix $\rmM$, the notation $\rmM_{[i]}$ (resp. $\rmM^{[i]}$) refers to the $i$th row (column) of $\rmM$, the entry at the $i$th row and the $j$th column is denoted by $\rmM[i,j]$, and $\|\rmM\|_F$ denotes the Frobenius norm, $\|\rmM\|_2$ the spectral norm and $\|\rmM\|_0$ counts the number of non-zero entries in $\rmM$. \addHK{other norms?}
%
%
%
%
%\todo[inline]{The text is redundant with the table. In addition, we should remove the "small Latin character0", "capital Latin characters" as they do not provide any meaning. We should prefer the trick with the transpose.}

%%%%%%%%%%%%%%%%%%%%%%%%%%%%%%%%%%%%%%%%%%%%%%%%%%%%%%%%%%%%
\begin{table}[t]
	\centering
	\begin{tabular}{|r|c|l|}
		\hline
		indices &  $i$, $j$, $m$, $n$, $p$, $q$ &  small  Latin characters  \\
		other integers &  $K$, $Q$, $N$, $\ldots$ &  capital  Latin characters \\
	%	vector spaces\footnotemark & $\mathcal{X}$, $\mathcal{Y}$, $\mathcal{H}$, $\ldots$ & Calligraphic letters \\ 
		vectors (or functions) & $\rvx$, $\rvt$, $\rvk$, $\ldots$ & small bold Latin characters \\
		matrices  & $\rmX$, $\rmU$, $\rmK$, $\ldots$ & capital bold Latin characters \\
		transpose & $\top$ & $\rmX^\top$ transpose of  $\rmX$ \\
		\hline
	\end{tabular}
	\caption{Notations used in this paper.}
	\label{tab:notation}
\end{table}
\addtocounter{footnote}{0}
\footnotetext{We also use the standard notations such as $\mathbb{R}^n$ and $\mathbb{M}_n$.}
%%%%%%%%%%%%%%%%%%%%%%%%%%%%%%%%%%%%%%%%%%%%%%%%%%%%%%%%%%%%






\subsection{K-means}
\label{sec:kmeans}
The K-means algorithm is used to partition a given set of observations $\rmX$ into a predefined $K$ clusters while minimizing the distance between the observations in each partition:

\begin{equation}
\label{eq:kmean_problem}
    \argmin_{\rmU, \rvt} \sum_{k=1}^{K} \sum_{j: \rvt_j = k} ||\rmX_j -\rmU_k||^2,
\end{equation}
% autre écriture de l'objectif de k-means
% = \argmin_{\rmU, \rvt} \sum_{k=1}^{K} c_k + \sum_{k=1}^{K} n_k||\hat{\rmU}_k - \rmU_k||^2
where $\rmU \in \R^{K \times d}$ is the matrix of the cluster's center-points and $\rvt \in  [\![K]\!]^n$ is the indicator vector.

The algorithm (Algorithm \ref{algo:kmeans}) starts with an initialized set of $K$ center-points ($\{\rmU_i \in \R^d\}_{i=1}^{K}$). Each update step $\tau$ is divided into two parts: (i) all observations $\rmX_i$ are assigned to their nearest cluster based on the center-points $\rmU_i^{(\tau-1)}$s at this step (Line \ref{line:kmeans:assignment}) in $\mathcal{O}(ndK)$ operations. (ii) the new center-points $\rmU_i^{(\tau)}$s are computed as the means of the assignated $\rmX_i$ (Line \ref{line:kmeans:compute_means}) for a total of $\mathcal{O}(nd)$ operations.

\begin{algorithm}
\caption{K-means algorithm}
\label{algo:kmeans}
\begin{algorithmic}[1]


\REQUIRE $\rmX \in \R^{n \times d}$, $K$, $\{\rmU_i \in \R^d\}_{i=1}^{K}$
\ENSURE $\{\rmU_i\}_{i=1}^{K}$ the K means of $n$ $d$-dimensional samples
\STATE $\tau \leftarrow 0$
\REPEAT
\STATE $\tau \leftarrow \tau + 1$
\STATE $\rvt^{(\tau)} \leftarrow \argmin_{\rvt \in [\![K]\!]^n} \sum_{i=1}^{n} {||\rmX_i - \rmU^{(\tau-1)}_{\rvt_i}||_2^2}$
\label{line:kmeans:assignment}
\FORALL {$k \in [\![K]\!]^n$}
\STATE $n_k^{(\tau)} \leftarrow |\{i: \rvt^{(\tau)}_i=k\}|$
\label{line:kmeans:count}
\STATE $\rmU^{(\tau)}_k \leftarrow \frac{1}{n_k^{(\tau)}} \sum_{i: \rvt^{(\tau)}_i = k} {\rmX_i}$
\label{line:kmeans:compute_means}
\ENDFOR
\UNTIL{stop criterion}
\RETURN $\rmU^{(\tau)}$
\end{algorithmic}
\end{algorithm}


Once the clusters have been defined, for any $\rvx \in \R^d$ the cluster associated with this $\rvx$ is:

\begin{equation}
\label{eq:assignment_problem_kmeans}
\argmin_{k \in [\![K]\!]} ||\rvx - \rmU_{k}||_2^2 = \argmin_{k \in [\![K]\!]} ||\rmU_{k}||_2^2 - 2 \rmU_{k}^T\rvx
\end{equation}.


We remark here that the computational bottleneck of this assignment lies in the computation of $\rmU_k^T\rvx$ for all $k$. This computation is also encountered in the assignment step (line \ref{line:kmeans:assignment}) of the Algorithm \ref{algo:kmeans}.


\subsection{Learning Fast transforms as the product of sparse matrices}
\label{sec:palm4led}


\begin{algorithm}
	\caption{PALM4MSA algorithm}
	\label{algo:palm4msa}
	\begin{algorithmic}[1]
		
		
		\REQUIRE The matrix to factorize $\rmU \in \R^{K \times d}$, the desired number of factors $Q$, the constraint sets $\mathcal{E}_j$ , $j \in [\![Q]\!]$ and a stopping criterion (e.g., here, a number of iterations $N_{iter}$ ).
		
		\ENSURE $\{\mathcal{S}_1 \dots \mathcal{S}_{Q}\}|\mathcal{S}_j \in \mathcal{E}_j$ such that $\prod_{j=1}^{Q}\mathcal{S}_j \approx \rmU$
		
		\FOR {$i = 0$ to $N_{iter}$}
		\FOR {$j = 1$ to $Q$}
		\STATE  $\rmL_j \leftarrow \prod_{l=j+1}^{Q} \mathcal{S}_{l}^{i}$
		\STATE  $\rmR_j \leftarrow \prod_{l=1}^{j-1} \mathcal{S}_{l}^{i+1}$
		\STATE $c_j^i :> (\lambda^i)^2 ||\rmR_j||_2^2 ||\rmL_j||_2^2$
		\STATE $\mathcal{S}^{i+1}_j \leftarrow P_{\mathcal{E}_j}(\mathcal{S}_j^i - \frac{1}{c_j^i} \lambda^i \rmL_j^T(\lambda \rmL_j \mathcal{S}_j^i \rmR_j - \rmU)\rmR_j^T)$
		\ENDFOR
		\STATE $\hat \rmU := \prod_{j=1}^{Q} \mathcal{S}_j^{i+1}$
		\STATE $\lambda^{i+1} \leftarrow \frac{Trace(\rmU^T\hat\rmU)}{Trace(\hat\rmU^T\hat\rmU)}$
		\ENDFOR
		
		\ENSURE $\lambda, \{\mathcal{S}_1 \dots \mathcal{S}_{Q}\}|\mathcal{S}_j \in \mathcal{E}_j$ such that $\lambda \prod_{j=1}^{Q}\mathcal{S}_j \approx \rmU$
		
	\end{algorithmic}
\end{algorithm}



A popular way for providing concise description of high-dimensional vectors $\rmU \in \R^{K \times d}$ is to compute a sparse representation using a dictionary:
%
\begin{equation}
\rmU^T \approx \rmD\rmGamma
\end{equation}
%
where $\rmD \in \R^{d \times d}$ is a dictionary and $\rmGamma \in \R^{d \times K}$ has sparse columns. Historically, the dictionary is either (i) analytic: $\rmD$ is chosen to give a fast reconstruction of the initial matrix by taking advantage of some fast-transform algorithm (the \textit{Fast Hadamard Transform} for instance) or (ii) learned: $\rmD$ is learned from the data itself to give a good reconstruction of the initial matrix.

Building on the observation that the fast-transform associated with an analytic dictionary can be expressed as the product of sparse matrices $\mathcal{S}_j$ from a set $\mathcal{S}$ of size $M$, \cite{magoarou2014learning} proposes an algorithm to learn a dictionary from the data with sparsity constraints such that this dictionary would be both well-suited with the data and fast to use:
%
\begin{equation}
\rmD = \lambda \prod_{j=1}^{M}\mathcal{S}_j
\end{equation}
%
with $\forall j \in \{1 \ldots M\}$, $\mathcal{S}_j \in \mathcal{E}_j$, $\mathcal{E}_j = \lbrace \rmA \in \R^{a \times a+1}~\text{s.t.}~||\rmA||_0^0 \leq p_j, ||\rmA||_1 = 1 \rbrace$ and $p_j$ being chosen suitably. The $\lambda$ parameter has been added along with the normalization constraint in the $\mathcal{E}_j$ in order to remove scaling ambiguity in the learned $\mathcal{S}_j$.

Considering $\rmGamma$ being a sparse matrice too, it can be renamed as $\rmGamma = \mathcal{S}_{M+1}$. We set $Q = M+1$ and the overall dictionary learning problem can be expressed as the following optimisation problem:
%
\begin{equation}
\label{eq:problem_gribon}
\min_{\{\mathcal{S}_1 \dots \mathcal{S}_Q, \lambda\}} ||\rmU - \lambda \prod_{j=1}^{Q}{\mathcal{S}_j}||_2^2 + \sum_{j=1}^{Q} \delta_j(\mathcal{S}_j)
\end{equation}
%
with the $\delta_j(\mathcal{S}_j) = 0$ if $\mathcal{S}_j \in \mathcal{E}_j$ being the sparsity constraints to satisfy on the associated $\mathcal{S}_j$.

Although this problem is highly non-convex, the authors derive an algorithm from the PALM algorithm \cite{bolte2014proximal}, which they call \textit{Hierarchical PALM4LED} to find a good local minima and give convergence guarantees to learn efficient dictionaries.



