%!TEX root=aaai2020_qmeans.tex

\section{\texttt{QuicK-means}}
\label{sec:contribution}

We here introduce our main contribution, \texttt{QuicK-means} (abbreviated by \qkmeans), 
show its convergence property and analyze its computational complexity.

% \todo[inline]{WATCH: CENTROIDS vs CENTERS... We have specified that we would consider
% centers from the get-go...}

\subsection{\qkmeans: Encoding Centroids as Products of Sparse Matrices}
\label{sec:qkmeans:algo}

\texttt{QuicK-means} is a variant of the \kmeans algorithm in which the matrix of centers $\rmU$
is approximated as a product $\rmV=\prod_{\in\intint{\nfactors}}\rmS_q$ of sparse matrices $\rmS_q$.
Doing so will allow us to cope with the computational bulk imposed by the product $\rmU\rvx$
---obtained by a simple rewriting of~\eqref{eq:assignment_problem_kmeans}--- that shows up in the cluster assignment 
process.

Building upon the \kmeans optimization problem~\eqref{eq:kmean_problem} and fast-operator approximation problem~\eqref{eq:palm4msa} the \qkmeans optimization problem 
writes:
%
\begin{align}
%\label{eq:qmean_problem}
 \argmin_{\left\{\rmS_q\right\}_{q=1}^{\nfactors}, \rvt} g\left(\left\{\rmS_q\right\}_{q=1}^{\nfactors}, \rvt\right)
 \eqdef & \sum_{k\in\intint{\nclusters}} \sum_{\substack{n\\t_n = k}} \left\|\rvx_n -\rvv_k\right\|^2 \nonumber \\
 & + \sum_{q\in\intint{\nfactors}} \delta_{\mathcal{E}_q}(\rmS_q)
 \label{eq:objective}
\end{align}
%\begin{align}
%\label{eq:qmean_problem}
% \argmin_{\left\{\rmS_q\right\}_{q=1}^{\nfactors}, \rvt} & g\left(\left\{\rmS_q\right\}_{q=1}^{\nfactors}, \rvt\right)\\
% \intertext{with}
%  g\left(\left\{\rmS_q\right\}_{q=1}^{\nfactors}, \rvt\right)&:=  \sum_{k\in\intint{\nclusters}}\sum_{\substack{n\\t_n = k}} \left\|\rvx_n -\rvv_k\right\|^2 + \sum_{q\in\intint{\nfactors}} \delta_{\mathcal{E}_q}(\rmS_q)\label{eq:objective}
%\end{align}
where $\rmV = \prod_{q\in\intint{\nfactors}}\rmS_q$.
%\begin{equation}
%	\label{eq:centers}
%	\rmV = \prod_{q\in\intint{\nfactors}}\rmS_q
%\end{equation}
%\begin{align}
%\label{eq:qmean_problem}
% \argmin_{\left\{\rmS_q\right\}_{q=1}^{\nfactors}, \rvt} g\left(\left\{\rmS_q\right\}_{q=1}^{\nfactors}, \rvt\right)
%\end{align}
%where
%\begin{align}\small
%g\left(\left\{\rmS_q\right\}_{q=1}^{\nfactors}, \rvt\right) \eqdef & \sum_{k\in\intint{\nclusters}} \sum_{\substack{n\\t_n = k}} \left\|\rvx_n -\rvv_k\right\|^2 + \sum_{q\in\intint{\nfactors}} \delta_{\mathcal{E}_q}(\rmS_q)
% \label{eq:objective}
%\end{align}

%
This is a constrained version of the \kmeans optimization problem~\eqref{eq:kmean_problem} in which centers $\rvv_k$ are constrained to form a matrix $\rmV$ with a fast-operator structure, the indicator functions $\delta_{\mathcal{E}_q}$ imposing the sparsity of matrices $\rmS_q$.
More details on the modeling choices are given in the experimental part. % in section~\ref{sec:uses:settings}.

This problem can be solved using Algorithm~\ref{algo:qmeans},
which proceeds in a similar way as Lloyd's algorithm by alternating an assignment step at line \ref{line:qmeans:assignment} and an update of the centers at lines~\ref{line:qmeans:compute_means}--\ref{line:qmeans:U}. The assignment step can be computed efficiently thanks to the fast-structure of $\rmV$. The update of the centers relies on learning a fast-structure operator $\rmV$ that approximate of the true center matrix $\rmU$ weighted by the number of examples $n_k$ assigned to each cluster $k$.


\begin{algorithm*}[t]
	\caption{\qkmeans algorithm and its time complexity. Here $A \eqdef \min\left (\nclusters, \datadim\right )$ and $B \eqdef \max\left (\nclusters, \datadim\right )$}
	\label{algo:qmeans}
	\begin{algorithmic}[1]
\REQUIRE $\rmX \in \R^{\nexamples \times \datadim}$, $\nclusters$, initialization $\left \lbrace \rmS_q^{(0)} : \rmS_q^{(0)} \in \mathcal{E}_q\right \rbrace_{q\in\intint{\nfactors}}$
%\COMMENT{$A \eqdef \min\left (\nclusters, \datadim\right )$}
%\STATE $\rmV^{(0)} \eqdef \prod_{q\in\intint{\nfactors}}{\rmS_q^{(0)}}$
\STATE Set $\rmV^{(0)} : \rvx \mapsto \prod_{q\in\intint{\nfactors}}{\rmS_q^{(0)}} \rvx$
%\COMMENT{$B \eqdef \max\left (\nclusters, \datadim\right )$}
\FOR{$\tau=1, 2, \ldots$ until convergence}
	\STATE $\rvt^{(\tau)} \eqdef \argmin_{\rvt \in \intint{\nclusters}^\nexamples} \sum_{n\in\intint{\nexamples}} {\left \|\rvx_n - \rvv^{(\tau -1)}_{t_n}\right \|^2}$
	\COMMENT{$\mathcal{O}\left (\nexamples\left(A\log A+B\right ) + AB\right )$}
	\label{line:qmeans:assignment}
	\STATE $\forall k\in\intint{\nclusters}, \rvu_k \eqdef \frac{1}{n_k} \sum_{n: t_n^{(\tau)}= k} {\rvx_n}$
with $n_k \eqdef |\{n: t_n^{(\tau)}=k\}|$
	\COMMENT{$\bigO{\nexamples\datadim}$}
	\label{line:qmeans:compute_means}
	\STATE $\rmA \eqdef \rmD_{\sqrt{\rvn}} \times \rmU $
	\COMMENT{$\bigO{\nclusters\datadim}$}
	\label{line:qmeans:A}
	\STATE $\mathcal{E}_0 \eqdef \left \lbrace \rmD_{\sqrt{\rvn}} \right \rbrace$
	\label{line:qmeans:E0}
	\STATE $\left \lbrace \rmS_q^{(\tau)}\right \rbrace_{q=0}^\nfactors \eqdef \argmin_{\left \lbrace \rmS_q\right \rbrace_{q=0}^\nfactors} \left \|\rmA - \prod_{q=0}^\nfactors\rmS_q\right \|_F^2 + \sum_{q=0}^\nfactors \delta_{\mathcal{E}_q}(\rmS_q)$
	\COMMENT{$\bigO{AB\left (\log^2 A+\log B\right )}$} % (or $\bigO{AB\left (\log^3A+\log A \log B\right )}$)}
	\label{line:qmeans:S}
	\STATE Set $\rmV^{(\tau)} : \rvx \mapsto \prod_{q\in\intint{\nfactors}}{\rmS_q^{(\tau)}} \rvx$
	\COMMENT{$\bigO{1}$}
	\label{line:qmeans:U}
	\ENDFOR
	\ENSURE assignement vector $\rvt$ and sparse matrices $\left \lbrace \rmS_q : \rmS_q \in \mathcal{E}_q\right \rbrace_{q\in\intint{\nfactors}}$ % such that $\prod_{q\in\intint{\nfactors}}\rmS_q \approx \rmU$ the $\nclusters$ means of the $\nexamples$ data points
\end{algorithmic}
\end{algorithm*}

\iffalse
\begin{remark}[Assignment/Re-estimation trade-off.]
A strategy to tackle this problem would be to first run the vanilla K-means algorithm,
 obtain the matrix of centers $U$ and then encode $U$ as a product of sparse matrices
 using Hierarchical Palm4MSA. This would however prevent us from taking advantage of 
 the expected low complexity product that plays a role in the assignement step of 
 the procedure.
\end{remark}

\todo[inline]{At some point, talk about the trade-off that we are playing with
regarding the cost of the assignment and the cost of the re-estimation procedure.}
\fi

\subsection{Convergence of \qkmeans}
\label{sec:qkmeans:convergence}
Similarly to \kmeans, \qkmeans converges locally as stated in the following proposition.

\begin{proposition}
\label{thm:convergence}
The iterates $\left \lbrace\rmS^{(\tau)} \right \rbrace_{q\in\intint{\nfactors}}$ and $\rvt^{(\tau)}$ in Algorithm~\ref{algo:qmeans} are such that, $\forall \tau\geq 1$
\begin{equation*}
g\left(\left\{\rmS_q^{(\tau+1)}\right\}_{q=1}^{\nfactors}, \rvt^{(\tau+1)}\right)
\leq g\left(\left\{\rmS_q^{(\tau)}\right\}_{q=1}^{\nfactors}, \rvt^{(\tau)}\right),
\end{equation*}
which guarantees the convergence (CHECK THAT STATEMENT).

\iffalse
\begin{equation}
\begin{split}
\label{eq:qmean_problem_2}
    g(\rmS_1^{(\tau)}, \ldots,\rmS_\nfactors^{(\tau)}, \rvt^{(\tau)} ) & \\
    = \sum_{k\in\intint{\nclusters}} \sum_{n: \rvt^{(\tau)}_n = k} & \norm{\rvx_n - \rvv^{(\tau)}_k}^2 + \sum_{q\in\intint{\nfactors}} \delta_{\mathcal{E}_q}\left (\rmS_q^{(\tau)}\right ) \\
    \text{ s.t. } \rmV = \prod_{q\in\intint{\nfactors}} & {\rmS_q^{(\tau)}}
\end{split} %
\end{equation}
\fi
\end{proposition}

\todo[inline]{remove the CHECK THAT STATEMENT in proposition}


\begin{proof}
We show that each step of the algorithm does not increase the overall objective function. 

\paragraph{Assignment step (Line \ref{line:qmeans:assignment})} For a fixed $\rmV^{(\tau-1)}$, the optimization problem at Line \ref{line:qmeans:assignment} is separable for each example indexed by $n \in \intint{\nexamples}$ and the new indicator vector $\rvt^{(\tau)}$ is thus defined as:
%
\begin{align}
\label{eq:qmean_problem_U_fixed}
 t^{(\tau)}_n = \argmin_{k \in \intint{\nclusters}} \norm{\rvx_n - \rvv_k^{(\tau-1)}}_2^2.
\end{align}
%
This step minimizes the first term in~\eqref{eq:objective} w.r.t. $\rvt$ while the second term is constant so we have 
\begin{align*}
g\left (\left\{\rmS_q^{(\tau-1)}\right\}_{q=1}^{\nfactors}, \rvt^{(\tau)}\right ) \leq g\left (\left\{\rmS_q^{(\tau-1)}\right\}_{q=1}^{\nfactors}, \rvt^{(\tau-1)}\right ).
\end{align*}
%\begin{align*}
%g(\rmS_1^{(\tau-1)}, \ldots,\rmS_\nfactors^{(\tau-1)}, \rvt^{(\tau)}) \leq g(\rmS_1^{(\tau-1)}, \ldots,\rmS_\nfactors^{(\tau-1)}, \rvt^{(\tau-1)}).
%\end{align*}

\paragraph{Centers update step (Lines \ref{line:qmeans:compute_means}--\ref{line:qmeans:U}).} We now consider a fixed assignment vector $\rvt$. We first note that for any cluster $k$ with true center $\rvu_k$ and approximated center $\rvv_k$, we have, by simple calculations
\begin{align}
    \label{eq:rewrite_assignement}
	\sum_{\substack{n\\t_n = k}} \norm{\rvx_n -\rvv_k}^2
%	 & =\sum_{\substack{n\\t_n = k}} \norm{\rvx_n -\rvu_k+\rvu_k-\rvv_k}^2 \notag\\
%	& = \sum_{\substack{n\\t_n = k}}\left(\norm{\rvx_n-\rvu_k}^2+\norm{\rvu_k-\rvv_k}^2 \right. \notag\\
%	& \qquad \left. \vphantom{\norm{\rvx_n-\rvu_k}^2+\norm{\rvu_k-\rvv_k}^2} - 2\langle\rvx_n-\rvu_k,\rvu_k-\rvv_k\rangle \right)\notag\\
%    & = \sum_{\substack{n\\t_n = k}} \norm{\rvx_n-\rvu_k}^2+n_k\norm{\rvu_k-\rvv_k}^2 \notag\\
 %   & \qquad - 2 \left\langle\underbrace{\sum_{\substack{n\\t_n = k}}\left (\rvx_n-\rvu_k\right )}_{=0},\rvu_k-\rvv_k\right\rangle\notag \\
%	&= \sum_{\substack{n\\t_n = k}} \norm{\rvx_n-\rvu_k}^2 + \norm{\sqrt{n_k}\left (\rvu_k-\rvv_k\right )}^2
	&= \sum_{\substack{n\\t_n = k}} \norm{\rvx_n-\rvu_k}^2 + n_k \norm{\rvu_k-\rvv_k}^2
\end{align}



For a fixed $\rvt$, the new sparsely-factorized centers are solutions of the subproblem $\argmin_{\rmS_1, \ldots,\rmS_Q} g(\rmS_1, \ldots,\rmS_Q, \rvt)$ which can be rewritten thanks to~\eqref{eq:rewrite_assignement}:
%
%\begin{align}
%\begin{split}
%\argmin_{\rmS_1, \ldots,\rmS_Q} & g(\rmS_1, \ldots,\rmS_Q, \rvt) \\
%= \argmin_{\rmS_1, \ldots,\rmS_Q} & \sum_{k\in\intint{\nclusters}}  \sum_{\substack{n\\t_n = k}} \norm{\rvx_n - \rvv_k}^2_2 + \sum_{q\in\intint{\nfactors}} \delta_q(\rmS_q) \\
%& \text{ s. t. } \rmV = \prod_{q\in\intint{\nfactors}}{\rmS_q} \nonumber \\
% \end{split}
%\end{align}



\begin{align}
 \argmin_{\rmS_1, \ldots,\rmS_Q} & \norm{\rmD_{\sqrt{\rvn}} (\rmU - \rmV)}_F^2 + \sum_{k\in\intint{\nclusters}} c_k + \sum_{q\in\intint{\nfactors}} \delta_q(\rmS_q) \notag \\
 & \text{ s. t. } \rmV = \prod_{q\in\intint{\nfactors}}{\rmS_q} \nonumber\\
 = \argmin_{\rmS_1, \ldots,\rmS_Q} & \norm{\rmA - \prod_{q\in\left\{\intint{\nfactors} \cup \{0\}\right\}}{\rmS_q}}_F^2 + \sum_{q\in\left\{\intint{\nfactors} \cup \{0\}\right\}} \delta_q(\rmS_q) \notag \\
 \label{eq:qmeans_problem_t_fixed}
 \end{align}
%
where:
%
\begin{itemize}
 \item $\sqrt{\rvn} \in \R^{\nclusters}$ is the pair-wise square root of the vector indicating the number of observations $n_k \eqdef \left | \left \lbrace n: t_n = k\right \rbrace \right |$  in each cluster $k$;
 \item $\rmD_{\sqrt{\rvn}} \in \R^{K \times K}$ refers to a diagonal matrix with vector $\sqrt{\rvn}$ on the diagonal;
 \item $\rmU\in \R^{K \times d}$ refers to the unconstrained center matrix obtained from the data matrix $\rmX$ and the indicator vector $\rvt$: $\rvu_k \eqdef \frac{1}{n_k}\sum_{n:t_n = k} {\rvx_n}$ (see Line~\ref{line:qmeans:compute_means});
 \item $\rmD_{\sqrt{\rvn}} (\rmU - \rmV)$ is the matrix with $\sqrt{n_k}\left (\rvu_k-\rvv_k\right )$ as row~$k$;
 \item $c_k \eqdef \sum_{\substack{n\\t_n = k}}\norm{\rvx_n - \rvu_k}$ is constant w.r.t. $ \rmS_1, \ldots,\rmS_Q$;
 \item $\rmA \eqdef \rmD_{\sqrt{\rvn}} \rmU$ is the unconstrained center matrix reweighted by the size of each cluster (see Line~\ref{line:qmeans:A}).
\end{itemize}

We note that the problem~\eqref{eq:qmeans_problem_t_fixed} has exactly the form of~\eqref{eq:palm4msa} hence the \palm algorithm is applied to obtain an approximation of $\rmA$ in a local minimum, as in Line~\ref{line:qmeans:S}. The first factor $\rmS_0$ is forced to equal $\rmD_{\sqrt{\rvn}}$ by setting $\mathcal{E}_0$ to a singleton at Line~\ref{line:qmeans:E0}. Using the previous estimates $\left \lbrace \rmS_q^{(\tau-1)}\right \rbrace_{q\in\intint{\nfactors}}$ to initialize this local minimization, we thus obtain that $g(\rmS_1^{(\tau)}, \ldots,\rmS_\nfactors^{(\tau)}, \rvt^{(\tau)}) \leq g(\rmS_1^{(\tau-1)}, \ldots,\rmS_\nfactors^{(\tau-1)}, \rvt^{(\tau)})$.
We finally have, for any $\tau$:
\begin{align*}
%g\left (\left \lbrace \rmS_q^{(\tau)}\right \rbrace_{q\in\intint{\nfactors}}, \rvt^{(\tau)}\right ) & \leq
%g\left (\left \lbrace \rmS_q^{(\tau-1)}\right \rbrace_{q\in\intint{\nfactors}}, \rvt^{(\tau)}\right ) \leq
%g\left (\left \lbrace \rmS_q^{(\tau-1)}\right \rbrace_{q\in\intint{\nfactors}}, \rvt^{(\tau-1)}\right ) \\
%& \leq
%\ldots \leq
%g\left (\left \lbrace \rmS_q^{(0)}\right \rbrace_{q\in\intint{\nfactors}}, \rvt^{(0)}\right )
%\\
g\left (\rmS_1^{(\tau)}, \ldots,\rmS_\nfactors^{(\tau)}, \rvt^{(\tau)}\right ) & \leq g\left (\rmS_1^{(\tau-1)}, \ldots,\rmS_\nfactors^{(\tau-1)}, \rvt^{(\tau)}\right ) \\
& \leq g\left (\rmS_1^{(\tau-1)}, \ldots,\rmS_\nfactors^{(\tau-1)}, \rvt^{(\tau-1)}\right ) \\
& \leq \ldots \leq g\left (\rmS_1^{(0)}, \ldots,\rmS_\nfactors^{(0)}, \rvt^{(0)}\right )
\end{align*}
which is a sufficient condition to assert the Algorithm~\ref{algo:qmeans} is a \textit{convergent} alternating algorithm.

\end{proof}


\subsection{Complexity analysis}
\label{sec:qkmeans:complexity}

Since the space complexity of the proposed \qkmeans algorithm is comparable to that of \kmeans, we only detail its time complexity. We set $A=\min\left (\nclusters, \datadim\right )$ and $B=\max\left (\nclusters, \datadim\right )$, and assume that the number of factors satisfies $\nfactors=\bigO{\log A}$.
The analysis is proposed under the following assumptions: the product between two dense matrices of shapes ${N_1\times N_2}$ and ${N_2\times N_3}$ can be done $\mathcal{O}\left (N_1 N_2 N_3 \right )$ operations; 
the product between a sparse matrix with $\bigO{S}$ non-zero entries and a dense vector can be done in $\bigO{S}$ operations; 
the product between two sparse matrices of shapes ${N_1\times N_2}$ and ${N_2\times N_3}$, both having $\bigO{S}$ non-zero values can be done in $\bigO{S \min\left (N_1, N_3\right )}$ and the number of non-zero entries in the resulting matrix is $\bigO{S^2}$.


\paragraph{Complexity of the \kmeans algorithm.}
We recall here that the \kmeans algorithm complexity is dominated by its cluster assignation step which requires $\bigO{\nexamples\nclusters\datadim}=\bigO{\nexamples A B}$ operations (see Eq.~\eqref{eq:assignment_problem_kmeans}).

\paragraph{Complexity of algorithm \palm.} The procedure consists in an alternate optimization of each sparse factor. 
At each iteration, the whole set of $\nfactors$ factors is updated with at a cost in $\bigO{AB\left (\log^2 A+\log B\right )}$, as detailed in the supplemental material (Section ~\ref{sec:app:palm4msa}). 
The bottleneck is the computation of the gradient, which benefits from fast computations with sparse matrices.
%The hierarchical version of \palm proposed in~\cite{LeMagoarou2016Flexible} consists in running $\palm$ $2Q$ times so that its time complexity is in $\bigO{AB\left (\log^3 A + \log A \log B\right )}$.


\paragraph{Complexity of the \qkmeans algorithm.} The overall complexity of \qkmeans is in $\bigO{\nexamples\left(A\log A+B\right ) + AB \log^2 A}$. The time complexities of the main steps are given in Algorithm~\ref{algo:qmeans}. 

The assignation step (line~\ref{line:qmeans:assignment} and Eq.~\eqref{eq:assignment_problem_kmeans}) benefits from the fast computation of $\rmV \rmX$ in~$\bigO{\nexamples\left(A\log A+B\right )}$ while the computation of the norms of the cluster centers is in $\bigO{AB}$.
One can see that the computational bottleneck of \kmeans is here reduced, which shows the advantage of using \qkmeans when $\nexamples$, $\nclusters$ and $\datadim$ are large.

The computation of the centers of each cluster at line~\ref{line:qmeans:compute_means} is the same as in \kmeans and takes $\bigO{\nexamples\datadim}$ operations.

The update of the fast transform, in lines~\ref{line:qmeans:A} to~\ref{line:qmeans:U} is a computational overload compared to \kmeans. 

Its time complexity is dominated by the update of the sparse factors at line~\ref{line:qmeans:S}, in $\bigO{AB \log^2 A}$. 
Note that this cost is dominated by the cost of the assignement step as soon as the number of examples $\nexamples$ is greater than $\log^3 A$. 

