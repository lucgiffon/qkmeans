\section{Conclusion}
\label{sec:conclusion}

In this paper, we have proposed a variant of the \kmeans algorithm, named \qkmeans, designed to achieve a similar goal -- clustering data points around $\nclusters$ learned centers -- with a much lower computational complexity as the dimension of the data, the number of examples and the number of clusters get high. 
Our approach is based on the approximation of the matrix of centers by an operator structured as a product of a small number of sparse matrices, resulting in a low time and space complexity when applied to data vectors.
We have shown the convergence properties of the proposed algorithm and provided its complexity analysis.

An implementation prototype has been run in several core machine learning use cases including clustering, nearest-neighbor search and Nystr\"om approximation. 
The experimental results illustrate the computational gain in high dimension at inference time as well as the good approximation qualities of the proposed model. \todo[inline]{pas besoin de preciser la haute dimension}

Beyond these modeling, algorithmic and experimental contributions to low-complexity high-dimensional machine learning, we have identified several important questions that are still to be addressed.
First, although learning the fast-structure operator has been nicely integrated in the training algorithm with an advantageous theoretical time and space complexity, exhibiting gains in actual running times has not been achieved yet for the \qkmeans learning procedure, compared to \kmeans.
This may be obtained in even higher dimensions than in the proposed experimental settings, which may require a new version of \qkmeans using batches of data in order to process amounts of data that do not fit in memory.
\todo[inline]{voir s'il est possible de montrer ces gains en terme de flop}
Second, the expressiveness of the fast-structure model is still to be theoretically studied, while our experiments seems to show that arbitrary matrices may be well fitted by such models.
Third, we believe that learning fast-structure linear operators during the training procedure may be generalized to many core machine learning methods in order to speed them up and make them scale to larger dimensions.

