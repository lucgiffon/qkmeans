%!TEX root=neurips2019_qmeans.tex
\section{Preliminaries}
\label{sec:background}
We briefly review the basics of \kmeans and
 give background on learning fast transforms.
 %
  To  assist  the  reading,  we  list  the notations used in the paper in Table~\ref{tab:notation}.

%\subsection{Notations}


%!TEX root=neurips2019_qmeans.tex
 
%\paragraph{Notations}


%%%%%%%%%%%%%%%%%%%%%%%%%%%%%%%%%%%%%%%%%%%%%%%%%%%%%%%%%%%%
\begin{table}[t]
	\centering
	\begin{footnotesize}
	\begin{tabular}{ll}\\
\toprule
		{\bf Symbol}  & {\bf Meaning}\\
\midrule
$\intint{M}$  & set of integers from $1$ to $M$\\
$\|\cdot\|$ & $L_2$-norm\\
$\|\cdot\|_F$ &    Frobenius norm  \\
$\|\cdot\|_0$ & $L_0$-norm\\
$\|\cdot\|_2$    &    spectral norm  \\
$\rmD_\rvv$ & diagonal matrix with vector $\rvv$ on the diagonal\\                                                
$N$           & number of data points\\
$D$           & data dimension\\
$K$           & number of clusters\\
$Q$           & number of sparse factors\\
$\rvx_1,\ldots, \rvx_N $        &    data points\\
$\rmX \in\mathbb{R}^{N\times D}$&    data matrix\\
$\rvt$        &  cluster assignment vector\\
$\rvu_1,\ldots, \rvu_K $        &    \kmeans centers\\
$\rmU\in\mathbb{R}^{K\times D}$ &    \kmeans center matrix\\
$\rvv_1,\ldots, \rvv_K $        &    \qkmeans centers\\
$\rmV\in\mathbb{R}^{K\times D}$ &    \qkmeans center matrix\\
$\rmS_1, \ldots, \rmS_Q$        &    sparse matrices\\
$\mathcal{E}_1, \ldots, \mathcal{E}_Q$ & sparsity constraint sets\\
$\delta_{\mathcal{E}}$ & 		indicator functions for set $\mathcal{E}$\\
$\tau$  & current iteration \\
\bottomrule
	\end{tabular}
	\end{footnotesize}
	\caption{Notation used in this paper.}
	\label{tab:notation}
\end{table}
%\begin{table}[t]
%	\centering
%	\begin{footnotesize}
%	\begin{tabular}{cllcl}\\
%		\cline{1-2}\cline{4-5}\vspace*{1mm}
%		{\bf Symbol}  & {\bf Meaning}                      &  &    {\bf Symbol}          & {\bf Meaning}                    \\ 		\cline{1-2}\cline{4-5}
%		$N$           & number of data points              &  &    $\rvx_1,\ldots, \rvx_N $        &    data points            \\
%		$D$           & data dimension &  &    $\rmX \in\mathbb{R}^{N\times D}$&    data matrix            \\
%		$K$           & number of clusters                 &  &    $\rvu_1,\ldots, \rvu_K $        &    \kmeans centroids        \\
%		$\rvt$        &  cluster assignment vector           &  &    $\rmU\in\mathbb{R}^{K\times D}$ &    \kmeans centroid matrix  \\
%		&                 &  &    $\rvv_1,\ldots, \rvv_K $        &    \qkmeans centroids        \\
%		&          &  &    $\rmV\in\mathbb{R}^{K\times D}$ &    \qkmeans centroid matrix  \\
%		$Q$           & number of sparse factors    &  &    $\rmS_1, \ldots, \rmS_Q$        &    sparse matrices        \\
%		$\|\cdot\|$, & $L_2$-norm&  &    $\|\cdot\|_F$, &    Frobenius norm  \\
%		$\|\cdot\|_0$ & $L_0$-norm&  &    $\|\cdot\|_2$    &    spectral norm  \\
%		$\mathcal{E}_1, \ldots, \mathcal{E}_Q$ & sparsity constraint sets           &  & $\delta_{\mathcal{E}}$ & 		indicator functions for set $\mathcal{E}$\\
%		$\intint{M}$  & set of integers from $1$ to $M$ &  & $\tau$  &                       		current iteration  \\
%		
%		$\rmD_\rvv$ & diagonal matrix with vector $\rvv$ on the diagonal\\                                                          		\cline{1-2}\cline{4-5}        \\      
%	\end{tabular}
%	\end{footnotesize}
%	\caption{Notation used in this paper.}
%	\label{tab:notation}
%\end{table}
%\addtocounter{footnote}{0}
%\footnotetext{We also use the standard notations such as $\mathbb{R}^n$ and $\mathbb{M}_n$.}
%%%%%%%%%%%%%%%%%%%%%%%%%%%%%%%%%%%%%%%%%%%%%%%%%%%%%%%%%%%%




%%%%%%%%%%%%%%%%%%%%%%%%%%%%%%%%%%%%%%%%%%%%%%%%%%%%%%%%%%%%%
%\begin{table}[t]
%	\centering
%	\begin{tabular}{|r|c|l|}
%		\hline
%		indices &  $i$, $j$, $m$, $n$, $p$, $q$ &  small  Latin characters  \\
%		other integers &  $K$, $Q$, $N$, $\ldots$ &  capital  Latin characters \\
%	%	vector spaces\footnotemark & $\mathcal{X}$, $\mathcal{Y}$, $\mathcal{H}$, $\ldots$ & Calligraphic letters \\ 
%		vectors (or functions) & $\rvx$, $\rvt$, $\rvk$, $\ldots$ & small bold Latin characters \\
%		matrices  & $\rmX$, $\rmU$, $\rmK$, $\ldots$ & capital bold Latin characters \\
%		transpose & $\top$ & $\rmX^\top$ transpose of  $\rmX$ \\
%		\hline
%	\end{tabular}
%	\caption{Notations used in this paper.}
%	\label{tab:notation}
%\end{table}
%\addtocounter{footnote}{0}
%\footnotetext{We also use the standard notations such as $\mathbb{R}^n$ and $\mathbb{M}_n$.}
%%%%%%%%%%%%%%%%%%%%%%%%%%%%%%%%%%%%%%%%%%%%%%%%%%%%%%%%%%%%%


%The notations frequently used in the paper are summarized in Table~\ref{tab:notation}. 
%%
%Throughout the paper we use $\nexamples$ as the number of data samples and $\datadim$ the dimensionality of a data point. 
%$\rmX \in \R^{\nexamples \times \datadim}$ is the data matrix. 
%For $K \in \sN$, we define $\intint{K}=\left \lbrace k\in \sN: 1 \leq k \leq K\right \rbrace$.
%%
%For a given vector $\rvv$, $\rvv[i]$ is the $i$th component of $\rvv$.
%%
%For a given matrix $\rmM$, the notation $\rmM_{[i]}$ (resp. $\rmM^{[i]}$) refers to the $i$th row (column) of $\rmM$, the entry at the $i$th row and the $j$th column is denoted by $\rmM[i,j]$, and $\|\rmM\|_F$ denotes the Frobenius norm, $\|\rmM\|_2$ the spectral norm and $\|\rmM\|_0$ counts the number of non-zero entries in $\rmM$. \addHK{other norms?}
%
%
%
%
%\todo[inline]{The text is redundant with the table. In addition, we should remove the "small Latin character0", "capital Latin characters" as they do not provide any meaning. We should prefer the trick with the transpose.}





\subsection{\kmeans}
\label{sec:kmeans}
The \kmeans algorithm is used to partition a set $\rmX=\{\rvx_1,\ldots,\rvx_N\}$ of $N$  vectors $\rvx_n\in\R^{\datadim}$ into a predefined number $\nclusters$ of clusters
with the aim of minimizing the distance between each $\rvx_n$ to the center $\rvu_k\in\R^{D}$
of the cluster $k$ it belongs to ---the center $\rvu_k$ of cluster $k$ is the
 mean vector of the points assigned to cluster $k$.
\kmeans attempts to solve
\begin{equation}
\label{eq:kmean_problem}
    \argmin_{\rmU, \rvt} \sum_{k\in\intint{\nclusters}} \sum_{n: t_n = k} \|\rvx_{n} -\rvu_{k}\|^2,
\end{equation}
% autre écriture de l'objectif de k-means
% = \argmin_{\rmU, \rvt} \sum_{k=1}^{K} c_k + \sum_{k=1}^{K} n_k||\hat{\rmU}_k - \rmU_k||^2
where $\rmU=\{\rvu_1,\ldots,\rvu_K\}$ is the set of cluster centers and $\rvt \in  \intint{\nclusters}^{\nexamples}$ is the assignment vector that puts $\rvx_n$ in cluster $k$
if $t_n=k$.


\paragraph{Lloyd's algorithm.} The most popular procedure to (approximately) 
solve the \kmeans problem is the iterative Lloyds algorithm, which alternates
i) an assignment step that decides the current cluster to which each point $\rvx_n$
belongs and ii) a reestimation step which refines the clusters and their centers.
In little more detail, the algorithm starts with an initialized set of $\nclusters$
 cluster centers $\rmU^{(0)}$ and proceeds as follows: at iteration $\tau$,
  the assignments are updated as
\begin{align}
\label{eq:assignment_problem_kmeans}
\forall n\in\intint{N}, t_n^{(\tau)} \leftarrow \argmin_{k \in \intint{\nclusters]}} \left\|\rvx_{n} - \rvu_{k}^{(\tau-1)}\right\|_2^2 = \argmin_{k \in \intint{\nclusters}} \left\|\rvu_{k}^{(\tau-1)}\right\|_2^2 - 2 \left\langle\rvu_{k}^{(\tau-1)}, \rvx_{n}\right\rangle,
\end{align}
 the reestimation of the cluster centers is performed as
\begin{align}
\label{eq:center_update}
\forall k\in\intint{K}, \rvu^{(\tau)}_k \leftarrow \hat{\rvx}_k(\rvt^{(\tau)}) \eqdef \frac{1}{n_k^{(\tau)}} \sum_{n: t^{(\tau)}_n= k} {\rvx_{n}}
\end{align}
where $n_k^{(\tau)}\eqdef |\{n: t^{(\tau)}_n=k\}|$ is the number of points in cluster $k$
at time $\tau$ and $\hat{\rvx}_k(\rvt)$ is the mean vector of the elements of cluster $k$ according to assignment $\rvt$. %at a total cost of $\mathcal{O}\left (\nexamples\datadim\right )$ operations.

\paragraph{Complexity of Lloyd's algorithm.} The assignment step \eqref{eq:assignment_problem_kmeans} costs $\mathcal{O}(\nexamples\datadim\nclusters)$ operations while the update of the centers~\eqref{eq:center_update} costs $\mathcal{O}\left (\nexamples\datadim\right )$ operations. Hence, the bottleneck of the overall time complexity $\mathcal{O}(\nexamples\datadim\nclusters)$ stems from the assignment step. Once the clusters have been defined, assigning $\nexamples'$ new points to these clusters is performed via \eqref{eq:assignment_problem_kmeans} at the cost of $\mathcal{O}\left (\nexamples'\datadim\nclusters \right )$ operations.

The main contribution in this paper relies on the idea that \eqref{eq:assignment_problem_kmeans} may be computed more efficiently by approximating $\rmU$ as a fast operator.

% Each update step $\tau$ is divided into two parts: (i) all observations $\rmX_{[n]}$ are assigned to their nearest cluster based on the center-points $\rmU_{[\rvt[n]]}^{(\tau-1)}$s at this step (Line \ref{line:kmeans:assignment}) in $\mathcal{O}(\nexamples\datadim\nclusters)$ operations
%
% (ii) the new center-points $\rmU_{[k]}^{(\tau)}$s are computed as the means of the assignated $\rmX_{[n]}$ (Line \ref{line:kmeans:compute_means}) for a total of $\mathcal{O}\left (\nexamples\datadim\right )$ operations.

%\begin{algorithm}
%\caption{\kmeans algorithm}
%\label{algo:kmeans}
%\begin{algorithmic}[1]
%
%
%\REQUIRE $\rmX \in \R^{\nexamples \times \datadim}$, $\nclusters$, $\{\rmU_i \in \R^\datadim\}_{i=1}^{\nclusters}$
%\ENSURE $\{\rmU_i\}_{i=1}^{\nclusters}$ the K means of $\nexamples$ $\datadim$-dimensional samples
%\STATE $\tau \leftarrow 0$
%\REPEAT
%\STATE $\tau \leftarrow \tau + 1$
%\STATE $\rvt^{(\tau)} \leftarrow \argmin_{\rvt \in [\![\nclusters]\!]^n} \sum_{i=1}^{\nexamples} {||\rmX_i - \rmU^{(\tau-1)}_{\rvt_i}||_2^2}$
%\label{line:kmeans:assignment}
%\FORALL {$k \in [\![\nclusters]\!]^\nexamples$}
%\STATE $n_k^{(\tau)} \leftarrow |\{i: \rvt^{(\tau)}_i=k\}|$
%\label{line:kmeans:count}
%\STATE $\rmU^{(\tau)}_k \leftarrow \frac{1}{n_k^{(\tau)}} \sum_{i: \rvt^{(\tau)}_i = k} {\rmX_i}$
%\label{line:kmeans:compute_means}
%\ENDFOR
%\UNTIL{stop criterion}
%\RETURN $\rmU^{(\tau)}$
%\end{algorithmic}
%\end{algorithm}


%Once the clusters have been defined, for any $\rvx \in \R^\datadim$ the cluster associated with this $\rvx$ is:
%
%\begin{equation}
%\label{eq:assignment_problem_kmeans}
%\argmin_{k \in [\![\nclusters]\!]} ||\rvx - \rmU_{[k]}||_2^2 = \argmin_{k \in [\![\nclusters]\!]} ||\rmU_{[k]}||_2^2 - 2 \rmU_{[k]}^T\rvx.
%\end{equation}
%
%
%We remark here that the computational bottleneck of this assignment lies in the computation of $\rmU_{[k]}^T\rvx$ for all $k$. This computation is also encountered in the assignment step (line \ref{line:kmeans:assignment}) of the Algorithm \ref{algo:kmeans}.


\subsection{Learning Fast Transforms as the Product of Sparse Matrices}
\label{sec:palm4msa}

\paragraph{Structured linear operators as products of sparse matrices.}
The popularity of some linear operators from $\R^{M}$ to $\R^{M}$ (with $M<\infty$)
 like Fourier or Hadamard transforms comes from both their mathematical 
 properties and their ability to compute the mapping of some input $\rvx\in\R^M$ with efficiency, typically in $\mathcal{O}\left (M\log\left (M\right )\right )$ rather than 
 in $\mathcal{O}\left (M^2\right)$ operations .
The main idea of the related fast algorithms is that the matrix $\rmU\in\sR^{M\times M}$ characterizing such linear operators can be written as the product $\rmU=\Pi_{q\in\intint{\nfactors}}\rmS_q$ of $\nfactors$ sparse matrices $\rmS_q$, with $Q=\mathcal{O}\left (\log M\right )$ factors and $\left \|\rmS_q\right \|_0=\mathcal{O}\left (M\right )$ non-zero coefficients per factor \cite{LeMagoarou2016Flexible,Morgenstern1975Linear}:
for any vector $\rvx\in\sR^M$, $\rmU\rvx$ can thus be computed as $\mathcal{O}\left (\log M\right )$ products $\rmS_0 \left (\rmS_1 \left (\ldots \left (\rmS_{Q-1}\rvx\right )\right )\right )$ between a sparse matrix and a vector, the cost of each product being $\mathcal{O}\left (M\right )$. This gives a $\mathcal{O}(M \log M)$ time complexity for computing $\rmU\rvx$ in that case.

\paragraph{Learning a computationally-efficient decomposition approximating an arbitrary operator.} When the linear operator $\rmU$ is an arbitrary matrix, one may approximate it with such a sparse-product structure by learning the factors $\left \lbrace\rmS_q\right \rbrace_{q\in\intint{Q}}$ in order to benefit from a fast algorithm.
A recent contribution~\cite{LeMagoarou2016Flexible} has proposed algorithmic strategies to learn such a factorization. Based on the proximal alternating linearized minimization (\texttt{PALM}) algorithm~\cite{bolte2014proximal}, the \texttt{PALM} for Multi-layer Sparse Approximation (\palm) algorithm~\cite{LeMagoarou2016Flexible} aims at approximating a matrix $\rmU\in\sR^{\nclusters\times\datadim}$ as a product of sparse matrices by solving
\begin{align}
\label{eq:palm4msa}
\min_{\left \lbrace\rmS_q\right \rbrace_{q\in\intint{Q}}} \left \|\rmU -  \prod_{q\in\intint{\nfactors}}{\rmS_q}\right \|_F^2 + \sum_{q\in\intint{\nfactors}} \delta_{\mathcal{E}_q}(\rmS_q)
\end{align}
where, for each $q\in\intint{Q}$, $\delta_{\mathcal{E}_q}(\rmS_q)=0$ 
if $\rmS_q \in \mathcal{E}_q$ and $\delta_{\mathcal{E}_q}(\rmS_q)=+\infty$ otherwise, $\mathcal{E}_q$ being a constraint set that typically impose a sparsity structure on its elements, as well as a scaling constraint. The \palm algorithm and more related details are given in Appendix~\ref{sec:app:palm4msa}.

\iffalse
TO BE COMPLETED + algo~\ref{algo:palm4msa} in appendix~\ref{sec:app:palm4msa}.

A popular way for providing concise description of high-dimensional vectors $\rmU \in \R^{K \times d}$ is to compute a sparse representation using a dictionary:
%
\begin{equation}
\rmU^T \approx \rmD\rmGamma
\end{equation}
%
where $\rmD \in \R^{d \times d}$ is a dictionary and $\rmGamma \in \R^{d \times K}$ has sparse columns. Historically, the dictionary is either (i) analytic: $\rmD$ is chosen to give a fast reconstruction of the initial matrix by taking advantage of some fast-transform algorithm (the \textit{Fast Hadamard Transform} for instance) or (ii) learned: $\rmD$ is learned from the data itself to give a good reconstruction of the initial matrix.

Building on the observation that the fast-transform associated with an analytic dictionary can be expressed as the product of sparse matrices $\mathcal{S}_j$ from a set $\mathcal{S}$ of size $M$, \cite{magoarou2014learning} proposes an algorithm to learn a dictionary from the data with sparsity constraints such that this dictionary would be both well-suited with the data and fast to use:
%
\begin{equation}
\rmD = \lambda \prod_{j=1}^{M}\mathcal{S}_j
\end{equation}
%
with $\forall j \in \{1 \ldots M\}$, $\mathcal{S}_j \in \mathcal{E}_j$, $\mathcal{E}_j = \lbrace \rmA \in \R^{a \times a+1}~\text{s.t.}~||\rmA||_0^0 \leq p_j, ||\rmA||_1 = 1 \rbrace$ and $p_j$ being chosen suitably. The $\lambda$ parameter has been added along with the normalization constraint in the $\mathcal{E}_j$ in order to remove scaling ambiguity in the learned $\mathcal{S}_j$.

Considering $\rmGamma$ being a sparse matrice too, it can be renamed as $\rmGamma = \mathcal{S}_{M+1}$. We set $Q = M+1$ and the overall dictionary learning problem can be expressed as the following optimisation problem:
%
\begin{equation}
\label{eq:problem_gribon}
\min_{\{\mathcal{S}_1 \dots \mathcal{S}_Q, \lambda\}} ||\rmU - \lambda \prod_{j=1}^{Q}{\mathcal{S}_j}||_2^2 + \sum_{j=1}^{Q} \delta_j(\mathcal{S}_j)
\end{equation}
%
with the $\delta_j(\mathcal{S}_j) = 0$ if $\mathcal{S}_j \in \mathcal{E}_j$ being the sparsity constraints to satisfy on the associated $\mathcal{S}_j$.
\fi

Although this problem is non-convex and the computation of a global optimum cannot be
ascertained, the \palm algorithm is able to find good local minima with convergence guarantees. 


