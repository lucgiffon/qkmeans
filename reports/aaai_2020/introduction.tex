%!TEX root=aaai2020_qmeans.tex
\section{Introduction}

\kmeans is one of the most popular clustering algorithms~\cite{hartigan1979algorithm,jain2010data} and it can be used beyond clustering, for tasks such as indexing, data compression,  nearest-neighbor search and prediction, and local network community detection~\cite{muja2014scalable,van2016local}.
\kmeans is also a pivotal process to increase the speed and the accuracy of learning procedures, e.g., for kernel machines~\cite{si2016computationally} and RBF networks~\cite{que2016back}, when combined with the Nyström approximation.
%
The  conventional  \kmeans  algorithm, i.e.  Lloyd's algorithm, has  a $\bigO{\nexamples\nclusters\datadim}$  complexity  per iteration when learning $\nclusters$ clusters from $\nexamples$ data points in dimension $\datadim$.
In addition, the larger the number of clusters, the more iterations are needed to converge~\cite{arthur2006slow}.
%
As data dimensionality and sample size grow, it is critical to have at hand cost-effective 
alternatives to the computationally expensive conventional \kmeans. 
Known strategies to alleviate its computational issues rely on batch-, sparsity- and randomization-based methods~\cite{Sculley2010Web,boutsidis2014randomized,shen2017compressed,liu2017sparse}.
%Settings with a large sample size in high dimension require
%computationally-efficient alternatives to the conventional \kmeans. 
%Known strategies rely on batch-, sparsity- and randomization-based methods~\cite{Sculley2010Web,boutsidis2014randomized,shen2017compressed,liu2017sparse}.

Fast transforms have recently received increased attention in machine learning as they can speed up random projections~\cite{le2013fastfood,gittens2016revisiting} and to improve landmark-based approximations~\cite{si2016computationally}.
%
These works focused on fixed fast transforms such as well-known Fourier and Hadamard transforms.
A question is whether one can go beyond and learn fast transforms that fit the data. 
%
Recently, \citet{LeMagoarou2016Flexible} introduced a sparse matrix approximation scheme aimed  at  reducing the  complexity  of  applying  linear  operators  in  high  dimension by   approximately   factorizing   the   corresponding   matrix   into few   sparse   factors. 
One interesting observation is that the aforementioned fixed fast transforms can be factorized into few sparse matrices.
%
In this paper, we take this idea further and investigate computationally less costly variants of \kmeans by learning fast transforms from data.
%
Specifically, we make the following contributions:
\begin{itemize}
	\item we introduce \qkmeans, a fast extension of \kmeans that rests on learning a fast transform that approximate the matrix of centers;
	\item we show that each update step in one iteration of our algorithm  reduces the overall objective, which is enough to guarantee the convergence of \qkmeans;
	\item we provide a complexity analysis of \qkmeans, showing that the computational gain has a direct impact in assigning a point to a cluster, which is not only tangible at prediction time, but also at training time;
	\item we provide an empirical evaluation of \qkmeans  performance which demonstrates its effectiveness on different datasets in the contexts of clustering, nearest neighbor search and kernel Nystr\"om approximation.
\end{itemize}
\todo[inline]{Annoncer le plan}


